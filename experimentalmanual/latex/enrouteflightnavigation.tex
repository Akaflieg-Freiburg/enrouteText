%% Generated by Sphinx.
\def\sphinxdocclass{report}
\documentclass[letterpaper,10pt,english]{sphinxmanual}
\ifdefined\pdfpxdimen
   \let\sphinxpxdimen\pdfpxdimen\else\newdimen\sphinxpxdimen
\fi \sphinxpxdimen=.75bp\relax

\PassOptionsToPackage{warn}{textcomp}
\usepackage[utf8]{inputenc}
\ifdefined\DeclareUnicodeCharacter
% support both utf8 and utf8x syntaxes
  \ifdefined\DeclareUnicodeCharacterAsOptional
    \def\sphinxDUC#1{\DeclareUnicodeCharacter{"#1}}
  \else
    \let\sphinxDUC\DeclareUnicodeCharacter
  \fi
  \sphinxDUC{00A0}{\nobreakspace}
  \sphinxDUC{2500}{\sphinxunichar{2500}}
  \sphinxDUC{2502}{\sphinxunichar{2502}}
  \sphinxDUC{2514}{\sphinxunichar{2514}}
  \sphinxDUC{251C}{\sphinxunichar{251C}}
  \sphinxDUC{2572}{\textbackslash}
\fi
\usepackage{cmap}
\usepackage[T1]{fontenc}
\usepackage{amsmath,amssymb,amstext}
\usepackage{babel}



\usepackage{times}
\expandafter\ifx\csname T@LGR\endcsname\relax
\else
% LGR was declared as font encoding
  \substitutefont{LGR}{\rmdefault}{cmr}
  \substitutefont{LGR}{\sfdefault}{cmss}
  \substitutefont{LGR}{\ttdefault}{cmtt}
\fi
\expandafter\ifx\csname T@X2\endcsname\relax
  \expandafter\ifx\csname T@T2A\endcsname\relax
  \else
  % T2A was declared as font encoding
    \substitutefont{T2A}{\rmdefault}{cmr}
    \substitutefont{T2A}{\sfdefault}{cmss}
    \substitutefont{T2A}{\ttdefault}{cmtt}
  \fi
\else
% X2 was declared as font encoding
  \substitutefont{X2}{\rmdefault}{cmr}
  \substitutefont{X2}{\sfdefault}{cmss}
  \substitutefont{X2}{\ttdefault}{cmtt}
\fi


\usepackage[Bjornstrup]{fncychap}
\usepackage{sphinx}

\fvset{fontsize=\small}
\usepackage{geometry}


% Include hyperref last.
\usepackage{hyperref}
% Fix anchor placement for figures with captions.
\usepackage{hypcap}% it must be loaded after hyperref.
% Set up styles of URL: it should be placed after hyperref.
\urlstyle{same}

\addto\captionsenglish{\renewcommand{\contentsname}{Getting started}}

\usepackage{sphinxmessages}
\setcounter{tocdepth}{1}

\input{../../latexPreamble.tex.txt}

\title{Enroute Flight Navigation}
\date{May 05, 2021}
\release{2.2.4}
\author{Stefan Kebekus}
\newcommand{\sphinxlogo}{\vbox{}}
\renewcommand{\releasename}{Release}
\makeindex
\begin{document}

\pagestyle{empty}
\sphinxmaketitle
\pagestyle{plain}
\sphinxtableofcontents
\pagestyle{normal}
\phantomsection\label{\detokenize{index::doc}}


\noindent{\hspace*{\fill}\sphinxincludegraphics[width=100\sphinxpxdimen]{{de.akaflieg_freiburg.enroute}.png}\hspace*{\fill}}

\sphinxAtStartPar
Enroute Flight Navigation is a free flight navigation app for Android and other
devices. Designed to be simple, functional and elegant, it takes the stress out
of your next flight. The program has been written by flight enthusiasts, as a
project of \sphinxhref{https://akaflieg-freiburg.de/}{Akaflieg Freiburg}%
\begin{footnote}[1]\sphinxAtStartFootnote
\sphinxnolinkurl{https://akaflieg-freiburg.de/}
%
\end{footnote}, a flight club
based in Freiburg, Germany.

\sphinxAtStartPar
Enroute Flight Navigation features a moving map, similar in style to the
official ICAO maps. Your current position and your flight path for the next five
minutes are marked, and so is your intended flight route. A double tap on the
display gives you all the information about airspaces, airfields and navaids \textendash{}
complete with frequencies, codes, elevations and runway information.

\sphinxAtStartPar
The free aeronautical maps can be downloaded for offline use. In addition to
airspaces, airfields and navaids, selected maps also show traffic circuits as
well as flight procedures for control zones. The maps receive near\sphinxhyphen{}weekly
updates and cover large parts of the world.

\sphinxAtStartPar
Enroute Flight Navigation includes flight weather data downloaded from the
\sphinxhref{https://www.aviationweather.gov/}{NOAA \sphinxhyphen{} Aviation Weather Center}%
\begin{footnote}[2]\sphinxAtStartFootnote
\sphinxnolinkurl{https://www.aviationweather.gov/}
%
\end{footnote}.

\sphinxAtStartPar
While Enroute Flight Navigation is no substitute for full\sphinxhyphen{}featured flight
planning software, it allows you to quickly and easily compute distances,
courses and headings, and gives you an estimate for flight time and fuel
consumption. If the weather turns bad, the app will show you the closest
airfields for landing, complete with distances, directions, runway information
and frequencies.


\chapter{Think before you fly}
\label{\detokenize{01-intro/01-think:think-before-you-fly}}\label{\detokenize{01-intro/01-think::doc}}
\sphinxAtStartPar
\sphinxstylestrong{Enroute Flight Navigation} is a free software product that has been published
in the hope that it might be useful as an aid to prudent navigation.  It comes
with no guarantees.  It may not work as expected.  Data shown to you might be
wrong.  Your hardware may fail.

\sphinxAtStartPar
This app is no substitute for proper flight preparation or good pilotage.  Any
information \sphinxstylestrong{must always} be validated using an official navigation and
airspace data source.

\begin{sphinxadmonition}{warning}{Warning:}
\sphinxAtStartPar
Always use official flight navigation data for flight preparation
and navigate by officially authorized means. The use of non\sphinxhyphen{}certified
navigation devices and software like \sphinxstylestrong{Enroute Flight Navigation} as a
primary source of navigation may cause accidents leading to loss of lives.
\end{sphinxadmonition}

\sphinxAtStartPar
We do not believe that the use of \sphinxstylestrong{Enroute Flight Navigation} fulfills the
requirement of the EU Regulation \sphinxhref{https://eur-lex.europa.eu/LexUriServ/LexUriServ.do?uri=OJ:L:2012:281:0001:0066:EN:PDF}{No 923/2012:SERA.2010}%
\begin{footnote}[3]\sphinxAtStartFootnote
\sphinxnolinkurl{https://eur-lex.europa.eu/LexUriServ/LexUriServ.do?uri=OJ:L:2012:281:0001:0066:EN:PDF}
%
\end{footnote}
\begin{quote}

\sphinxAtStartPar
Before beginning a flight, the pilot\sphinxhyphen{}in\sphinxhyphen{}command of an aircraft shall become
familiar with all available information appropriate to the intended operation.
\end{quote}

\sphinxAtStartPar
To put it simply: relying on \sphinxstylestrong{Enroute Flight Navigation} as a primary means of
navigation is most likely illegal in your jurisdiction.  It is most certainly
stupid and potentially suicidal.


\section{Software limitations}
\label{\detokenize{01-intro/01-think:software-limitations}}
\sphinxAtStartPar
\sphinxstylestrong{Enroute Flight Navigation} is not an officially approved flight navigation
software.  It is not officially approved or certified in any way.  The software
comes with no guarantee and might contain bugs.


\section{Navigational data and aviation data}
\label{\detokenize{01-intro/01-think:navigational-data-and-aviation-data}}
\sphinxAtStartPar
Navigational\textendash{} and aviation data, including airspace and airfield information,
are provided “as is” and without any guarantee, official validation,
certification or warranty.  The data does not come from official sources.  It
might be incomplete, outdated or otherwise incorrect.


\section{Operating system limitations}
\label{\detokenize{01-intro/01-think:operating-system-limitations}}
\sphinxAtStartPar
We expect that most users will run the software on mobile phones or tablet
computers running the Android operating system.  Android is not officially
approved or certified for aviation.  While we expect that the app will run fine
for the vast majority of Android users, please keep the following in mind.
\begin{itemize}
\item {} 
\sphinxAtStartPar
The Android operating system can decide at any time to terminate \sphinxstylestrong{Enroute
Flight Navigation} or to slow it down to clear resources for other apps.

\item {} 
\sphinxAtStartPar
Other apps might interfere with the operation of \sphinxstylestrong{Enroute Flight
Navigation}.

\item {} 
\sphinxAtStartPar
Many hardware vendors, most notably One Plus, Huawei and Samsung equip their
phone with “battery saving apps” that randomly kill long\sphinxhyphen{}running processes.
These apps cannot be uninstalled by the users, do not comply with Android
standards and are often extremely buggy.  At times, users can manually excempt
apps from “battery saving mode”, but the settings are usually lost on system
updates.  Google’s own “Pixel” and “Nexus” devices do not have these problems.
See the website \sphinxhref{https://dontkillmyapp.com}{Don’t kill my app}%
\begin{footnote}[4]\sphinxAtStartFootnote
\sphinxnolinkurl{https://dontkillmyapp.com}
%
\end{footnote} for more
information.

\end{itemize}


\section{Hardware limitations}
\label{\detokenize{01-intro/01-think:hardware-limitations}}
\sphinxAtStartPar
\sphinxstylestrong{Enroute Flight Navigation} runs on a variety of hardware platforms, but we
expect that most users will run the software on mobile phones, tablet computers
and comparable consumer electronic devices that are not certified to meet
aviation standards.  Keep the following in mind.
\begin{itemize}
\item {} 
\sphinxAtStartPar
Your device might not be designed to operate continuously for extended periods
of time, in particular if the display is on.

\item {} 
\sphinxAtStartPar
Your device can overheat. Batteries can catch fire.

\item {} 
\sphinxAtStartPar
Battery capacity is limited.  Even if your device is connected to power via a
USB cable, the display and/or CPU might use more energy than USB can deliver.

\end{itemize}


\chapter{Installation and setup}
\label{\detokenize{01-intro/02-installation:installation-and-setup}}\label{\detokenize{01-intro/02-installation::doc}}

\section{App installation}
\label{\detokenize{01-intro/02-installation:app-installation}}\begin{description}
\item[{Installation on Android devices}] \leavevmode
\sphinxAtStartPar
\sphinxstylestrong{Enroute Flight Navigation} is available as an Android App in the \sphinxhref{https://play.google.com/store/apps/details?id=de.akaflieg\_freiburg.enroute}{Google
Play Store}%
\begin{footnote}[5]\sphinxAtStartFootnote
\sphinxnolinkurl{https://play.google.com/store/apps/details?id=de.akaflieg\_freiburg.enroute}
%
\end{footnote}.

\sphinxAtStartPar
An unofficial version of the app is also available at \sphinxhref{https://f-droid.org/de/packages/de.akaflieg\_freiburg.enroute/}{F\sphinxhyphen{}Droid}%
\begin{footnote}[6]\sphinxAtStartFootnote
\sphinxnolinkurl{https://f-droid.org/de/packages/de.akaflieg\_freiburg.enroute/}
%
\end{footnote}.  While the
author of \sphinxstylestrong{Enroute Flight Navigation} endorses publication at F\sphinxhyphen{}Droid, he
has not tested this unofficial app for quality.

\item[{Installation on Linux desktop machines}] \leavevmode
\sphinxAtStartPar
\sphinxstylestrong{Enroute Flight Navigation} is available for free download at \sphinxhref{https://flathub.org/apps/details/de.akaflieg\_freiburg.enroute}{flathub.org}%
\begin{footnote}[7]\sphinxAtStartFootnote
\sphinxnolinkurl{https://flathub.org/apps/details/de.akaflieg\_freiburg.enroute}
%
\end{footnote} and
\sphinxhref{https://snapcraft.io/enroute-flight-navigation}{snapcraft.io}%
\begin{footnote}[8]\sphinxAtStartFootnote
\sphinxnolinkurl{https://snapcraft.io/enroute-flight-navigation}
%
\end{footnote}.  Most likely
you will also find the app in the software management application on your
computer.

\end{description}

\sphinxAtStartPar
After installation, start the app.  Depending on the platform, you might need to
grant the necessary permissions.  You will be asked to accept the terms and
conditions.


\section{Map download}
\label{\detokenize{01-intro/02-installation:map-download}}
\sphinxAtStartPar
\sphinxstylestrong{Enroute Flight Navigation} cannot be used without geographic maps.  Two types
of maps need to be installed for every region where you fly.
\begin{itemize}
\item {} 
\sphinxAtStartPar
Aeronautical maps.  These contain airspaces, airfields and navaids.  Some maps
also contain reporting points, airfield traffic circuits and control zone
entry/exit routes.

\item {} 
\sphinxAtStartPar
Base maps.  These contain geographic data, such as rivers, roads, railroads
and land use.

\end{itemize}

\sphinxAtStartPar
Follow these steps to install the maps that you need.
\begin{itemize}
\item {} 
\sphinxAtStartPar
Open the Menu by touching the menu button in the upper right side of the
screen.  The button is marked with the symbol ‘☰’.

\item {} 
\sphinxAtStartPar
Choose the menu item \sphinxstyleemphasis{Library}, then \sphinxstyleemphasis{Maps}.  The map management page will
then open.

\item {} 
\sphinxAtStartPar
On the map management page, click or tap on the desired maps.  The maps will
be downloaded and installed on your device.

\end{itemize}

\sphinxAtStartPar
Please download only those maps that you will actually need.  The infrastructure
and bandwidth for map downloads is kindly sponsored by the University of
Freiburg, under the assumption that the cost stays within reasonable limits.
You will also find that the app performs much better if it does not have to
process many megabytes of map data.

\begin{sphinxadmonition}{note}{Note:}
\sphinxAtStartPar
Do not forget that you need aeronautical maps \sphinxstylestrong{and} base maps for
the desired area of flight.  The base maps are large.  Make sure that you
have a good internet connection before you download maps.  It might be
inadvisable to download base maps via the mobile phone network.
\end{sphinxadmonition}


\section{Done.}
\label{\detokenize{01-intro/02-installation:done}}
\sphinxAtStartPar
Once the map download has finished, \sphinxstylestrong{Enroute Flight Navigation} will process
the map data and update the map display after a minute or so.  Tap or click on
the arrow symbol ‘←’ or use the Android ‘Back’ button to leave the map page and
return to the main screen.

\sphinxAtStartPar
You are now ready to go.  There are many things that you could set up at this
stage, but we recommend that you simply look around any play with the app.
Continue with the next section and take it for your first flight.


\chapter{Before your first flight}
\label{\detokenize{01-intro/03-firstFlight:before-your-first-flight}}\label{\detokenize{01-intro/03-firstFlight::doc}}
\sphinxAtStartPar
Now you are ready for the first use of \sphinxstylestrong{Enroute Flight Navigation}.  General
operation is very intuitive.  Still, we recommend that you take a minute to make
yourself familiar with the moving map display and with the basic controls before
you take the app on its first flight.


\section{The moving map}
\label{\detokenize{01-intro/03-firstFlight:the-moving-map}}
\sphinxAtStartPar
After startup, the app will show a moving map, similar in style to the standard
ICAO maps that most pilots are used to.  You can use the standard gestures to
zoom and pan the map to your liking.  The figures {\hyperref[\detokenize{01-intro/03-firstFlight:movingmapground}]{\sphinxcrossref{\DUrole{std,std-ref}{Moving map display on the ground}}}} and
{\hyperref[\detokenize{01-intro/03-firstFlight:movingmapflight}]{\sphinxcrossref{\DUrole{std,std-ref}{Moving map display in flight}}}} shows how the map will typically look.

\begin{figure}[htbp]
\centering
\capstart

\noindent\sphinxincludegraphics[scale=0.3]{{fig_ground}.png}
\caption{Moving map display on the ground}\label{\detokenize{01-intro/03-firstFlight:id1}}\label{\detokenize{01-intro/03-firstFlight:movingmapground}}\end{figure}

\begin{figure}[htbp]
\centering
\capstart

\noindent\sphinxincludegraphics[scale=0.3]{{fig_flight}.png}
\caption{Moving map display in flight}\label{\detokenize{01-intro/03-firstFlight:id2}}\label{\detokenize{01-intro/03-firstFlight:movingmapflight}}\end{figure}

\sphinxAtStartPar
Initially, your own position is shown as a blue circle (or gray if the system
has not yet acquired a valid position).  Once you are moving, your own position
is shown as a blue arrow shape.  The flight path vector shows the projected
track for the next five minutes.

\sphinxAtStartPar
The bottom of the display shows a little panel with the following information.


\begin{savenotes}\sphinxattablestart
\centering
\begin{tabulary}{\linewidth}[t]{|T|T|}
\hline

\sphinxAtStartPar
T.TALT
&
\sphinxAtStartPar
True altitude (=geometric altitude) above sea level.
\\
\hline
\sphinxAtStartPar
FL
&
\sphinxAtStartPar
Flight level.
\\
\hline
\sphinxAtStartPar
GS
&
\sphinxAtStartPar
Ground speed.
\\
\hline
\sphinxAtStartPar
TT
&
\sphinxAtStartPar
True track.
\\
\hline
\sphinxAtStartPar
UTC
&
\sphinxAtStartPar
Current time.
\\
\hline
\end{tabulary}
\par
\sphinxattableend\end{savenotes}

\sphinxAtStartPar
The flight level is only available if your device is connected to a traffic
receiver (such as a PowerFLARM device) that reports the pressure altitude.
Flight level and current time are hidden if the display is not wide enough.

\begin{sphinxadmonition}{warning}{Warning:}
\sphinxAtStartPar
Vertical airspace boundaries are defined by pressure altitudes
(with respect to QNH or standard pressure).  Depending on temperature and air
density, the pressure altitude will differ from the true altitude that is
shown by the app.  \sphinxstylestrong{Never use true altitude to judge vertical distances to
airspaces.}
\end{sphinxadmonition}


\section{Interactive controls}
\label{\detokenize{01-intro/03-firstFlight:interactive-controls}}
\sphinxAtStartPar
In addition to the pan and pinch gestures, you can use the following buttons to
control the app.


\begin{savenotes}\sphinxattablestart
\centering
\begin{tabulary}{\linewidth}[t]{|T|T|}
\hline

\noindent\sphinxincludegraphics{{ic_menu}.png}
&
\sphinxAtStartPar
Open main menu
\\
\hline
\noindent\sphinxincludegraphics{{NorthArrow}.png}
&
\sphinxAtStartPar
Switch between display modes \sphinxstylestrong{north up} and \sphinxstylestrong{track up}.
\\
\hline
\noindent\sphinxincludegraphics{{ic_my_location}.png}
&
\sphinxAtStartPar
Center map about own position.
\\
\hline
\noindent\sphinxincludegraphics{{ic_add}.png}
&
\sphinxAtStartPar
Zoom in
\\
\hline
\noindent\sphinxincludegraphics{{ic_remove}.png}
&
\sphinxAtStartPar
Zoom out
\\
\hline
\end{tabulary}
\par
\sphinxattableend\end{savenotes}


\section{Information about airspaces, airfields and other facilities}
\label{\detokenize{01-intro/03-firstFlight:information-about-airspaces-airfields-and-other-facilities}}
\sphinxAtStartPar
Double tap or tap\sphinxhyphen{}and\sphinxhyphen{}hold anywhere in the map to obtain information about the
airspace situation at that point.  If you double tap or tap\sphinxhyphen{}and\sphinxhyphen{}hold on an
airfield, navaid or reporting point, detailed information about the facility
will be shown.  The figure {\hyperref[\detokenize{01-intro/03-firstFlight:stuttgartinfo}]{\sphinxcrossref{\DUrole{std,std-ref}{Information about Stuttgart airport}}}} shows how this will typically
look.

\begin{figure}[htbp]
\centering
\capstart

\noindent\sphinxincludegraphics[scale=0.3]{{fig_wpInfo}.png}
\caption{Information about Stuttgart airport}\label{\detokenize{01-intro/03-firstFlight:id3}}\label{\detokenize{01-intro/03-firstFlight:stuttgartinfo}}\end{figure}


\section{Go flying!}
\label{\detokenize{01-intro/03-firstFlight:go-flying}}
\sphinxAtStartPar
\sphinxstylestrong{Enroute Flight Navigation} is designed to be simple.  We think that you are
now ready to take \sphinxstylestrong{Enroute Flight Navigation} on its first flight.  There are
of course many more things that you can do.  Play with the app and have a look
at the next section {\hyperref[\detokenize{index:sec-steps}]{\sphinxcrossref{\DUrole{std,std-ref}{Further Steps}}}}.


\chapter{Connect your traffic receiver}
\label{\detokenize{02-steps/traffic:connect-your-traffic-receiver}}\label{\detokenize{02-steps/traffic::doc}}
\sphinxAtStartPar
In order to display nearby traffic on the moving map, \sphinxstylestrong{Enroute Flight
Navigation} can connect to your aircraft’s traffic receiver (typically a FLARM
device).  In order to show only relevant information, \sphinxstylestrong{Enroute Flight
Navigation} will not display traffic more than 1,500 m above or below the own
position.

\sphinxAtStartPar
The app author has tested the \sphinxstylestrong{Enroute Flight Navigation} with the following
traffic receivers.
\begin{itemize}
\item {} 
\sphinxAtStartPar
\sphinxhref{http://www.air-avionics.com/?page\_id=253}{AT\sphinxhyphen{}1 AIR Traffic}%
\begin{footnote}[9]\sphinxAtStartFootnote
\sphinxnolinkurl{http://www.air-avionics.com/?page\_id=253}
%
\end{footnote} by \sphinxhref{http://www.air-avionics.com/}{Air
Avionics}%
\begin{footnote}[10]\sphinxAtStartFootnote
\sphinxnolinkurl{http://www.air-avionics.com/}
%
\end{footnote} with software version 5.

\end{itemize}

\sphinxAtStartPar
Users reported success with the following traffic receivers.
\begin{itemize}
\item {} 
\sphinxAtStartPar
\sphinxhref{http://stratux.me/}{Stratux devices}%
\begin{footnote}[11]\sphinxAtStartFootnote
\sphinxnolinkurl{http://stratux.me/}
%
\end{footnote}

\item {} 
\sphinxAtStartPar
\sphinxhref{https://www.amazon.de/TTGO-T-Beam-915Mhz-Wireless-Bluetooth/dp/B07SFVQ3Z8}{TTGO T\sphinxhyphen{}Beam devices}%
\begin{footnote}[12]\sphinxAtStartFootnote
\sphinxnolinkurl{https://www.amazon.de/TTGO-T-Beam-915Mhz-Wireless-Bluetooth/dp/B07SFVQ3Z8}
%
\end{footnote}

\end{itemize}


\section{Before you connect}
\label{\detokenize{02-steps/traffic:before-you-connect}}
\sphinxAtStartPar
Before you try to connect this app to your traffic receiver, make sure that the
following conditions are met.
\begin{itemize}
\item {} 
\sphinxAtStartPar
Your traffic receiver has an integrated Wi\sphinxhyphen{}Fi interface that acts as a
wireless access point. Bluetooth devices are currently not supported.

\item {} 
\sphinxAtStartPar
You know the network name (=SSID) of the Wi\sphinxhyphen{}Fi network deployed by your
traffic receiver. If the network is encrypted, you also need to know the Wi\sphinxhyphen{}Fi
password.

\item {} 
\sphinxAtStartPar
Some devices require an additional password in order to access traffic
data. This is currently \sphinxstylestrong{not} supported. Set up your device so that no
additional password is required.

\end{itemize}


\section{Connect to the traffic receiver}
\label{\detokenize{02-steps/traffic:connect-to-the-traffic-receiver}}
\sphinxAtStartPar
It takes two steps to connect \sphinxstylestrong{Enroute Flight Navigation} to the traffic
receiver for the first time. Once things are set up properly, your device should
automatically detect the traffic receiver’s Wi\sphinxhyphen{}Fi network, enter the network and
connect to the traffic data stream whenever you go flying.


\subsection{Step 1: Enter the traffic receiver’s Wi\sphinxhyphen{}Fi network}
\label{\detokenize{02-steps/traffic:step-1-enter-the-traffic-receiver-s-wi-fi-network}}\begin{itemize}
\item {} 
\sphinxAtStartPar
Make sure that the traffic receiver has power and is switched on. In a typical
aircraft installation, the traffic receiver is connected to the ‘Avionics’
switch and will automatically switch on. You may need to wait a minute before
the Wi\sphinxhyphen{}Fi comes online and is visible to your device.

\item {} 
\sphinxAtStartPar
Enter the Wi\sphinxhyphen{}Fi network deployed by your traffic receiver. This is usually
done in the “Wi\sphinxhyphen{}Fi Settings” of your device. Enter the Wi\sphinxhyphen{}Fi password if
required. Some devices will issue a warning that the Wi\sphinxhyphen{}Fi is not connected to
the internet. In this case, you might need to confirm that you wish to enter
the Wi\sphinxhyphen{}Fi network.

\end{itemize}

\sphinxAtStartPar
Most operating systems will offer to remember the connection, so that your
device will automatically connect to this Wi\sphinxhyphen{}Fi in the future. We recommend
using this option.


\subsection{Step 2: Connect to the traffic data stream}
\label{\detokenize{02-steps/traffic:step-2-connect-to-the-traffic-data-stream}}
\sphinxAtStartPar
Open the main menu and navigate to the “Information” menu.
\begin{itemize}
\item {} 
\sphinxAtStartPar
If the entry “Traffic Receiver” is highlighted in green, then \sphinxstylestrong{Enroute Flight
Navigation} has already found the traffic receiver in the network and has
connected to it. Congratulations, you are done!

\item {} 
\sphinxAtStartPar
If the entry “Traffic Receiver” is not highlighted in green, then select the
entry. The “Traffic Receiver Status” page will open. The page explains the
connection status in detail, and explains how to establish a connection
manually.

\end{itemize}


\section{Troubleshooting}
\label{\detokenize{02-steps/traffic:troubleshooting}}
\sphinxAtStartPar
\sphinxstylestrong{The app cannot connect to the traffic data stream.}
\begin{itemize}
\item {} 
\sphinxAtStartPar
Check that your device is connected to the Wi\sphinxhyphen{}Fi network deployed by your
traffic receiver.

\end{itemize}

\sphinxAtStartPar
\sphinxstylestrong{The connection breaks down after a few seconds.}

\sphinxAtStartPar
Most traffic receivers cannot serve more than one client and abort connections
at random if more than one device tries to access.
\begin{itemize}
\item {} 
\sphinxAtStartPar
Make sure that there no second device connected to the traffic receiver’s
Wi\sphinxhyphen{}Fi network. The other device might well be in your friend’s pocket!

\item {} 
\sphinxAtStartPar
Make sure that there is no other app trying to connect to the traffic
receiver’s data stream.

\item {} 
\sphinxAtStartPar
Many traffic receivers offer “configuration panels” that can be accessed via a
web browser. Close all web browsers.

\end{itemize}


\chapter{Connect your flight simulator}
\label{\detokenize{02-steps/simulator:connect-your-flight-simulator}}\label{\detokenize{02-steps/simulator::doc}}
\sphinxAtStartPar
\sphinxstylestrong{Enroute Flight Navigation} can connect to flight simulator software.  The app
has been tested with the following programs.
\begin{itemize}
\item {} 
\sphinxAtStartPar
\sphinxhref{https://www.x-plane.com/}{X\sphinxhyphen{}Plane 11}%
\begin{footnote}[13]\sphinxAtStartFootnote
\sphinxnolinkurl{https://www.x-plane.com/}
%
\end{footnote}

\end{itemize}

\sphinxAtStartPar
Please contact us if you are aware of other programs that also work.

\begin{sphinxadmonition}{note}{Note:}
\sphinxAtStartPar
\sphinxstylestrong{Enroute Flight Navigation} treats flight simulators as traffic
receivers.  To see the connection status, open the main menu and navigate to
the “Information” menu.
\end{sphinxadmonition}


\section{Before you connect}
\label{\detokenize{02-steps/simulator:before-you-connect}}
\sphinxAtStartPar
This manual assumes a typical home setup, where both the computer that runs the
flight simulator and the device that runs \sphinxstylestrong{Enroute Flight Navigation} are
connected to a Wi\sphinxhyphen{}Fi network deployed by a home router.  Make sure that the
following conditions are met.
\begin{itemize}
\item {} 
\sphinxAtStartPar
The computer that runs the flight simulator and the device that runs \sphinxstylestrong{Enroute
Flight Navigation} are connected to the same Wi\sphinxhyphen{}Fi network.  Some routers
deploy two networks, often called “main network” and a “guest network”.

\item {} 
\sphinxAtStartPar
Make sure that the router allows data transfer between the devices in the
Wi\sphinxhyphen{}Fi network.  Some routers have “security settings” that disallow data
transfer between the devices in the “guest network”

\end{itemize}


\section{Set up your flight simulator}
\label{\detokenize{02-steps/simulator:set-up-your-flight-simulator}}
\sphinxAtStartPar
Your flight simulation software needs to broadcast position and traffic
information over the Wi\sphinxhyphen{}Fi network.  Once this is done, there is no further
setup required.  As soon as the flight simulator starts to broadcast information
over the Wi\sphinxhyphen{}Fi network, the moving map of \sphinxstylestrong{Enroute Flight Navigation} will
adjust accordingly.  To end the connection to the flight simulator, simply leave
the flight simulator’s Wi\sphinxhyphen{}Fi network.


\subsection{X\sphinxhyphen{}Plane 11}
\label{\detokenize{02-steps/simulator:id1}}
\sphinxAtStartPar
Open the “Settings” window and choose the “Network” tab.  Locate the settings
group “This machine’s role” on the right\sphinxhyphen{}hand side of the tab. Open the section
“iPHONE, iPAD, and EXTERNAL APPS” and select the item “Broadcast to all mapping
apps on the network” under the headline “OTHER MAPPING APPS”.

\noindent\sphinxincludegraphics{{X-Plane-11}.png}


\subsection{MS Flight Simulator}
\label{\detokenize{02-steps/simulator:ms-flight-simulator}}
\sphinxAtStartPar
UNKNOWN AS OF NOW.


\subsection{Other programs}
\label{\detokenize{02-steps/simulator:other-programs}}
\sphinxAtStartPar
The flight simulator needs to be set up to send UDP datagrams in one of the
standard formats “GDL90” or “XGPS” to ports 4000 or 49002.  Given the choice,
GDL90 is generally the preferred format.


\section{Troubleshooting}
\label{\detokenize{02-steps/simulator:troubleshooting}}
\sphinxAtStartPar
\sphinxstylestrong{Enroute Flight Navigation} treats flight simulators as traffic receivers.  To
see the connection status, open the main menu and navigate to the “Information”
menu.  If the entry “Traffic Receiver” is highlighted in green, then \sphinxstylestrong{Enroute
Flight Navigation} has already found the program in the network and has
connected to it.  If not, then select the entry. The “Traffic Receiver Status”
page will open, which explains the connection status in more detail.


\chapter{Map Data}
\label{\detokenize{03-reference/map_data:map-data}}\label{\detokenize{03-reference/map_data::doc}}
\sphinxAtStartPar
The Information displayed by the Map of Enroute Flight Navigation is provided by the following resources:
\begin{itemize}
\item {} 
\sphinxAtStartPar
openAIP

\item {} 
\sphinxAtStartPar
open flightmaps

\item {} 
\sphinxAtStartPar
Map Tiler

\item {} 
\sphinxAtStartPar
Open Street Map

\end{itemize}
\begin{description}
\item[{To get more detailed Information on these Resources you may touch the link on the lower edge of the map Display \sphinxstylestrong{Map Data Copyright Info}. After touching the line \sphinxstylestrong{Map Data Copyright Info} a sub window will open showing links to the contributor web sites:}] \leavevmode\begin{itemize}
\item {} 
\sphinxAtStartPar
\sphinxurl{https://www.openaip.net}

\item {} 
\sphinxAtStartPar
\sphinxurl{https://www.openflightmaps.org}

\item {} 
\sphinxAtStartPar
\sphinxurl{https://www.maptiler.com}

\item {} 
\sphinxAtStartPar
\sphinxurl{https://www.openstreetmap.org}

\end{itemize}

\end{description}

\sphinxAtStartPar
\sphinxstylestrong{Open AIP}

\sphinxAtStartPar
Open AIP has the goal to deliver free, current and precise data for air navigation to everyone. Open AIP is a web based and crowd\sphinxhyphen{}sourced platform.
The Open AIP provides the basic source aeronautical data for display in Enroute Flight Navigation.

\sphinxAtStartPar
\sphinxstylestrong{Open Flight Maps}

\sphinxAtStartPar
Open Flight Maps is an open\sphinxhyphen{}source project providing aeronautical data for a high quality VFR Map.
Open Flight Maps is providing some additional information, where available.

\sphinxAtStartPar
The detailed split of the data sources for the Enroute Flight Naviagtion map is shown below:


\begin{savenotes}\sphinxattablestart
\centering
\begin{tabulary}{\linewidth}[t]{|T|T|}
\hline
\sphinxstyletheadfamily 
\sphinxAtStartPar
Map Feature
&\sphinxstyletheadfamily 
\sphinxAtStartPar
Data Origin
\\
\hline
\sphinxAtStartPar
Airfields
&
\sphinxAtStartPar
openAIP
\\
\hline
\sphinxAtStartPar
Airspace: Nature Preserve Areas
&
\sphinxAtStartPar
open flightmaps
\\
\hline
\sphinxAtStartPar
Airspace: all other
&
\sphinxAtStartPar
openAIP
\\
\hline
\sphinxAtStartPar
Navaids
&
\sphinxAtStartPar
openAIP
\\
\hline
\sphinxAtStartPar
Procedures (Traffic Circuits, …)
&
\sphinxAtStartPar
open flightmaps
\\
\hline
\sphinxAtStartPar
Reporting Points
&
\sphinxAtStartPar
open flightmaps
\\
\hline
\end{tabulary}
\par
\sphinxattableend\end{savenotes}

\sphinxAtStartPar
\sphinxstylestrong{Map Tiler}

\sphinxAtStartPar
Is a software application to combine multiple layers of data for maps and provide the map in a format for loading and display.
The Enroute Flight Naviagtion base maps are edited versions of maps kindly provided by Klokan Technologies through the OpenMapTiles project.

\sphinxAtStartPar
\sphinxstylestrong{Open Street Map}

\sphinxAtStartPar
Open Street Map (OSM) is a collaborative project to create a free editable map of the world. The geodata underlying the map is considered the primary output of the project. The creation and growth of OSM has been motivated by restrictions on use or availability of map data across much of the world, and the advent of inexpensive portable satellite navigation devices.
The Open Street Map data is used to crate the base maps.


\chapter{Other}
\label{\detokenize{03-reference/map_data:other}}
\sphinxAtStartPar
The Map display is composed of two layers selected in the respective Tabs of the
‘Map Library’ screen:
\begin{itemize}
\item {} 
\sphinxAtStartPar
Aeronautical Map

\item {} 
\sphinxAtStartPar
Base Map

\end{itemize}

\sphinxAtStartPar
\sphinxstylestrong{Aeronautical Maps}

\sphinxAtStartPar
The Aeronautical Map layers is showing the airspace data on the Map screen. If
no Base Map is installed for the area only the information coming from the
Aviation Map data is displayed.

\sphinxAtStartPar
The Aeronautical Map contains:
\begin{itemize}
\item {} 
\sphinxAtStartPar
Airfields

\item {} 
\sphinxAtStartPar
Airspace boundaries

\item {} 
\sphinxAtStartPar
Navaids

\item {} 
\sphinxAtStartPar
Reporting points and routes (if available)

\end{itemize}

\sphinxAtStartPar
The display used for aerospace data is using the following basic color scheme:
\begin{itemize}
\item {} \begin{description}
\item[{Red:}] \leavevmode\begin{itemize}
\item {} 
\sphinxAtStartPar
Line with shadow inside for Restricted Airspace

\item {} 
\sphinxAtStartPar
Shadow with dashed blue border for Aerodrome Control Zone (CTR)

\item {} 
\sphinxAtStartPar
Dashed Line for Parachute Jumping Exercise area

\item {} 
\sphinxAtStartPar
Line for Glider or Microlight Traffic pattern

\end{itemize}

\end{description}

\item {} 
\sphinxAtStartPar
Blue:
\begin{itemize}
\item {} 
\sphinxAtStartPar
Line with shadow for controlled airspace (A, B, C, D)

\item {} 
\sphinxAtStartPar
Shadow with dashed blue border for Radio Mandatory Zone (RMZ)

\item {} 
\sphinxAtStartPar
Airport, reporting point or Navaid  symbols

\item {} 
\sphinxAtStartPar
For Route or Traffic Pattern for powered aircraft

\end{itemize}

\item {} 
\sphinxAtStartPar
Green:
\begin{itemize}
\item {} 
\sphinxAtStartPar
Line with shadow for Natural Reserve Area (NRA)

\item {} 
\sphinxAtStartPar
Line for airspace control sector boundaries

\end{itemize}

\item {} 
\sphinxAtStartPar
Black:
\begin{itemize}
\item {} 
\sphinxAtStartPar
Dashed Line for Transponder Mandatory Zone (TMZ)

\end{itemize}

\end{itemize}

\sphinxAtStartPar
\sphinxstylestrong{Class 1 and Class 2 maps:}
\begin{itemize}
\item {} 
\sphinxAtStartPar
Class 1 maps are compiled from openAIP and open flightmaps data. These maps
contain complete information about airspaces, airfields and navaids. In
addition, the maps contain (mandatory) reporting points. Some of our tier 1
maps also show traffic circuits and flight procedures for control zones.

\item {} 
\sphinxAtStartPar
Class 2 maps are compiled from openAIP data only. They contain complete
information about airspaces, airfields and navaids.

\end{itemize}

\sphinxAtStartPar
Details on the maps may be found at
\textless{}\sphinxurl{https://akaflieg-freiburg.github.io/enroute/maps/}\textgreater{} The Aeronautical Map data is
selected on the “Map Library” \textendash{} “Aviation Data” page accessed via the “Settings”
Menu.  To update the list of available maps the “…” option in the upper right
corner of the screen may be used.  You may install or uninstall the aviation Map
data for a county by the selection on the right hand side of the country
list. To find a country you have to scroll up and down in the list.

\begin{sphinxadmonition}{note}{Note:}
\sphinxAtStartPar
To have optimum presentation of the \sphinxstylestrong{Enroute Flight Navigation} map
display install the Aviation Map and the Base Map for all areas you intend
to use \sphinxstylestrong{Enroute Flight Navigation}.
\end{sphinxadmonition}

\begin{sphinxadmonition}{caution}{Caution:}
\sphinxAtStartPar
No airspace information will be provided in country when the
Aeronautical Map is not installed for it.
\end{sphinxadmonition}

\begin{sphinxadmonition}{note}{Note:}
\sphinxAtStartPar
\sphinxstylestrong{Enroute Flight Navigation} will automatically check for updated
Maps on the Enroute server and show a pop\sphinxhyphen{}up window after start if updated
maps have been detected.  You will be asked if you want to update the map or
delay the update.
\end{sphinxadmonition}

\sphinxAtStartPar
\sphinxstylestrong{Base Map}

\sphinxAtStartPar
The Base Map layers is showing the geographic data on the Map screen. If no Base
Map is shown for an area it will be shown in the white background color. If no
Aviation Map is installed for the area only the information coming from the Base
Map data is displayed. The Base Map is organized in tiles. This will result in
not stopping the Base Map display abruptly at the border of an installed
country, but showing some overlap.  The Base Map will show:
\begin{itemize}
\item {} 
\sphinxAtStartPar
Landmass

\item {} 
\sphinxAtStartPar
Water Surface (oceans, lakes and rivers)

\item {} 
\sphinxAtStartPar
Forests

\item {} 
\sphinxAtStartPar
Main Roads

\item {} 
\sphinxAtStartPar
Railroad lines

\item {} 
\sphinxAtStartPar
City names

\end{itemize}

\begin{sphinxadmonition}{note}{Note:}
\sphinxAtStartPar
To have optimum presentation of the \sphinxstylestrong{Enroute Flight Navigation} map
display install the Aeronautical Map and the Base Map for all areas you
intend to use \sphinxstylestrong{Enroute Flight Navigation}.
\end{sphinxadmonition}

\begin{sphinxadmonition}{note}{Note:}
\sphinxAtStartPar
\sphinxstylestrong{Enroute Flight Navigation} will not show most cultural build ups
and limits or settled area boundaries to reduce the map size.
\end{sphinxadmonition}


\section{Flight mode and ground mode}
\label{\detokenize{03-reference/map_data:flight-mode-and-ground-mode}}
\sphinxAtStartPar
\sphinxstylestrong{Ground Mode}

\sphinxAtStartPar
Ground Mode is displayed by \sphinxstylestrong{Enroute Flight Navigation} whenever the sensed
speed is below the threshold and the Menu item ‘Automatic Flight Detection’ is
not set to ‘Always in Flight Mode’.  Ground Mode does not display the Flight
Data line at the lower end of the screen and is intended for flight planning.


\chapter{Airspace Display}
\label{\detokenize{03-reference/airspace_display:airspace-display}}\label{\detokenize{03-reference/airspace_display::doc}}
\sphinxAtStartPar
The display of airspace will generally follow the common ICAO symbology.
Restricted Airspace
Restricted airspace will be surrounded by an intense red dashed line and a thick transparent red line inside the restricted area boundaries.
When selecting a point inside the restricted area by double touching the screen the information to the related area is given with the waypoint pop\sphinxhyphen{}up window:
\begin{itemize}
\item {} 
\sphinxAtStartPar
Area Name

\item {} 
\sphinxAtStartPar
Area altitude limits

\item {} 
\sphinxAtStartPar
Area activation time

\end{itemize}

\begin{figure}[htbp]
\centering

\noindent\sphinxincludegraphics{{fig_Restricted}.png}
\end{figure}

\sphinxAtStartPar
\sphinxstyleemphasis{Legend}:
\begin{enumerate}
\sphinxsetlistlabels{\arabic}{enumi}{enumii}{}{.}%
\item {} 
\sphinxAtStartPar
Outline of Restricted Airspace

\item {} 
\sphinxAtStartPar
Designation and activation time of airspace

\end{enumerate}


\section{Controlled Airspace}
\label{\detokenize{03-reference/airspace_display:controlled-airspace}}
\sphinxAtStartPar
All boundaries of controlled airspace are shown by a solid blue line and a thick transparent blue line inside the airspace. Figure 13:  Controlled Airspace
When selecting a point inside the controlled airspace by double touching the screen the information to the related area is given with the waypoint pop\sphinxhyphen{}up window:
\begin{itemize}
\item {} 
\sphinxAtStartPar
Area Name

\item {} 
\sphinxAtStartPar
Area altitude limits

\end{itemize}

\begin{sphinxadmonition}{caution}{Caution:}
\sphinxAtStartPar
All controlled airspace (Class A \textendash{} Class D) are shown in the same way even if different restrictions or ATC clearance requirements may be present.
\end{sphinxadmonition}


\section{Control Zone}
\label{\detokenize{03-reference/airspace_display:control-zone}}
\sphinxAtStartPar
The Control Zone of an airport is shown with a dashed blue line filled in transparent red color. Figure 13:  Controlled Airspace
When selecting a point inside the Control Zone (CTR) by double touching the screen the information to the related area is given with the waypoint pop\sphinxhyphen{}up window:
\begin{itemize}
\item {} 
\sphinxAtStartPar
Area Name

\item {} 
\sphinxAtStartPar
Area altitude limits

\end{itemize}

\begin{figure}[htbp]
\centering

\noindent\sphinxincludegraphics{{fig_AirspaceMUC}.png}
\end{figure}

\sphinxAtStartPar
\sphinxstyleemphasis{Legend}:
\begin{enumerate}
\sphinxsetlistlabels{\arabic}{enumi}{enumii}{}{.}%
\item {} 
\sphinxAtStartPar
Airport ICAO Symbol

\item {} 
\sphinxAtStartPar
Airport Control Zone (CTR)

\item {} 
\sphinxAtStartPar
Radio Mandatory Zone (RMZ)

\item {} 
\sphinxAtStartPar
Boundary of Controlled Airspace

\item {} 
\sphinxAtStartPar
Restricted Airspace

\end{enumerate}


\section{Transponder Mandatory Zones}
\label{\detokenize{03-reference/airspace_display:transponder-mandatory-zones}}
\sphinxAtStartPar
Transponder Mandatory Zones TMZ are shown with a black dashed outline.
When selecting a point inside the Transponder Mandatory Zone (TMZ) by double touching the screen the information to the related ares is given with the waypoint pop\sphinxhyphen{}up window:
\begin{itemize}
\item {} 
\sphinxAtStartPar
Area Name

\item {} 
\sphinxAtStartPar
Area altitude limits

\item {} 
\sphinxAtStartPar
Monitoring Frequency

\item {} 
\sphinxAtStartPar
Mode 3 Squawk

\end{itemize}


\section{Radio Mandatory Zone}
\label{\detokenize{03-reference/airspace_display:radio-mandatory-zone}}
\sphinxAtStartPar
Radio Mandatory Zones (RMZ) are shown with a solid blue dashed outline and filled in transparent blue.
When selecting a point inside the Radio Mandatory Zone (RMZ) by double touching the screen the information to the related area is given with the waypoint pop\sphinxhyphen{}up window:
\begin{itemize}
\item {} 
\sphinxAtStartPar
Area Name

\item {} 
\sphinxAtStartPar
Area altitude limits

\item {} 
\sphinxAtStartPar
Radio Frequency

\end{itemize}


\section{Parachute Jumping Areas}
\label{\detokenize{03-reference/airspace_display:parachute-jumping-areas}}
\sphinxAtStartPar
Parachute Jumping Exercise areas (PJE) are shown with a solid red dashed outline.
When selecting a point inside the PJE by double touching the screen the information to the related area is given with the waypoint pop\sphinxhyphen{}up window:
\begin{itemize}
\item {} 
\sphinxAtStartPar
Area Name

\item {} 
\sphinxAtStartPar
Area altitude limits

\item {} 
\sphinxAtStartPar
Radio Frequency

\end{itemize}


\section{Nature Reserve Areas}
\label{\detokenize{03-reference/airspace_display:nature-reserve-areas}}
\sphinxAtStartPar
Nature Reserve Areas (NRA) are shown with a solid green outline.
When selecting a point inside the NRA by double touching the screen the information to the related area is given with the waypoint pop\sphinxhyphen{}up window:
\begin{itemize}
\item {} 
\sphinxAtStartPar
Area Name

\item {} 
\sphinxAtStartPar
Area altitude limits

\end{itemize}

\begin{sphinxadmonition}{caution}{Caution:}
\sphinxAtStartPar
Check restrictions applicable for flying inside NRA when planning your flight. For example in Austria high fines are applicable when flying inside NRA.
\begin{quote}

\sphinxAtStartPar
Figure 14:  Nature Reserve Area
\end{quote}
\end{sphinxadmonition}

\begin{figure}[htbp]
\centering

\noindent\sphinxincludegraphics{{fig_nra}.png}
\end{figure}

\sphinxAtStartPar
\sphinxstyleemphasis{Legend}:
\begin{enumerate}
\sphinxsetlistlabels{\arabic}{enumi}{enumii}{}{.}%
\item {} 
\sphinxAtStartPar
Outline of Nature Reserve Area (NRA)

\item {} 
\sphinxAtStartPar
Designation of NRA

\end{enumerate}


\section{Airfields}
\label{\detokenize{03-reference/airspace_display:airfields}}
\sphinxAtStartPar
The symbology used to display airfields follows the ICAO rules.
Airfield Information
When selecting an airfield by double touching the screen the related information is given in a pop\sphinxhyphen{}up window:
\begin{itemize}
\item {} 
\sphinxAtStartPar
Airfield Name and Identifier

\item {} 
\sphinxAtStartPar
Radio Frequency including COM and Information frequencies

\item {} 
\sphinxAtStartPar
Navaid frequencies

\item {} 
\sphinxAtStartPar
Runway orientation, dimensions and surface

\item {} 
\sphinxAtStartPar
Field elevation

\item {} 
\sphinxAtStartPar
Data for associated airspace

\end{itemize}


\section{Approach and Departure Routes}
\label{\detokenize{03-reference/airspace_display:approach-and-departure-routes}}
\sphinxAtStartPar
Approach routes to airfields are shown as solid blue lines. The designation of the route is written along the paths. The associated reporting points are shown as blue triangles with a dashed circle and the reporting point designation.
Approach Routes will be shown by a solid line and Departure Routes will be shown as  dashed lines.
Note
Approach Routes will only be displayed when zooming into the area.
Traffic Pattern
Traffic pattern for motorized aircraft are shown as blue lines.
Traffic circuits for gliders or Ultralight aircraft are shown as red lines.
Entry and exit routes to traffic pattern are indicated by open ends of the pattern.
The traffic circuit will show the traffic circuit altitude when the information is available.
Note
Traffic pattern will only be displayed when zooming into the area.


\chapter{Weather}
\label{\detokenize{03-reference/weather:weather}}\label{\detokenize{03-reference/weather::doc}}
\sphinxAtStartPar
The Weather page is opened via the Menu by touching the “Weather” entry.
The Weather page will display the station overview list for all currently available meteorological reports within 200 NM of the current position.

\begin{figure}[htbp]
\centering

\noindent\sphinxincludegraphics{{fig_Weather}.png}
\end{figure}

\sphinxAtStartPar
\sphinxstyleemphasis{Legend}:
\begin{enumerate}
\sphinxsetlistlabels{\arabic}{enumi}{enumii}{}{.}%
\item {} 
\sphinxAtStartPar
Weather Menu

\item {} 
\sphinxAtStartPar
Station data

\item {} 
\sphinxAtStartPar
Meteorological data closest to own position

\end{enumerate}

\sphinxAtStartPar
The weather data is downloaded from the National Weather Service of the United States of America.

\begin{sphinxadmonition}{note}{Note:}
\sphinxAtStartPar
When opening the Weather page the first time you will have to confirm that you agree to download data from the NWS server to use this service.
\end{sphinxadmonition}

\sphinxAtStartPar
The menu of the Waether page will allow to:
\begin{itemize}
\item {} 
\sphinxAtStartPar
Update the METAR and TAF data

\item {} 
\sphinxAtStartPar
Disallow he internet connection

\end{itemize}

\sphinxAtStartPar
The Weather overview window will provide the following information based on the METAR:
\begin{itemize}
\item {} 
\sphinxAtStartPar
ICAO identifier for Station and Airport name

\item {} 
\sphinxAtStartPar
Distance and magnetic Bearing to Airport

\item {} 
\sphinxAtStartPar
Time of METAR and summary weather state

\end{itemize}

\sphinxAtStartPar
On the lower end of the weather page the following data relevant to your current position will be displayed:
\begin{itemize}
\item {} 
\sphinxAtStartPar
QNH

\item {} 
\sphinxAtStartPar
Location and time of the report the QNH was extracted

\item {} 
\sphinxAtStartPar
Sunset during day or Sunrise during night at current location

\item {} 
\sphinxAtStartPar
Remaining time until sunset or sunrise

\end{itemize}

\sphinxAtStartPar
The information of each airport will be color coded by a system established by the US National Weather Service. The coding scheme is explained in the table below.
When touching a station line METAR and TAF (if available) will be shown in a weather detail sub\sphinxhyphen{}page

\begin{figure}[htbp]
\centering

\noindent\sphinxincludegraphics{{fig_WeatherDetail}.png}
\end{figure}

\sphinxAtStartPar
\sphinxstyleemphasis{Legend}:
\begin{enumerate}
\sphinxsetlistlabels{\arabic}{enumi}{enumii}{}{.}%
\item {} 
\sphinxAtStartPar
Station data including bering and distance

\item {} 
\sphinxAtStartPar
Current meteorological report

\item {} 
\sphinxAtStartPar
Decoded view of Current meteorological report

\item {} 
\sphinxAtStartPar
Weather forecast for station

\item {} 
\sphinxAtStartPar
Decoded view of weather forecast

\end{enumerate}

\begin{sphinxadmonition}{note}{Note:}
\sphinxAtStartPar
To view the full weather forecast you have to scroll down in most cases
\end{sphinxadmonition}

\begin{sphinxadmonition}{caution}{Caution:}
\sphinxAtStartPar
The color coding used for station weather does not match to European VFR criteria. Assessment of  meteorological flight conditions has to be done via an officially approved source of flight weather.
\end{sphinxadmonition}


\begin{savenotes}\sphinxattablestart
\centering
\begin{tabulary}{\linewidth}[t]{|T|T|T|T|T|}
\hline
\sphinxstyletheadfamily 
\sphinxAtStartPar
Category
&\sphinxstyletheadfamily 
\sphinxAtStartPar
Color
&\sphinxstyletheadfamily 
\sphinxAtStartPar
Ceiling
&\sphinxstyletheadfamily &\sphinxstyletheadfamily 
\sphinxAtStartPar
Visibility
\\
\hline
\sphinxAtStartPar
IFR
Instrument Flight Rules
&
\sphinxAtStartPar
Red
&
\sphinxAtStartPar
500 to below
1,000 feet AGL
&
\sphinxAtStartPar
and
/or
&
\sphinxAtStartPar
1 mile to
less than 3 miles
\\
\hline
\sphinxAtStartPar
MVFR
Marginal Visual Flight Rules
&
\sphinxAtStartPar
Yellow
&
\sphinxAtStartPar
1,000 to
3,000 feet AGL
&
\sphinxAtStartPar
and
/or
&
\sphinxAtStartPar
3 to 5 miles
\\
\hline
\sphinxAtStartPar
VFR
Visual Flight Rules
&
\sphinxAtStartPar
Green
&
\sphinxAtStartPar
greater than
3,000 feet AGL
&
\sphinxAtStartPar
and
/or
&
\sphinxAtStartPar
greater than
5 miles
\\
\hline
\end{tabulary}
\par
\sphinxattableend\end{savenotes}

\begin{sphinxadmonition}{note}{Note:}
\sphinxAtStartPar
By definition, IFR is ceiling less than 1,000 feet AGL.
\end{sphinxadmonition}

\begin{sphinxadmonition}{note}{Note:}
\sphinxAtStartPar
By definition, VFR is ceiling greater than or equal to 3,000 feet AGL and visibility greater than or equal to 5 miles while MVFR is a sub\sphinxhyphen{}category of VFR.
\end{sphinxadmonition}

\part{Appendix}


\chapter{Software licenses}
\label{\detokenize{04-appendix/licenses:software-licenses}}\label{\detokenize{04-appendix/licenses::doc}}

\section{License of Enroute Flight Navigation}
\label{\detokenize{04-appendix/licenses:license-of-enroute-flight-navigation}}
\sphinxAtStartPar
The program \sphinxstylestrong{Enroute Flight Navigation} is licensed under the \sphinxhref{https://www.gnu.org/licenses/gpl-3.0-standalone.html}{GNU General
Public License V3}%
\begin{footnote}[14]\sphinxAtStartFootnote
\sphinxnolinkurl{https://www.gnu.org/licenses/gpl-3.0-standalone.html}
%
\end{footnote} or,
at your choice, any later version of this license.


\section{Third\sphinxhyphen{}Party software included in this program}
\label{\detokenize{04-appendix/licenses:third-party-software-included-in-this-program}}\begin{itemize}
\item {} 
\sphinxAtStartPar
This program includes several libraries from the \sphinxhref{https://qt.io}{Qt project}%
\begin{footnote}[15]\sphinxAtStartFootnote
\sphinxnolinkurl{https://qt.io}
%
\end{footnote}, licensed under the \sphinxhref{https://www.qt.io/download-open-source}{GNU Lesser General Public License
(LGPL) version 3}%
\begin{footnote}[16]\sphinxAtStartFootnote
\sphinxnolinkurl{https://www.qt.io/download-open-source}
%
\end{footnote}.

\item {} 
\sphinxAtStartPar
This program includes the library \sphinxhref{https://github.com/nitroshare/qhttpengine}{qhttpengine}%
\begin{footnote}[17]\sphinxAtStartFootnote
\sphinxnolinkurl{https://github.com/nitroshare/qhttpengine}
%
\end{footnote}, which is licensed under the
\sphinxhref{https://github.com/nitroshare/qhttpengine/blob/master/LICENSE.txt}{MIT license}%
\begin{footnote}[18]\sphinxAtStartFootnote
\sphinxnolinkurl{https://github.com/nitroshare/qhttpengine/blob/master/LICENSE.txt}
%
\end{footnote}.

\item {} 
\sphinxAtStartPar
This program includes the library \sphinxhref{https://openssl.org}{OpenSSL}%
\begin{footnote}[19]\sphinxAtStartFootnote
\sphinxnolinkurl{https://openssl.org}
%
\end{footnote}, licensed
under the \sphinxhref{https://www.openssl.org/source/license.html}{Apache License 2.0}%
\begin{footnote}[20]\sphinxAtStartFootnote
\sphinxnolinkurl{https://www.openssl.org/source/license.html}
%
\end{footnote}.

\end{itemize}


\section{Data included in this program}
\label{\detokenize{04-appendix/licenses:data-included-in-this-program}}\begin{itemize}
\item {} 
\sphinxAtStartPar
This program includes versions of the \sphinxhref{https://github.com/google/roboto}{Google Roboto Fonts}%
\begin{footnote}[21]\sphinxAtStartFootnote
\sphinxnolinkurl{https://github.com/google/roboto}
%
\end{footnote}, which are licensed under the \sphinxhref{https://github.com/google/roboto/blob/master/LICENSE}{Apache
License 2.0}%
\begin{footnote}[22]\sphinxAtStartFootnote
\sphinxnolinkurl{https://github.com/google/roboto/blob/master/LICENSE}
%
\end{footnote}.

\item {} 
\sphinxAtStartPar
This program includes several \sphinxhref{https://github.com/google/material-design-icons}{Google Material Design Icons}%
\begin{footnote}[23]\sphinxAtStartFootnote
\sphinxnolinkurl{https://github.com/google/material-design-icons}
%
\end{footnote}, which are licensed under
the \sphinxhref{https://github.com/google/material-design-icons/blob/master/LICENSE}{Apache License 2.0}%
\begin{footnote}[24]\sphinxAtStartFootnote
\sphinxnolinkurl{https://github.com/google/material-design-icons/blob/master/LICENSE}
%
\end{footnote}.

\item {} 
\sphinxAtStartPar
The style specification of the basemap is a modified version of the \sphinxhref{https://github.com/maputnik/osm-liberty}{OSM
liberty map design}%
\begin{footnote}[25]\sphinxAtStartFootnote
\sphinxnolinkurl{https://github.com/maputnik/osm-liberty}
%
\end{footnote}, which is in
turn originally derived from OSM Bright from Mapbox Open Styles. The code is
licensed under the \sphinxhref{https://github.com/maputnik/osm-liberty/blob/gh-pages/LICENSE.md}{BSD license}%
\begin{footnote}[26]\sphinxAtStartFootnote
\sphinxnolinkurl{https://github.com/maputnik/osm-liberty/blob/gh-pages/LICENSE.md}
%
\end{footnote}. The OSM
style Bright from Mapbox Open Styles is licensed under the \sphinxhref{https://github.com/maputnik/osm-liberty/blob/gh-pages/LICENSE.md}{Creative Commons
Attribution 3.0 license}%
\begin{footnote}[27]\sphinxAtStartFootnote
\sphinxnolinkurl{https://github.com/maputnik/osm-liberty/blob/gh-pages/LICENSE.md}
%
\end{footnote}.

\end{itemize}


\section{Base maps}
\label{\detokenize{04-appendix/licenses:base-maps}}\begin{itemize}
\item {} 
\sphinxAtStartPar
The base maps are modified data from \sphinxhref{https://github.com/openmaptiles/openmaptiles}{OpenMapTiles}%
\begin{footnote}[28]\sphinxAtStartFootnote
\sphinxnolinkurl{https://github.com/openmaptiles/openmaptiles}
%
\end{footnote}, published under a \sphinxhref{https://github.com/openmaptiles/openmaptiles/blob/master/LICENSE.md}{CC\sphinxhyphen{}BY 4.0
design license}%
\begin{footnote}[29]\sphinxAtStartFootnote
\sphinxnolinkurl{https://github.com/openmaptiles/openmaptiles/blob/master/LICENSE.md}
%
\end{footnote}.

\end{itemize}


\section{Aviation maps}
\label{\detokenize{04-appendix/licenses:aviation-maps}}\begin{itemize}
\item {} 
\sphinxAtStartPar
The aviation maps contain data from \sphinxhref{http://www.openaip.net}{openAIP}%
\begin{footnote}[30]\sphinxAtStartFootnote
\sphinxnolinkurl{http://www.openaip.net}
%
\end{footnote},
licensed under a \sphinxhref{https://creativecommons.org/licenses/by-nc-sa/3.0/}{CC BY\sphinxhyphen{}NC\sphinxhyphen{}SA license}%
\begin{footnote}[31]\sphinxAtStartFootnote
\sphinxnolinkurl{https://creativecommons.org/licenses/by-nc-sa/3.0/}
%
\end{footnote}.

\item {} 
\sphinxAtStartPar
The aviation maps contain data from \sphinxhref{https://www.openflightmaps.org/}{open flightmaps}%
\begin{footnote}[32]\sphinxAtStartFootnote
\sphinxnolinkurl{https://www.openflightmaps.org/}
%
\end{footnote}, licensed under the \sphinxhref{https://www.openflightmaps.org/live/downloads/20150306-LCN.pdf}{OFMA General Users´
License}%
\begin{footnote}[33]\sphinxAtStartFootnote
\sphinxnolinkurl{https://www.openflightmaps.org/live/downloads/20150306-LCN.pdf}
%
\end{footnote}.

\end{itemize}

\sphinxAtStartPar
———\sphinxhyphen{}{\color{red}\bfseries{}|\sphinxhyphen{}\sphinxhyphen{}\sphinxhyphen{}\sphinxhyphen{}\sphinxhyphen{}\sphinxhyphen{}\sphinxhyphen{}\sphinxhyphen{}\sphinxhyphen{}\sphinxhyphen{}\sphinxhyphen{}\sphinxhyphen{}\sphinxhyphen{}\sphinxhyphen{}\sphinxhyphen{}\sphinxhyphen{}\sphinxhyphen{}\sphinxhyphen{}\sphinxhyphen{}\sphinxhyphen{}\sphinxhyphen{}\sphinxhyphen{}\sphinxhyphen{}\sphinxhyphen{}\sphinxhyphen{}\sphinxhyphen{}\sphinxhyphen{}\sphinxhyphen{}\sphinxhyphen{}\sphinxhyphen{}\sphinxhyphen{}\sphinxhyphen{}\sphinxhyphen{}\sphinxhyphen{}\sphinxhyphen{}\sphinxhyphen{}\sphinxhyphen{}\sphinxhyphen{}\sphinxhyphen{}
Component |License
\sphinxhyphen{}\sphinxhyphen{}\sphinxhyphen{}\sphinxhyphen{}\sphinxhyphen{}\sphinxhyphen{}\sphinxhyphen{}\sphinxhyphen{}\sphinxhyphen{}\sphinxhyphen{}|}—————————————
Assimp \sphinxhyphen{} Open Asset Import Library|BSD 3\sphinxhyphen{}Clause “New” or “Revised” License
Assimp \sphinxhyphen{} Clipper|Boost Software License 1.0
Assimp \sphinxhyphen{} irrXML|zlib License
Assimp \sphinxhyphen{} Open3DGC|MIT License and BSD 2\sphinxhyphen{}Clause “Simplified” License
Assimp \sphinxhyphen{} The OpenDDL\sphinxhyphen{}Parser|MIT License
Assimp \sphinxhyphen{} Poly2Tri Polygon Triangulation Library|BSD 3\sphinxhyphen{}clause “New” or “Revised” License
Assimp \sphinxhyphen{} RapidJSON|MIT License and BSD 3\sphinxhyphen{}Clause “New” or “Revised” License
Assimp \sphinxhyphen{} Unzip|zlib License
Assimp \sphinxhyphen{} Utf8Cpp|Boost Software License 1.0
Assimp \sphinxhyphen{} Zip|Public Domain
Dear ImGui|MIT License
Dear ImGui \sphinxhyphen{} ProggyClean.ttf|MIT License
Dear ImGui \sphinxhyphen{} stb|MIT License or Public Domain
Native Style for Android|Apache License 2.0
ANGLE Library|BSD 3\sphinxhyphen{}clause “New” or “Revised” License
ANGLE: Array Bounds Clamper for WebKit|BSD 2\sphinxhyphen{}clause “Simplified” License
ANGLE: Murmurhash|Public Domain
ANGLE: Systeminfo|BSD 2\sphinxhyphen{}clause “Simplified” License
ANGLE: trace\_event|BSD 3\sphinxhyphen{}clause “New” or “Revised” License
ANGLE: Khronos Headers|MIT License
Efficient Binary\sphinxhyphen{}Decimal and Decimal\sphinxhyphen{}Binary Conversion Routines for IEEE Doubles|BSD 3\sphinxhyphen{}clause “New” or “Revised” License
Easing Equations by Robert Penner|BSD 3\sphinxhyphen{}clause “New” or “Revised” License
forkfd|MIT License
FreeBSD strtoll and strtoull|BSD 3\sphinxhyphen{}clause “New” or “Revised” License
Freetype 2|Freetype Project License or GNU General Public License v2.0 only
Freetype 2 \sphinxhyphen{} zlib|zlib License
Freetype 2 \sphinxhyphen{} Bitmap Distribution Format (BDF) support|MIT License
Freetype 2 \sphinxhyphen{} Portable Compiled Format (PCF) support|MIT License
HarfBuzz|MIT License
HarfBuzz\sphinxhyphen{}NG|MIT License
IAccessible2 IDL Specification|BSD 3\sphinxhyphen{}clause “New” or “Revised” License
sRGB color profile icc file|International Color Consortium License
LibJPEG\sphinxhyphen{}turbo|Independent JPEG Group License
LibPNG|libpng License and PNG Reference Library version 2
MD4|Public Domain
MD4C|MIT License
MD5|Public Domain
PCRE2|BSD 3\sphinxhyphen{}clause “New” or “Revised” License
PCRE2 \sphinxhyphen{} Stack\sphinxhyphen{}less Just\sphinxhyphen{}In\sphinxhyphen{}Time Compiler|BSD 2\sphinxhyphen{}clause “Simplified” License
Pixman|MIT License
Secure Hash Algorithms SHA\sphinxhyphen{}384 and SHA\sphinxhyphen{}512|BSD 3\sphinxhyphen{}clause “New” or “Revised” License
Secure Hash Algorithm SHA\sphinxhyphen{}1|Public Domain
Secure Hash Algorithm SHA\sphinxhyphen{}3 \sphinxhyphen{} brg\_endian|BSD 2\sphinxhyphen{}clause “Simplified” License
Secure Hash Algorithm SHA\sphinxhyphen{}3 \sphinxhyphen{} Keccak|Creative Commons Zero v1.0 Universal
SQLite|Public Domain
TinyCBOR|MIT License
Vulkan Memory Allocator|MIT License
Bitstream Vera Font|Bitstream Vera Font License
DejaVu Fonts|Bitstream Vera Font License
Wintab API|LCS\sphinxhyphen{}Telegraphics License
XCB\sphinxhyphen{}XInput|MIT License
Data Compression Library (zlib)|zlib License
Text Codecs: Big5, Big5\sphinxhyphen{}HKSCS|BSD 2\sphinxhyphen{}clause “Simplified” License
Text Codec: EUC\sphinxhyphen{}JP|BSD 2\sphinxhyphen{}clause “Simplified” License
Text Codec: EUC\sphinxhyphen{}KR|BSD 2\sphinxhyphen{}clause “Simplified” License
Text Codec: ISO 2022\sphinxhyphen{}JP (JIS)|BSD 2\sphinxhyphen{}clause “Simplified” License
Text Codec: Shift\sphinxhyphen{}JIS|BSD 2\sphinxhyphen{}clause “Simplified” License
Text Codec: TSCII|BSD 2\sphinxhyphen{}clause “Simplified” License
Text Codec: GBK|BSD 2\sphinxhyphen{}clause “Simplified” License
The Public Suffix List|Mozilla Public License 2.0
QEventDispatcher on macOS|BSD 3\sphinxhyphen{}clause “New” or “Revised” License
Unicode Character Database (UCD)|Unicode License Agreement \sphinxhyphen{} Data Files and Software (2016)
Unicode Common Locale Data Repository (CLDR)|Unicode License Agreement \sphinxhyphen{} Data Files and Software (2016)
libdus\sphinxhyphen{}1 headers|Academic Free License v2.1, or GNU General Public License v2.0 or later
OpenGL Headers|MIT License
OpenGL ES 2 Headers|MIT License
Anti\sphinxhyphen{}aliasing rasterizer from FreeType 2|Freetype Project License or GNU General Public License v2.0 only
Smooth Scaling Algorithm|BSD 2\sphinxhyphen{}clause “Simplified” License and Imlib2 License
WebGradients|MIT License
X Server helper|X11 License and Historical Permission Notice and Disclaimer
Adobe Glyph List For New Fonts|BSD 3\sphinxhyphen{}Clause “New” or “Revised” License
Vulkan API Registry|MIT License
Cocoa Platform Plugin|BSD 3\sphinxhyphen{}clause “New” or “Revised” License
Valgrind|BSD 4\sphinxhyphen{}clause “Original” or “Old” License
Cycle|MIT License
Linux Performance Events|GNU General Public License v2.0 only with Linux Syscall Note
BlueZ|GNU General Public License v2.0 only (This does not force user code to be GPL’ed. For more info see details.)
JavaScriptCore Macro Assembler|BSD 2\sphinxhyphen{}clause “Simplified” License
TIFF Software Distribution (libtiff)|libtiff License
WebP (libwebp)|BSD 3\sphinxhyphen{}clause “New” or “Revised” License
Clip2Tri Polygon Triangulation Library|MIT License
Clipper Polygon Clipping Library|Boost Software License 1.0
Earcut Polygon Triangulation Library|ISC License
geosimplify\sphinxhyphen{}js polyline simplification library|geosimplify\sphinxhyphen{}js License
Mapbox GL Native|BSD 2\sphinxhyphen{}clause “Simplified” License and zlib License
CSS Color Parser|MIT License
cURL Parse Date|MIT License
Boost|Boost Software License 1.0
Earcut|ISC License
geojson\sphinxhyphen{}cpp|ISC License
geojson\sphinxhyphen{}vt\sphinxhyphen{}cpp|ISC License
geometry.hpp|ISC License
kdbush.hpp|ISC License
Optional|Boost Software License 1.0
polylabel|ISC License
protozero|BSD 2\sphinxhyphen{}clause “Simplified” License
RapidJSON|MIT License
shelf\sphinxhyphen{}pack\sphinxhyphen{}cpp|ISC License
supercluster.hpp|ISC License
tao\_tuple|MIT License
unique\_resource|Boost Software License 1.0
variant|BSD 3\sphinxhyphen{}clause “New” or “Revised” License
Vector Tile Library|ISC License
Wagyu Geometry Processing Library|MIT License
nunicode|MIT License
Poly2Tri Polygon Triangulation Library|BSD 3\sphinxhyphen{}clause “New” or “Revised” License
In\sphinxhyphen{}app billing service|Apache License 2.0
Base64 Decoder|Apache License 2.0
Public Key Verification|Apache License 2.0
Open Asset Import Library|BSD 3\sphinxhyphen{}clause “New” or “Revised” Licensee
Shadow values from Angular Material|MIT License
JavaScriptCore|GNU Library General Public License v2 or later
XSVG|Historical Permission Notice and Disclaimer \sphinxhyphen{} sell variant
Example Attribution|GNU General Public License v3.0 only
Lipi Toolkit|MIT License
OpenWnn|Apache License 2.0
PinyinIME|Apache License 2.0
Traditional Chinese IME (tcime)|Apache License 2.0 and BSD 3\sphinxhyphen{}clause “New” or “Revised” License
Wayland Fullscreen Shell Protocol|MIT License
Wayland Protocol|MIT License
Wayland IVI Extension Protocol|MIT License
Wayland Primary Selection Protocol|MIT License
Wayland Scaler Protocol|MIT License
Wayland Tablet Protocol|MIT License
Wayland Viewporter Protocol|MIT License
Wayland xdg\sphinxhyphen{}decoration Protocol|MIT License
Wayland XDG Output Protocol|MIT License
Wayland XDG Shell Protocol|MIT License
Wayland Text Input Protocol|HPND License
Wayland Linux Dmabuf Unstable V1 Protocol|MIT License
Wayland EGLStream Controller Protocol|MIT License
XML Schema|W3C Software Notice and Document License (2015\sphinxhyphen{}05\sphinxhyphen{}13)
———\sphinxhyphen{}{\color{red}\bfseries{}|}—————————————


\chapter{Technical Notes}
\label{\detokenize{04-appendix/technical:technical-notes}}\label{\detokenize{04-appendix/technical::doc}}

\section{Platform notes}
\label{\detokenize{04-appendix/technical:platform-notes}}

\subsection{Android}
\label{\detokenize{04-appendix/technical:android}}
\sphinxAtStartPar
Wi\sphinxhyphen{}Fi locking
\begin{quote}

\sphinxAtStartPar
When running on Android, the app acquires a Wi\sphinxhyphen{}Fi lock as soon as the app
receives heartbeat messages from one of the channels where it listens for
traffic receivers.  The lock is released when the messages no longer arrive.
\end{quote}


\section{Traffic Receiver}
\label{\detokenize{04-appendix/technical:traffic-receiver}}

\subsection{Communication}
\label{\detokenize{04-appendix/technical:communication}}
\sphinxAtStartPar
\sphinxstylestrong{Enroute Flight Navigation} expects that the traffic receiver deploys a WLAN
network via Wi\sphinxhyphen{}Fi and publishes traffic data via that network.  In order to
support a wide range of devices, including flight simulators, the app listens to
several network addresses simultaneously and understands a variety of protocols.

\sphinxAtStartPar
\sphinxstylestrong{Enroute Flight Navigation} watches the following data channels, in order of
preference.
\begin{itemize}
\item {} 
\sphinxAtStartPar
A TCP connection to port 2000 at the IP addresses 192.168.1.1, where the app
expects a stream of FLARM/NMEA sentences.

\item {} 
\sphinxAtStartPar
A TCP connection to port 2000 at the IP addresses 192.168.10.1, where the app
expects a stream of FLARM/NMEA sentences.

\item {} 
\sphinxAtStartPar
A UDP connection to port 4000, where the app expects datagrams in GDL90 or
XGPS format.

\item {} 
\sphinxAtStartPar
A UDP connection to port 49002, where the app expects datagrams in GDL90 or
XGPS format.

\end{itemize}

\sphinxAtStartPar
\sphinxstylestrong{Enroute Flight Navigation} expects traffic data in the following formats.
\begin{itemize}
\item {} 
\sphinxAtStartPar
FLARM/NMEA sentences must conform to the specification outlined in the
document FTD\sphinxhyphen{}012 \sphinxhref{https://flarm.com/support/manuals-documents/}{Data Port Interface Control Document (ICD)}%
\begin{footnote}[34]\sphinxAtStartFootnote
\sphinxnolinkurl{https://flarm.com/support/manuals-documents/}
%
\end{footnote}, Version 7.13, as published
by \sphinxhref{https://flarm.com/}{FLARM Technology Ltd.}%
\begin{footnote}[35]\sphinxAtStartFootnote
\sphinxnolinkurl{https://flarm.com/}
%
\end{footnote}.

\item {} 
\sphinxAtStartPar
Datagrams in GDL90 format must conform to the \sphinxhref{https://www.faa.gov/nextgen/programs/adsb/archival/media/gdl90\_public\_icd\_reva.pdf}{GDL 90 Data Interface
Specification}%
\begin{footnote}[36]\sphinxAtStartFootnote
\sphinxnolinkurl{https://www.faa.gov/nextgen/programs/adsb/archival/media/gdl90\_public\_icd\_reva.pdf}
%
\end{footnote}.

\item {} 
\sphinxAtStartPar
Datagrams in XGPS format must conform to the format specified on the
\sphinxhref{https://www.foreflight.com/support/network-gps/}{ForeFlight Web site}%
\begin{footnote}[37]\sphinxAtStartFootnote
\sphinxnolinkurl{https://www.foreflight.com/support/network-gps/}
%
\end{footnote}.

\end{itemize}


\subsection{Known issues}
\label{\detokenize{04-appendix/technical:known-issues}}
\sphinxAtStartPar
The GDL90 protocol has a number of shortcomings, and we recommend to use
FLARM/NMEA whenever possible.  We are aware of the following issues.

\sphinxAtStartPar
Altitude measurements
\begin{quote}

\sphinxAtStartPar
According to the GDL90 Specification, the ownship geometric height is reported
as height above WGS\sphinxhyphen{}84 ellipsoid.  There are however many devices on the
market that wrongly report height above main sea level.  Different apps have
different strategies to deal with these shortcomings.
\begin{itemize}
\item {} 
\sphinxAtStartPar
\sphinxstylestrong{Enroute Flight Navigation} as well as the app Skydemon expect that
traffic receivers comply with the GDL90 Specification.

\item {} 
\sphinxAtStartPar
ForeFlight has extended the GDL90 Specification so that traffic receivers
can indicate if they comply with the specification or not.

\item {} 
\sphinxAtStartPar
Many other apps expect wrong GDL90 implementations and interpret the
geometric height has height above main sea level.

\end{itemize}
\end{quote}

\sphinxAtStartPar
MODE\sphinxhyphen{}S traffic
\begin{quote}

\sphinxAtStartPar
Most traffic receivers see traffic equipped with MODE\sphinxhyphen{}S transponders and can
give an estimate for the distance to the traffic.  They are, however, unable
to obtain the precise traffic position.  Unlike FLARM/NMEA, the GDL90
Specification does not support traffic factors whose position is unknown.
Different devices implement different workarounds.
\begin{itemize}
\item {} 
\sphinxAtStartPar
Stratux devices generate a ring of eight virtual targets around the own
position.  These targets are named “Mode S”.

\item {} 
\sphinxAtStartPar
Air Avioncs devices do the same, but only with one target.

\item {} 
\sphinxAtStartPar
Other devices create a virtual target, either at the ownship position or at
the north pole and abuse the field “Navigation Accuracy Category for
Position” to give the approximate position to the target.

\end{itemize}

\sphinxAtStartPar
\sphinxstylestrong{Enroute Flight Navigation} has special provisions for handling targets
called “Mode S”, but users should expect that this workaround is not perfect.
\end{quote}


\subsection{ForeFlight Broadcast}
\label{\detokenize{04-appendix/technical:foreflight-broadcast}}
\sphinxAtStartPar
Following the standards established by the app ForeFlight, \sphinxstylestrong{Enroute Flight
Navigation} broadcasts a UDP message on port 63093 every 5 seconds while the
app is running in the foreground.  This message allows devices to discover
Enroute’s IP address, which can be used as the target of UDP unicast messages.
This broadcast will be a JSON message, with at least these fields:

\begin{sphinxVerbatim}[commandchars=\\\{\}]
\PYG{p}{\PYGZob{}}
   \PYG{n+nt}{\PYGZdq{}App\PYGZdq{}}\PYG{p}{:}\PYG{l+s+s2}{\PYGZdq{}Enroute Flight Navigation\PYGZdq{}}\PYG{p}{,}
   \PYG{n+nt}{\PYGZdq{}GDL90\PYGZdq{}}\PYG{p}{:}\PYG{p}{\PYGZob{}}
      \PYG{n+nt}{\PYGZdq{}port\PYGZdq{}}\PYG{p}{:}\PYG{l+m+mi}{4000}
   \PYG{p}{\PYGZcb{}}
\PYG{p}{\PYGZcb{}}
\end{sphinxVerbatim}

\sphinxAtStartPar
The GDL90 “port” field is currently 4000, but might change in the future.



\renewcommand{\indexname}{Index}
\footnotesize\raggedright\printindex
\end{document}