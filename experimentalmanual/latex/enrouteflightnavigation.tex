%% Generated by Sphinx.
\def\sphinxdocclass{report}
\documentclass[letterpaper,10pt,english]{sphinxmanual}
\ifdefined\pdfpxdimen
   \let\sphinxpxdimen\pdfpxdimen\else\newdimen\sphinxpxdimen
\fi \sphinxpxdimen=.75bp\relax

\PassOptionsToPackage{warn}{textcomp}
\usepackage[utf8]{inputenc}
\ifdefined\DeclareUnicodeCharacter
% support both utf8 and utf8x syntaxes
  \ifdefined\DeclareUnicodeCharacterAsOptional
    \def\sphinxDUC#1{\DeclareUnicodeCharacter{"#1}}
  \else
    \let\sphinxDUC\DeclareUnicodeCharacter
  \fi
  \sphinxDUC{00A0}{\nobreakspace}
  \sphinxDUC{2500}{\sphinxunichar{2500}}
  \sphinxDUC{2502}{\sphinxunichar{2502}}
  \sphinxDUC{2514}{\sphinxunichar{2514}}
  \sphinxDUC{251C}{\sphinxunichar{251C}}
  \sphinxDUC{2572}{\textbackslash}
\fi
\usepackage{cmap}
\usepackage[T1]{fontenc}
\usepackage{amsmath,amssymb,amstext}
\usepackage{babel}



\usepackage{times}
\expandafter\ifx\csname T@LGR\endcsname\relax
\else
% LGR was declared as font encoding
  \substitutefont{LGR}{\rmdefault}{cmr}
  \substitutefont{LGR}{\sfdefault}{cmss}
  \substitutefont{LGR}{\ttdefault}{cmtt}
\fi
\expandafter\ifx\csname T@X2\endcsname\relax
  \expandafter\ifx\csname T@T2A\endcsname\relax
  \else
  % T2A was declared as font encoding
    \substitutefont{T2A}{\rmdefault}{cmr}
    \substitutefont{T2A}{\sfdefault}{cmss}
    \substitutefont{T2A}{\ttdefault}{cmtt}
  \fi
\else
% X2 was declared as font encoding
  \substitutefont{X2}{\rmdefault}{cmr}
  \substitutefont{X2}{\sfdefault}{cmss}
  \substitutefont{X2}{\ttdefault}{cmtt}
\fi


\usepackage[Bjarne]{fncychap}
\usepackage{sphinx}

\fvset{fontsize=\small}
\usepackage{geometry}


% Include hyperref last.
\usepackage{hyperref}
% Fix anchor placement for figures with captions.
\usepackage{hypcap}% it must be loaded after hyperref.
% Set up styles of URL: it should be placed after hyperref.
\urlstyle{same}

\addto\captionsenglish{\renewcommand{\contentsname}{Getting started}}

\usepackage{sphinxmessages}
\setcounter{tocdepth}{1}



\title{Enroute Flight Navigation}
\date{Mar 05, 2021}
\release{2.2.4}
\author{Stefan Kebekus}
\newcommand{\sphinxlogo}{\vbox{}}
\renewcommand{\releasename}{Release}
\makeindex
\begin{document}

\pagestyle{empty}
\sphinxmaketitle
\pagestyle{plain}
\sphinxtableofcontents
\pagestyle{normal}
\phantomsection\label{\detokenize{index::doc}}


\noindent{\hspace*{\fill}\sphinxincludegraphics[width=100\sphinxpxdimen]{{de.akaflieg_freiburg.enroute}.png}\hspace*{\fill}}

\sphinxAtStartPar
Enroute Flight Navigation is a free flight navigation app for Android and other
devices. Designed to be simple, functional and elegant, it takes the stress out
of your next flight. The program has been written by flight enthusiasts, as a
project of \sphinxhref{https://akaflieg-freiburg.de/}{Akaflieg Freiburg}, a flight club
based in Freiburg, Germany.

\sphinxAtStartPar
Enroute Flight Navigation features a moving map, similar in style to the
official ICAO maps. Your current position and your flight path for the next five
minutes are marked, and so is your intended flight route. A double tap on the
display gives you all the information about airspaces, airfields and navaids \textendash{}
complete with frequencies, codes, elevations and runway information.

\sphinxAtStartPar
The free aeronautical maps can be downloaded for offline use. In addition to
airspaces, airfields and navaids, selected maps also show traffic circuits as
well as flight procedures for control zones. The maps receive near\sphinxhyphen{}weekly
updates and cover large parts of the world.

\sphinxAtStartPar
Enroute Flight Navigation includes flight weather data downloaded from the
\sphinxhref{https://www.aviationweather.gov/}{NOAA \sphinxhyphen{} Aviation Weather Center}.

\sphinxAtStartPar
While Enroute Flight Navigation is no substitute for full\sphinxhyphen{}featured flight
planning software, it allows you to quickly and easily compute distances,
courses and headings, and gives you an estimate for flight time and fuel
consumption. If the weather turns bad, the app will show you the closest
airfields for landing, complete with distances, directions, runway information
and frequencies.

\part{Getting started}


\chapter{Think before you fly}
\label{\detokenize{01-intro/think:think-before-you-fly}}\label{\detokenize{01-intro/think::doc}}
\sphinxAtStartPar
\sphinxstylestrong{Enroute Flight Navigation} is a free software product that has been published
in the hope that it might be useful as an aid to prudent navigation.  It comes
with no guarantees.  It may not work as expected.  Data shown to you might be
wrong.  Your hardware may fail.

\sphinxAtStartPar
This app is no substitute for proper flight preparation or good pilotage.  Any
information \sphinxstylestrong{must always} be validated using an official navigation and
airspace data source.

\begin{sphinxadmonition}{warning}{Warning:}
\sphinxAtStartPar
Always use official flight navigation data for flight preparation
and navigate by officially authorized means. The use of non\sphinxhyphen{}certified
navigation devices and software like \sphinxstylestrong{Enroute Flight Navigation} as
primary source of navigation may cause accidents leading to loss of lives.
\end{sphinxadmonition}

\sphinxAtStartPar
We do not believe that the use of \sphinxstylestrong{Enroute Flight Navigation} fulfills the
requirement of the EU Regulation \sphinxhref{https://eur-lex.europa.eu/LexUriServ/LexUriServ.do?uri=OJ:L:2012:281:0001:0066:EN:PDF}{No 923/2012:SERA.2010}
\begin{quote}

\sphinxAtStartPar
Before beginning a flight, the pilot\sphinxhyphen{}in\sphinxhyphen{}command of an aircraft shall become
familiar with all available information appropriate to the intended operation.
\end{quote}

\sphinxAtStartPar
To put it simply: relying on \sphinxstylestrong{Enroute Flight Navigation} as a primary means of
navigation is most likely illegal in your jurisdiction.  It is most certainly
stupid and potentially suicidal.


\section{Software limitations}
\label{\detokenize{01-intro/think:software-limitations}}
\sphinxAtStartPar
\sphinxstylestrong{Enroute Flight Navigation} is not an officially approved flight navigation
software.  It is not officially approved or certified in any way.  The software
comes with no guarantee and might contain bugs.


\section{Navigational data and aviation data}
\label{\detokenize{01-intro/think:navigational-data-and-aviation-data}}
\sphinxAtStartPar
Navigational\textendash{} and aviation data, including airspace and airfield information,
are provided “as is” and without any guarantee, official validation,
certification or warranty.  The data does not come from official sources.  It
might be incomplete, outdated or otherwise incorrect.


\section{Operating system limitations}
\label{\detokenize{01-intro/think:operating-system-limitations}}
\sphinxAtStartPar
We expect that most users will run the software on mobile phones or tablet
computers running the Android operating system.  Android is not officially
approved or certified for aviation.  While we expect that the app will run fine
for the vast majority of Android users, please keep the following in mind.
\begin{itemize}
\item {} 
\sphinxAtStartPar
The Android operating system can decide at any time to terminate \sphinxstylestrong{Enroute
Flight Navigation} or to slow it down to clear resources for other apps.

\item {} 
\sphinxAtStartPar
Other apps might interfere with the operation of \sphinxstylestrong{Enroute Flight
Navigation}.

\item {} 
\sphinxAtStartPar
Many hardware vendors, most notably One Plus, Huawei and Samsung equip their
phone with “battery saving apps” that randomly kill long\sphinxhyphen{}running processes.
These apps cannot be uninstalled by the users, do not comply with Android
standards and are often extremely buggy.  At times, users can manually excempt
apps from “battery saving mode”, but the settings are usually lost on system
updates.  Google’s own “Pixel” and “Nexus” devices do not have these problems.
See the web site \sphinxhref{https://dontkillmyapp.com}{Don’t kill my app} for more
information.

\end{itemize}


\section{Hardware limitations}
\label{\detokenize{01-intro/think:hardware-limitations}}
\sphinxAtStartPar
\sphinxstylestrong{Enroute Flight Navigation} runs on a variety of hardware platforms, but we
expect that most users will run the software on mobile phones, tablet computers
and comparable consumer electronic devices that are not certified to meet
aviation standards.  Keep the following in mind.
\begin{itemize}
\item {} 
\sphinxAtStartPar
Your device might not be designed to operate continuously for extended periods
of time, in particular if the display is on.

\item {} 
\sphinxAtStartPar
Your device can overheat. Batteries can catch fire.

\item {} 
\sphinxAtStartPar
Battery capacity is limited.  Even if your is connected to power via a USB
cable, note that there are devices where display and/or CPU use more energy
than USB can deliver.

\end{itemize}


\chapter{Getting started}
\label{\detokenize{01-intro/getting_started:getting-started}}\label{\detokenize{01-intro/getting_started::doc}}

\section{App Installation: Android}
\label{\detokenize{01-intro/getting_started:app-installation-android}}
\sphinxAtStartPar
\sphinxstylestrong{Enroute Flight Navigation} is available as an Android App in the \sphinxhref{https://play.google.com/store/apps/details?id=de.akaflieg\_freiburg.enroute}{Google Play
Store}.

\sphinxAtStartPar
An unofficial version of the app is also available at \sphinxhref{https://f-droid.org/de/packages/de.akaflieg\_freiburg.enroute/}{F\sphinxhyphen{}Droid}.  While the
author of \sphinxstylestrong{Enroute Flight Navigation} endorses publication at F\sphinxhyphen{}Droid, he has
not tested this unofficial app for quality.


\section{App Installation: Linux Desktop}
\label{\detokenize{01-intro/getting_started:app-installation-linux-desktop}}
\sphinxAtStartPar
Enroute Flight Navigation is available for free download at \sphinxhref{https://flathub.org/apps/details/de.akaflieg\_freiburg.enroute}{flathub.org} and
\sphinxhref{https://snapcraft.io/enroute-flight-navigation}{snapcraft.io}, but you might
also find the app in the software management application on your computer.


\section{Installing maps}
\label{\detokenize{01-intro/getting_started:installing-maps}}
\sphinxAtStartPar
To use \sphinxstylestrong{Enroute Flight Navigation} you have to install Maps covering the area
of flight.  For installing Maps the following steps have to be followed:
\begin{itemize}
\item {} 
\sphinxAtStartPar
Open the Menu by touching the area in the upper right side of the screen

\item {} 
\sphinxAtStartPar
Open the menu item ‘Settings’

\item {} 
\sphinxAtStartPar
touch the item ‘Maps’ in the \sphinxstyleemphasis{Libraries} section

\item {} 
\sphinxAtStartPar
Select the desired Maps by clicking the download Symbol

\end{itemize}

\sphinxAtStartPar
The Map display is composed of two layers selected in the respective Tabs of the
‘Map Library’ screen:
\begin{itemize}
\item {} 
\sphinxAtStartPar
Aeronautical Map

\item {} 
\sphinxAtStartPar
Base Map

\end{itemize}

\begin{sphinxadmonition}{caution}{Caution:}
\sphinxAtStartPar
Make sure Aeronautical and Base Map are installed for desired area
of flight to avoid flight into areas without map display.
\end{sphinxadmonition}

\sphinxAtStartPar
\sphinxstylestrong{Aeronautical Maps}

\sphinxAtStartPar
The Aeronautical Map layers is showing the airspace data on the Map screen. If no Base Map is installed for the area only the information coming from the Aviation Map data is displayed.

\sphinxAtStartPar
The Aeronautical Map contains:
\begin{itemize}
\item {} 
\sphinxAtStartPar
Airfields

\item {} 
\sphinxAtStartPar
Airspace boundaries

\item {} 
\sphinxAtStartPar
Navaids

\item {} 
\sphinxAtStartPar
Reporting points and routes (if available)

\end{itemize}

\sphinxAtStartPar
The display used for aerospace data is using the following basic color scheme:
\begin{itemize}
\item {} \begin{description}
\item[{Red:}] \leavevmode\begin{itemize}
\item {} 
\sphinxAtStartPar
Line with shadow inside for Restricted Airspace

\item {} 
\sphinxAtStartPar
Shadow with dashed blue border for Aerodrome Control Zone (CTR)

\item {} 
\sphinxAtStartPar
Dashed Line for Parachute Jumping Exercise area

\item {} 
\sphinxAtStartPar
Line for Glider or Microlight Traffic pattern

\end{itemize}

\end{description}

\item {} 
\sphinxAtStartPar
Blue:
\begin{itemize}
\item {} 
\sphinxAtStartPar
Line with shadow for controlled airspace (A, B, C, D)

\item {} 
\sphinxAtStartPar
Shadow with dashed blue border for Radio Mandatory Zone (RMZ)

\item {} 
\sphinxAtStartPar
Airport, reporting point or Navaid  symbols

\item {} 
\sphinxAtStartPar
For Route or Traffic Pattern for powered aircraft

\end{itemize}

\item {} 
\sphinxAtStartPar
Green:
\begin{itemize}
\item {} 
\sphinxAtStartPar
Line with shadow for Natural Reserve Area (NRA)

\item {} 
\sphinxAtStartPar
Line for airspace control sector boundaries

\end{itemize}

\item {} 
\sphinxAtStartPar
Black:
\begin{itemize}
\item {} 
\sphinxAtStartPar
Dashed Line for Transponder Mandatory Zone (TMZ)

\end{itemize}

\end{itemize}

\sphinxAtStartPar
\sphinxstylestrong{Class 1 and Class 2 maps:}
\begin{itemize}
\item {} 
\sphinxAtStartPar
Class 1 maps are compiled from openAIP and open flightmaps data. These maps
contain complete information about airspaces, airfields and navaids. In
addition, the maps contain (mandatory) reporting points. Some of our tier 1
maps also show traffic circuits and flight procedures for control zones.

\item {} 
\sphinxAtStartPar
Class 2 maps are compiled from openAIP data only. They contain complete
information about airspaces, airfields and navaids.

\end{itemize}

\sphinxAtStartPar
Details on the maps may be found at
\textless{}\sphinxurl{https://akaflieg-freiburg.github.io/enroute/maps/}\textgreater{} The Aeronautical Map data is
selected on the “Map Library” \textendash{} “Aviation Data” page accessed via the “Settings”
Menu.  To update the list of available maps the “…” option in the upper right
corner of the screen may be used.  You may install or uninstall the aviation Map
data for a county by the selection on the right hand side of the country
list. To find a country you have to scroll up and down in the list.

\begin{sphinxadmonition}{note}{Note:}
\sphinxAtStartPar
To have optimum presentation of the \sphinxstylestrong{Enroute Flight Navigation} map
display install the Aviation Map and the Base Map for all areas you intend
to use \sphinxstylestrong{Enroute Flight Navigation}.
\end{sphinxadmonition}

\begin{sphinxadmonition}{caution}{Caution:}
\sphinxAtStartPar
No airspace information will be provided in country when the
Aeronautical Map is not installed for it.
\end{sphinxadmonition}

\begin{sphinxadmonition}{note}{Note:}
\sphinxAtStartPar
\sphinxstylestrong{Enroute Flight Navigation} will automatically check for updated
Maps on the Enroute server and show a pop\sphinxhyphen{}up window after start if updated
maps have been detected.  You will be asked if you want to update the map or
delay the update.
\end{sphinxadmonition}

\sphinxAtStartPar
\sphinxstylestrong{Base Map}

\sphinxAtStartPar
The Base Map layers is showing the geographic data on the Map screen. If no Base Map is shown for an area it will be shown in the white background color. If no Aviation Map is installed for the area only the information coming from the Base Map data is displayed. The Base Map is organized in tiles. This will result in not stopping the Base Map display abruptly at the border of an installed country, but showing some overlap.
The Base Map will show:
\begin{itemize}
\item {} 
\sphinxAtStartPar
Landmass

\item {} 
\sphinxAtStartPar
Water Surface (oceans, lakes and rivers)

\item {} 
\sphinxAtStartPar
Forests

\item {} 
\sphinxAtStartPar
Main Roads

\item {} 
\sphinxAtStartPar
Railroad lines

\item {} 
\sphinxAtStartPar
City names

\end{itemize}

\begin{sphinxadmonition}{note}{Note:}
\sphinxAtStartPar
To have optimum presentation of the \sphinxstylestrong{Enroute Flight Navigation} map
display install the Aeronautical Map and the Base Map for all areas you
intend to use \sphinxstylestrong{Enroute Flight Navigation}.
\end{sphinxadmonition}

\begin{sphinxadmonition}{note}{Note:}
\sphinxAtStartPar
\sphinxstylestrong{Enroute Flight Navigation} will not show most cultural build ups
and limits or settled area boundaries to reduce the map size.
\end{sphinxadmonition}


\section{Flight mode and ground mode}
\label{\detokenize{01-intro/getting_started:flight-mode-and-ground-mode}}
\sphinxAtStartPar
\sphinxstylestrong{Ground Mode}

\sphinxAtStartPar
Ground Mode is displayed by \sphinxstylestrong{Enroute Flight Navigation} whenever the sensed
speed is below the threshold and the Menu item ‘Automatic Flight Detection’ is
not set to ‘Always in Flight Mode’.  Ground Mode does not display the Flight
Data line at the lower end of the screen and is intended for flight planning.

\begin{figure}[htbp]
\centering

\noindent\sphinxincludegraphics{{fig_GroundMode}.png}
\end{figure}

\sphinxAtStartPar
\sphinxstyleemphasis{Legend}:
\begin{enumerate}
\sphinxsetlistlabels{\arabic}{enumi}{enumii}{}{.}%
\item {} 
\sphinxAtStartPar
Own Position (No valid GPS position)

\item {} 
\sphinxAtStartPar
North Indicator, also area to switch between track up and north up

\item {} 
\sphinxAtStartPar
Zoom area to increase map scale (+) and reduce map scale (\sphinxhyphen{})

\item {} 
\sphinxAtStartPar
Map Scale reference indicator

\item {} 
\sphinxAtStartPar
Menu area

\end{enumerate}

\sphinxAtStartPar
There are two basic ways to plan a flight route:
\begin{itemize}
\item {} 
\sphinxAtStartPar
Menu \sphinxhyphen{} Route:
\begin{itemize}
\item {} 
\sphinxAtStartPar
Enter Waypoints

\item {} 
\sphinxAtStartPar
Edit existing Route

\item {} 
\sphinxAtStartPar
Enter Wind data

\item {} 
\sphinxAtStartPar
Enter Aircraft data

\end{itemize}

\item {} \begin{description}
\item[{Double touch Maps and open Waypoints}] \leavevmode\begin{itemize}
\item {} 
\sphinxAtStartPar
Direct will make a route from present position to Waypoint

\item {} 
\sphinxAtStartPar
‘+’ to Route will add the Waypoint to the current Route

\end{itemize}

\end{description}

\end{itemize}

\sphinxAtStartPar
A Route will remain in \sphinxstylestrong{Enroute Flight Navigation} until overwritten or
removed. Routes may be saved or shared.

\sphinxAtStartPar
\sphinxstylestrong{Flight Mode}

\sphinxAtStartPar
When \sphinxstylestrong{Enroute Flight Navigation} senses a speed above the threshold it will
automatically switch to flight mode.  For the displays given in flight mode
refer to Figure 3: Flight Mode (Track Up) In flight mode the following
additional items will be displayed:
* The own position will be changes from a dot to an arrow
* A segmented flight path for the next 5 minutes will be indicated
* A flight data line will indicate the following GPS data:
* Altitude in feet (or meters if metric units selected)
* Ground Speed in knots (or km/h if metric units selected)
* Track in reference to true north
* Universal Coordinated Time (UTC)

\begin{figure}[htbp]
\centering

\noindent\sphinxincludegraphics{{fig_FlightModeTu}.png}
\end{figure}

\sphinxAtStartPar
\sphinxstyleemphasis{Legend}:
\begin{enumerate}
\sphinxsetlistlabels{\arabic}{enumi}{enumii}{}{.}%
\item {} 
\sphinxAtStartPar
Own Position

\item {} 
\sphinxAtStartPar
Flight Path Vector (5 Minutes)

\item {} 
\sphinxAtStartPar
North Indicator, also area to switch between track up and north up

\item {} 
\sphinxAtStartPar
Center on Position area

\item {} 
\sphinxAtStartPar
Zoom area

\item {} 
\sphinxAtStartPar
Menu area

\end{enumerate}

\sphinxAtStartPar
The \sphinxstylestrong{Enroute Flight Navigation} map display is automatically centered to
display the own position to have about 80 \% of the display area in direction of
flight.  The map display may be shifted by touching the display and moving it to
the desired position. After shifting the “Center on Position” Symbol will be
displayed. After touching he “Center on Position” Symbol the map will be
centered to give maximum map area in direction of flight again.

\sphinxAtStartPar
\sphinxstylestrong{Track Up and North Up Mode}

\sphinxAtStartPar
The \sphinxstylestrong{Enroute Flight Navigation} map display may be switched between a Track Up
display and a North Up display by touching the gray window in the upper right
area.  Touching the display orientation area toggles between North up and Track
Up.

\begin{figure}[htbp]
\centering

\noindent\sphinxincludegraphics{{fig_FlightModeTu}.png}
\end{figure}

\sphinxAtStartPar
\sphinxstyleemphasis{Legend}:
\begin{enumerate}
\sphinxsetlistlabels{\arabic}{enumi}{enumii}{}{.}%
\item {} 
\sphinxAtStartPar
Own Position

\item {} 
\sphinxAtStartPar
Flight Path Vector (5 Minutes)

\item {} 
\sphinxAtStartPar
North Indicator, also area to switch between track up and north up

\item {} 
\sphinxAtStartPar
Zoom area

\item {} 
\sphinxAtStartPar
Scale

\item {} 
\sphinxAtStartPar
Menu area

\end{enumerate}

\sphinxAtStartPar
The North Up mode provides a map display always showing the map north up.  The
\sphinxstylestrong{Enroute Flight Navigation} map display in North Up mode will center the
display to provide about 80\% area in direction of flight.  In case the map
display has been manually rotated the area besides the direction arrow will show
the map orientation in degrees.


\section{Your first flight}
\label{\detokenize{01-intro/getting_started:your-first-flight}}
\sphinxAtStartPar
Now you are ready for the first use of \sphinxstylestrong{Enroute Flight Navigation}. General
operation is very intuitive. The primary purpose of \sphinxstylestrong{Enroute Flight
Navigation} of displaying a moving aeronautical map is directly available after
starting the app.  Before using the moving map function you have to make sure
the GPS of your mobile device is operating properly. The own position indicator
will be gray if GPS position is not available and will be displayed in blue
color if GPS position is available. The own position will be indicated as round
shape when no motion is sensed and displayed as arrow with flight path marker
when moving.

\begin{sphinxadmonition}{warning}{Warning:}
\sphinxAtStartPar
Make sure the GPS position is correct and valid to avoid loss of situational awareness. Loss of situational awareness is a common cause of severe accidents in aviation.
\end{sphinxadmonition}

\sphinxAtStartPar
To show a planned route on the moving map display you may:
\begin{enumerate}
\sphinxsetlistlabels{\arabic}{enumi}{enumii}{}{.}%
\item {} \begin{description}
\item[{Use ‘Direct’}] \leavevmode\begin{itemize}
\item {} 
\sphinxAtStartPar
Double Touch the desired Waypoint

\item {} 
\sphinxAtStartPar
Select ‘Direct’

\end{itemize}

\end{description}

\item {} \begin{description}
\item[{Plan a route}] \leavevmode\begin{itemize}
\item {} 
\sphinxAtStartPar
Double Touch the desired Waypoint

\item {} 
\sphinxAtStartPar
Select (+) ‘to Route’

\end{itemize}

\end{description}

\end{enumerate}

\sphinxAtStartPar
The planned route will be displayed as light green line on the map display. More detailed information on route planning will be given in the dedicated section.

\sphinxAtStartPar
\sphinxstylestrong{Airspace awareness}

\sphinxAtStartPar
Information related to any selected point on the Map will be displayed when double touching a point.

\sphinxAtStartPar
The displayed Information for arbitrary points will include:
\begin{itemize}
\item {} 
\sphinxAtStartPar
Distance to point

\item {} 
\sphinxAtStartPar
True bearing to point

\item {} 
\sphinxAtStartPar
Airspace classification including related frequencies and transponder code

\end{itemize}

\sphinxAtStartPar
The displayed Information for reporting points or Navaids will include:
\begin{itemize}
\item {} 
\sphinxAtStartPar
Distance to point

\item {} 
\sphinxAtStartPar
True bearing to point

\item {} 
\sphinxAtStartPar
Designation, controlling agency and radio frequencies

\item {} 
\sphinxAtStartPar
Airspace classification including related radio frequencies and transponder code

\end{itemize}

\sphinxAtStartPar
The displayed Information for airfields will include:
\begin{itemize}
\item {} 
\sphinxAtStartPar
Distance to point

\item {} 
\sphinxAtStartPar
True bearing to point

\item {} 
\sphinxAtStartPar
Meteorological information summary if available

\item {} 
\sphinxAtStartPar
Designation, controlling agency and radio frequencies and Navaids

\item {} 
\sphinxAtStartPar
Airfield data for Runways and field elevation

\item {} 
\sphinxAtStartPar
Airspace classification including related radio frequencies and transponder code

\end{itemize}

\sphinxAtStartPar
More information on the features and operation will be given in the ‘Further
Steps’ part of the \sphinxstylestrong{Enroute Flight Navigation} manual.

\sphinxAtStartPar
The following topics are described in more detail \sphinxstylestrong{Enroute Flight Navigation}
‘Reference’ section of the manual:
\begin{itemize}
\item {} 
\sphinxAtStartPar
Display of Airspace

\item {} 
\sphinxAtStartPar
Display of Aeronautical Data

\item {} 
\sphinxAtStartPar
Weather Data

\item {} 
\sphinxAtStartPar
Settings

\end{itemize}


\chapter{Further Steps}
\label{\detokenize{01-intro/further_steps:further-steps}}\label{\detokenize{01-intro/further_steps::doc}}
\sphinxAtStartPar
When using \sphinxstylestrong{Enroute Flight Navigation} on a frequent basis you may want to get
into some more detail on the features and configure the app.


\section{Menu}
\label{\detokenize{01-intro/further_steps:menu}}
\sphinxAtStartPar
Functions not directly accessible on the moving map display and more options are accessed via the menu symbol in the upper left edge of the moving map display.

\sphinxAtStartPar
When touching the Menu area in the left upper corner of the screen the menu will open and give the following options:
\begin{itemize}
\item {} 
\sphinxAtStartPar
Route — see Flight Routes described below

\item {} 
\sphinxAtStartPar
Nearby Waypoints  — will show the closest 20 aerodromes, navaids or reporting points

\item {} 
\sphinxAtStartPar
Weather   — will open the weather display

\item {} 
\sphinxAtStartPar
Set Altimeter — allows to enter current altitude to have altitude displayed using an offset

\item {} 
\sphinxAtStartPar
Settings — see below

\item {} \begin{description}
\item[{Information}] \leavevmode\begin{itemize}
\item {} 
\sphinxAtStartPar
Satellite Status  — will open a sub\sphinxhyphen{}window showing the status of GPS reception

\item {} 
\sphinxAtStartPar
Manual — will show this manual

\item {} 
\sphinxAtStartPar
About \sphinxstylestrong{Enroute Flight Navigation}

\item {} 
\sphinxAtStartPar
Participate

\item {} 
\sphinxAtStartPar
Donate

\end{itemize}

\end{description}

\item {} 
\sphinxAtStartPar
Bug Report — will open a link to provide feedback

\item {} 
\sphinxAtStartPar
Exit — will shut down the application

\end{itemize}

\sphinxAtStartPar
Only some reference to Menu items is given in this section. More details may be found in the ‘Reference’ section of  the manual.


\section{Settings}
\label{\detokenize{01-intro/further_steps:settings}}
\sphinxAtStartPar
The settings Menu will allow to customize \sphinxstylestrong{Enroute Flight Navigation} and give
access to program status.  The settings Menu gives the following options:
\begin{itemize}
\item {} 
\sphinxAtStartPar
\sphinxstylestrong{Hide Airspace above FL 100} — allows to select display of airspace above FL100 displayed.

\item {} 
\sphinxAtStartPar
\sphinxstylestrong{Automatic flight detection} — allows to select the display of flight mode on the ground

\item {} 
\sphinxAtStartPar
\sphinxstylestrong{Flight Routes} — will show a window with the previously stored routes

\item {} 
\sphinxAtStartPar
\sphinxstylestrong{Maps} — will show a window with the available and previously installed Aviation and Base Maps

\item {} 
\sphinxAtStartPar
\sphinxstylestrong{Use metric units} — allows to select display in metric units

\item {} 
\sphinxAtStartPar
\sphinxstylestrong{Use English} — allows to select English Language for display

\end{itemize}

\begin{sphinxadmonition}{note}{Note:}
\sphinxAtStartPar
The items to be selected by the on\sphinxhyphen{}off slider will enable the related function. The current status of the selected item is shown below the item.
\end{sphinxadmonition}

\begin{sphinxadmonition}{note}{Note:}
\sphinxAtStartPar
If you do not select “Hide Airspace above FL 100” the FIS frequencies for the Airspace C above FL100 will be displayed. In general this frequencies are also applicable below FL 100.
\end{sphinxadmonition}

\begin{sphinxadmonition}{note}{Note:}
\sphinxAtStartPar
When Automatic flight detection is not selected the display will always be in flight mode.
\end{sphinxadmonition}

\begin{sphinxadmonition}{note}{Note:}
\sphinxAtStartPar
If “Use English” is not selected the standard language selected for your device will be used if available.
\end{sphinxadmonition}


\section{Waypoints}
\label{\detokenize{01-intro/further_steps:waypoints}}
\sphinxAtStartPar
Waypoints are the central element of aeronautical navigation. A waypoint is selected by touching the moving map at the location of the waypoint display. Waypoints may also be directly added to a route by list selection or search.

\sphinxAtStartPar
The following types of Waypoints are available:
\begin{itemize}
\item {} 
\sphinxAtStartPar
Aerodromes

\item {} 
\sphinxAtStartPar
Reporting points

\item {} 
\sphinxAtStartPar
Navaids

\item {} 
\sphinxAtStartPar
Arbitrary points on the map.

\end{itemize}

\sphinxAtStartPar
For Aerodromes the full set of information will be displayed:
\begin{itemize}
\item {} 
\sphinxAtStartPar
Aerodrome symbol indicating type of airport

\item {} 
\sphinxAtStartPar
Aerodrome designation

\item {} 
\sphinxAtStartPar
Distance and Bearing to Aerodrome

\item {} 
\sphinxAtStartPar
Meteorological information if available

\item {} 
\sphinxAtStartPar
Aerodrome communication information

\item {} 
\sphinxAtStartPar
Airspace data

\end{itemize}

\sphinxAtStartPar
For Reporting Points the following information will be displayed:
\begin{itemize}
\item {} 
\sphinxAtStartPar
Reporting Point designation

\item {} 
\sphinxAtStartPar
Distance and bearing to Reporting Point

\item {} 
\sphinxAtStartPar
Reporting Point communication information

\item {} 
\sphinxAtStartPar
Airspace data

\end{itemize}

\sphinxAtStartPar
For Navaids the following information will be displayed:
\begin{itemize}
\item {} 
\sphinxAtStartPar
Navaid symbol, designation and type

\item {} 
\sphinxAtStartPar
Distance and bearing to Navaid

\item {} 
\sphinxAtStartPar
Navaid ID, frequency and elevation

\item {} 
\sphinxAtStartPar
Airspace data

\end{itemize}

\sphinxAtStartPar
For Arbitrary Points the following information will be displayed:
\begin{itemize}
\item {} 
\sphinxAtStartPar
Point designation if manually entered

\item {} 
\sphinxAtStartPar
Distance and bearing to Point

\item {} 
\sphinxAtStartPar
Airspace data

\end{itemize}

\begin{figure}[htbp]
\centering

\noindent\sphinxincludegraphics{{fig_waypoint}.png}
\end{figure}

\sphinxAtStartPar
\sphinxstyleemphasis{Legend}:
\begin{enumerate}
\sphinxsetlistlabels{\arabic}{enumi}{enumii}{}{.}%
\item {} 
\sphinxAtStartPar
Waypoint designation, bearing and distance

\item {} 
\sphinxAtStartPar
Waypoint meteorological information (only shown for airports if available)

\item {} 
\sphinxAtStartPar
Communication information related to waypoint (only shown if applicable)

\item {} 
\sphinxAtStartPar
Airspace information for waypoint

\item {} 
\sphinxAtStartPar
Area to select direct Navigation to waypoint

\item {} 
\sphinxAtStartPar
Area to add waypoint to current route

\end{enumerate}


\section{Flight Routes}
\label{\detokenize{01-intro/further_steps:flight-routes}}
\sphinxAtStartPar
\sphinxstylestrong{Enroute Flight Navigation} provides direct planning of one flight Route. A
Route will remain present until it is cleared.  Route planning is entered via
the Menu point Route. The Menu is entered via the Menu Symbol in the upper left
corner of the map area. Then the Route Symbol has to be touched to go to the
Route area.

\sphinxAtStartPar
A Route may be planned in the following ways:
\begin{itemize}
\item {} 
\sphinxAtStartPar
“Direct” in the waypoint window will provide a Route between current position and desired waypoint

\item {} 
\sphinxAtStartPar
“+” symbol in the waypoint window will add the waypoint to the last position of the Route.

\item {} 
\sphinxAtStartPar
“Add Waypoint” in the Route window will open a selection window for a waypoint and add the selected waypoint to the route.

\end{itemize}

\sphinxAtStartPar
The Route Display will show the following information:
\begin{itemize}
\item {} 
\sphinxAtStartPar
Symbol of the waypoint

\item {} 
\sphinxAtStartPar
Designation of the waypoint

\item {} 
\sphinxAtStartPar
Route Point Menu

\item {} \begin{description}
\item[{Navigation Data}] \leavevmode\begin{itemize}
\item {} 
\sphinxAtStartPar
Distance between way points

\item {} 
\sphinxAtStartPar
Time calculated between way points using the cruise speed set in the “Aircraft and Wind” page

\item {} 
\sphinxAtStartPar
True Course between way points

\item {} 
\sphinxAtStartPar
True Heading between way points

\end{itemize}

\end{description}

\end{itemize}

\begin{sphinxadmonition}{note}{Note:}
\sphinxAtStartPar
A Route may also be imported from a GPX file from another PC. After
sending the GPX file as Email attachment \sphinxstylestrong{Enroute Flight Navigation}
will offer to open the GPX file.
\end{sphinxadmonition}

\sphinxAtStartPar
The Route Point Menu provides the option to:
\begin{itemize}
\item {} 
\sphinxAtStartPar
Move a waypoint up in the Route

\item {} 
\sphinxAtStartPar
Move a waypoint down in the Route

\item {} 
\sphinxAtStartPar
Remove a waypoint from the Route

\end{itemize}

\sphinxAtStartPar
The Route Menu is entered by touching the Route Menu Symbol on the Route page.
For Arbitrary Points the standard designation “Waypoint” may be changed by touching the pencil symbol and entering a designation.

\sphinxAtStartPar
The following options are available from the Route Menu:
\begin{itemize}
\item {} 
\sphinxAtStartPar
Open a previously stored route from the library

\item {} 
\sphinxAtStartPar
Save the current route to the library

\item {} 
\sphinxAtStartPar
View the route library

\item {} 
\sphinxAtStartPar
Share the Route in JSON or GPX format

\item {} 
\sphinxAtStartPar
Open the Route in another APP using the JSON or GPX format

\item {} 
\sphinxAtStartPar
Clear Route

\item {} 
\sphinxAtStartPar
Reverse Route

\end{itemize}

\sphinxAtStartPar
The previously created and stored routes will be kept in a data base within
\sphinxstylestrong{Enroute Flight Navigation}. Routes consist of the data for the selected way
points. The Route data may be exported for use in other applications.
\begin{description}
\item[{The Route display has 3 Sub windows:}] \leavevmode\begin{itemize}
\item {} 
\sphinxAtStartPar
Route

\item {} 
\sphinxAtStartPar
Wind

\item {} 
\sphinxAtStartPar
ACFT

\end{itemize}

\end{description}

\begin{figure}[htbp]
\centering

\noindent\sphinxincludegraphics{{fig_route}.png}
\end{figure}

\sphinxAtStartPar
\sphinxstyleemphasis{Legend}:
\begin{enumerate}
\sphinxsetlistlabels{\arabic}{enumi}{enumii}{}{.}%
\item {} 
\sphinxAtStartPar
Route sub\sphinxhyphen{}window

\item {} 
\sphinxAtStartPar
Selection area for wind sub\sphinxhyphen{}window

\item {} 
\sphinxAtStartPar
Selection area for aircraft sub\sphinxhyphen{}window

\item {} 
\sphinxAtStartPar
Route point sub\sphinxhyphen{}menu

\item {} 
\sphinxAtStartPar
Edit route point designation for arbitrary waypoints

\item {} 
\sphinxAtStartPar
Total distance, flight time and fuel consumption for flight route

\end{enumerate}

\sphinxAtStartPar
.

\sphinxAtStartPar
\sphinxstylestrong{Route \textendash{} Aircraft and Wind}

\sphinxAtStartPar
The Aircraft and Wind sub\sphinxhyphen{}pages of the Route page allows to enter aircraft
performance and wind data required for navigational calculations.  The Aircraft
Data will be used to determine the distance of the flight and the true course.
The Wind data will will be used to calculate the true heading and duration of
the flight. The duration of the flight will determine the fuel used.  \sphinxstylestrong{Enroute
Flight Navigation} only offers a very superficial flight planning and cannot
replace a full flight planning, but is only intended to provide quick reference.

\begin{sphinxadmonition}{warning}{Warning:}
\sphinxAtStartPar
Always perform a full flight preparation in accordance with the flight
manual of the aircraft used. The use of \sphinxstylestrong{Enroute Flight Navigation}
as primary flight planning may cause accidents leading to loss of
lives.
\end{sphinxadmonition}
\begin{description}
\item[{The Wind sub\sphinxhyphen{}page of the Route page offers the following input fields:}] \leavevmode\begin{itemize}
\item {} 
\sphinxAtStartPar
Direction in degrees

\item {} 
\sphinxAtStartPar
Speed in knots

\end{itemize}

\end{description}

\sphinxAtStartPar
Only one speed, fuel consumption and wind may be entered for the whole route.

\sphinxAtStartPar
The Aircraft sub\sphinxhyphen{}page of the Route page offers the following input fields:
\begin{itemize}
\item {} \begin{description}
\item[{Aircraft}] \leavevmode\begin{itemize}
\item {} 
\sphinxAtStartPar
Cruise Speed: Average Speed for Route

\item {} 
\sphinxAtStartPar
Descent Speed: Allows to enter a different speed for the descent phase (Currently not used)

\item {} 
\sphinxAtStartPar
Fuel Consumption: Average Fuel consumption per hour

\end{itemize}

\end{description}

\end{itemize}


\chapter{Connecting your traffic receiver}
\label{\detokenize{02-steps/traffic:connecting-your-traffic-receiver}}\label{\detokenize{02-steps/traffic::doc}}
\sphinxAtStartPar
In order to display nearby traffic on the moving map, \sphinxstylestrong{Enroute Flight
Navigation} can connect to your aircraft’s traffic receiver (typically a FLARM
device).

\sphinxAtStartPar
The app author has tested the \sphinxstylestrong{Enroute Flight Navigation} with the following
traffic receivers.
\begin{itemize}
\item {} 
\sphinxAtStartPar
\sphinxhref{http://www.air-avionics.com/?page\_id=253}{AT\sphinxhyphen{}1 AIR Traffic} by \sphinxhref{http://www.air-avionics.com/}{Air
Avionics} with software version 5.

\end{itemize}

\sphinxAtStartPar
Users reported success with the following traffic receivers.
\begin{itemize}
\item {} 
\sphinxAtStartPar
\sphinxhref{http://stratux.me/}{Stratux devices}

\item {} 
\sphinxAtStartPar
\sphinxhref{https://www.amazon.de/TTGO-T-Beam-915Mhz-Wireless-Bluetooth/dp/B07SFVQ3Z8}{TTGO T\sphinxhyphen{}Beam devices}

\end{itemize}


\section{Before you connect}
\label{\detokenize{02-steps/traffic:before-you-connect}}
\sphinxAtStartPar
Before you try to connect this app to your traffic receiver, make sure that the
following conditions are met.
\begin{itemize}
\item {} 
\sphinxAtStartPar
Your traffic receiver has an integrated Wi\sphinxhyphen{}Fi interface that acts as a
wireless access point. Bluetooth devices are currently not supported.

\item {} 
\sphinxAtStartPar
You know the network name (=SSID) of the WLAN network deployed by your traffic
receiver. If the network is encrypted, you also need to know the WLAN
password.

\item {} 
\sphinxAtStartPar
If you use a Stratux or T\sphinxhyphen{}Beam device, set the device IP address to
192.168.1.1.  Most FLARM devices use this address default and  no
configuration is required.

\item {} 
\sphinxAtStartPar
Some devices require an additional password in order to access traffic
data. This is currently \sphinxstylestrong{not} supported. Set up your device so that no
additional password is required.

\end{itemize}


\section{Connecting to the traffic receiver}
\label{\detokenize{02-steps/traffic:connecting-to-the-traffic-receiver}}
\sphinxAtStartPar
It takes a two steps to connect \sphinxstylestrong{Enroute Flight Navigation} to the traffic
receiver for the first time. Once things are set up properly, your device should
automatically detect the traffic receiver’s WLAN network, enter the network and
connect to the traffic data stream whenever you go flying.

\sphinxAtStartPar
Step 1: Enter the traffic receiver’s WLAN network
\begin{itemize}
\item {} 
\sphinxAtStartPar
Make sure that the traffic receiver has power and is switched on. In a typical
aircraft installation, the traffic receiver is connected to the ‘Avionics’
switch and will automatically switch on. You may need to wait a minute before
the WLAN comes online and is visible to your device.

\item {} 
\sphinxAtStartPar
Enter the WLAN network deployed by your traffic receiver. This is usually done
in the “WLAN Settings” of your device. Enter the WLAN password if
required. Some devices will issue a warning that the WLAN is not connected to
the internet. In this case, you might need to confirm that you wish to enter
the WLAN network.

\end{itemize}

\sphinxAtStartPar
Most operating systems will offer to remember the connection, so that your
device will automatically connect to this WLAN in the future. We recommend to
use this option.

\sphinxAtStartPar
Step 2: Connect to the traffic data stream

\sphinxAtStartPar
Open the main menu and navigate to the “Information” menu.
\begin{itemize}
\item {} 
\sphinxAtStartPar
If the entry “Traffic Receiver” is highlighted in green, then \sphinxstylestrong{Enroute Flight
Navigation} has already found the traffic receiver in the network and has
connected to it. Congratulations, you are done!

\item {} 
\sphinxAtStartPar
If the entry “Traffic Receiver” is not highlighted in green, then select the
entry. The “Traffic Receiver Status” page will open. The page explains the
connection status in detail, and explains how to establish a connection
manually.

\end{itemize}


\section{Troubleshooting}
\label{\detokenize{02-steps/traffic:troubleshooting}}
\sphinxAtStartPar
\sphinxstylestrong{The app cannot connect to the traffic data stream.}
\begin{itemize}
\item {} 
\sphinxAtStartPar
Check that your device is connected to the WLAN network deployed by your
traffic receiver.

\end{itemize}

\sphinxAtStartPar
\sphinxstylestrong{The connection breaks down after a few seconds.}

\sphinxAtStartPar
Most traffic receivers cannot serve more than one client and abort connections
at random if more than one device tries to access.
\begin{itemize}
\item {} 
\sphinxAtStartPar
Make sure that there no second device connected to the traffic receiver’s WLAN
network. The other device might well be in your friend’s pocket!

\item {} 
\sphinxAtStartPar
Make sure that there is no other app trying to connected to the traffic
receiver’s data stream.

\item {} 
\sphinxAtStartPar
Many traffic receivers offer “configuration panels” that can be accessed via a
web browser. Close all web browsers.

\end{itemize}

 \part{Reference manual}


\chapter{Map Data}
\label{\detokenize{03-reference/map_data:map-data}}\label{\detokenize{03-reference/map_data::doc}}
\sphinxAtStartPar
The Information displayed by the Map of Enroute Flight Navigation is provided by the following resources:
\begin{itemize}
\item {} 
\sphinxAtStartPar
openAIP

\item {} 
\sphinxAtStartPar
open flightmaps

\item {} 
\sphinxAtStartPar
Map Tiler

\item {} 
\sphinxAtStartPar
Open Street Map

\end{itemize}
\begin{description}
\item[{To get more detailed Information on these Resources you may touch the link on the lower edge of the map Display \sphinxstylestrong{Map Data Copyright Info}. After touching the line \sphinxstylestrong{Map Data Copyright Info} a sub window will open showing links to the contributor web sites:}] \leavevmode\begin{itemize}
\item {} 
\sphinxAtStartPar
\sphinxurl{https://www.openaip.net}

\item {} 
\sphinxAtStartPar
\sphinxurl{https://www.openflightmaps.org}

\item {} 
\sphinxAtStartPar
\sphinxurl{https://www.maptiler.com}

\item {} 
\sphinxAtStartPar
\sphinxurl{https://www.openstreetmap.org}

\end{itemize}

\end{description}

\sphinxAtStartPar
\sphinxstylestrong{Open AIP}

\sphinxAtStartPar
Open AIP has the goal to deliver free, current and precise data for air navigation to everyone. Open AIP is a web based and crowd\sphinxhyphen{}sourced platform.
The Open AIP provides the basic source aeronautical data for display in Enroute Flight Navigation.

\sphinxAtStartPar
\sphinxstylestrong{Open Flight Maps}

\sphinxAtStartPar
Open Flight Maps is an open\sphinxhyphen{}source project providing aeronautical data for a high quality VFR Map.
Open Flight Maps is providing some additional information, where available.

\sphinxAtStartPar
The detailed split of the data sources for the Enroute Flight Naviagtion map is shown below:


\begin{savenotes}\sphinxattablestart
\centering
\begin{tabulary}{\linewidth}[t]{|T|T|}
\hline
\sphinxstyletheadfamily 
\sphinxAtStartPar
Map Feature
&\sphinxstyletheadfamily 
\sphinxAtStartPar
Data Origin
\\
\hline
\sphinxAtStartPar
Airfields
&
\sphinxAtStartPar
openAIP
\\
\hline
\sphinxAtStartPar
Airspace: Nature Preserve Areas
&
\sphinxAtStartPar
open flightmaps
\\
\hline
\sphinxAtStartPar
Airspace: all other
&
\sphinxAtStartPar
openAIP
\\
\hline
\sphinxAtStartPar
Navaids
&
\sphinxAtStartPar
openAIP
\\
\hline
\sphinxAtStartPar
Procedures (Traffic Circuits, …)
&
\sphinxAtStartPar
open flightmaps
\\
\hline
\sphinxAtStartPar
Reporting Points
&
\sphinxAtStartPar
open flightmaps
\\
\hline
\end{tabulary}
\par
\sphinxattableend\end{savenotes}

\sphinxAtStartPar
\sphinxstylestrong{Map Tiler}

\sphinxAtStartPar
Is a software application to combine multiple layers of data for maps and provide the map in a format for loading and display.
The Enroute Flight Naviagtion base maps are edited versions of maps kindly provided by Klokan Technologies through the OpenMapTiles project.

\sphinxAtStartPar
\sphinxstylestrong{Open Street Map}

\sphinxAtStartPar
Open Street Map (OSM) is a collaborative project to create a free editable map of the world. The geodata underlying the map is considered the primary output of the project. The creation and growth of OSM has been motivated by restrictions on use or availability of map data across much of the world, and the advent of inexpensive portable satellite navigation devices.
The Open Street Map data is used to crate the base maps.


\chapter{Airspace Display}
\label{\detokenize{03-reference/airspace_display:airspace-display}}\label{\detokenize{03-reference/airspace_display::doc}}
\sphinxAtStartPar
The display of airspace will generally follow the common ICAO symbology.
Restricted Airspace
Restricted airspace will be surrounded by an intense red dashed line and a thick transparent red line inside the restricted area boundaries.
When selecting a point inside the restricted area by double touching the screen the information to the related area is given with the waypoint pop\sphinxhyphen{}up window:
\begin{itemize}
\item {} 
\sphinxAtStartPar
Area Name

\item {} 
\sphinxAtStartPar
Area altitude limits

\item {} 
\sphinxAtStartPar
Area activation time

\end{itemize}

\begin{figure}[htbp]
\centering

\noindent\sphinxincludegraphics{{fig_Restricted}.png}
\end{figure}

\sphinxAtStartPar
\sphinxstyleemphasis{Legend}:
\begin{enumerate}
\sphinxsetlistlabels{\arabic}{enumi}{enumii}{}{.}%
\item {} 
\sphinxAtStartPar
Outline of Restricted Airspace

\item {} 
\sphinxAtStartPar
Designation and activation time of airspace

\end{enumerate}


\section{Controlled Airspace}
\label{\detokenize{03-reference/airspace_display:controlled-airspace}}
\sphinxAtStartPar
All boundaries of controlled airspace are shown by a solid blue line and a thick transparent blue line inside the airspace. Figure 13:  Controlled Airspace
When selecting a point inside the controlled airspace by double touching the screen the information to the related area is given with the waypoint pop\sphinxhyphen{}up window:
\begin{itemize}
\item {} 
\sphinxAtStartPar
Area Name

\item {} 
\sphinxAtStartPar
Area altitude limits

\end{itemize}

\begin{sphinxadmonition}{caution}{Caution:}
\sphinxAtStartPar
All controlled airspace (Class A \textendash{} Class D) are shown in the same way even if different restrictions or ATC clearance requirements may be present.
\end{sphinxadmonition}


\section{Control Zone}
\label{\detokenize{03-reference/airspace_display:control-zone}}
\sphinxAtStartPar
The Control Zone of an airport is shown with a dashed blue line filled in transparent red color. Figure 13:  Controlled Airspace
When selecting a point inside the Control Zone (CTR) by double touching the screen the information to the related area is given with the waypoint pop\sphinxhyphen{}up window:
\begin{itemize}
\item {} 
\sphinxAtStartPar
Area Name

\item {} 
\sphinxAtStartPar
Area altitude limits

\end{itemize}

\begin{figure}[htbp]
\centering

\noindent\sphinxincludegraphics{{fig_AirspaceMUC}.png}
\end{figure}

\sphinxAtStartPar
\sphinxstyleemphasis{Legend}:
\begin{enumerate}
\sphinxsetlistlabels{\arabic}{enumi}{enumii}{}{.}%
\item {} 
\sphinxAtStartPar
Airport ICAO Symbol

\item {} 
\sphinxAtStartPar
Airport Control Zone (CTR)

\item {} 
\sphinxAtStartPar
Radio Mandatory Zone (RMZ)

\item {} 
\sphinxAtStartPar
Boundary of Controlled Airspace

\item {} 
\sphinxAtStartPar
Restricted Airspace

\end{enumerate}


\section{Transponder Mandatory Zones}
\label{\detokenize{03-reference/airspace_display:transponder-mandatory-zones}}
\sphinxAtStartPar
Transponder Mandatory Zones TMZ are shown with a black dashed outline.
When selecting a point inside the Transponder Mandatory Zone (TMZ) by double touching the screen the information to the related ares is given with the waypoint pop\sphinxhyphen{}up window:
\begin{itemize}
\item {} 
\sphinxAtStartPar
Area Name

\item {} 
\sphinxAtStartPar
Area altitude limits

\item {} 
\sphinxAtStartPar
Monitoring Frequency

\item {} 
\sphinxAtStartPar
Mode 3 Squawk

\end{itemize}


\section{Radio Mandatory Zone}
\label{\detokenize{03-reference/airspace_display:radio-mandatory-zone}}
\sphinxAtStartPar
Radio Mandatory Zones (RMZ) are shown with a solid blue dashed outline and filled in transparent blue.
When selecting a point inside the Radio Mandatory Zone (RMZ) by double touching the screen the information to the related area is given with the waypoint pop\sphinxhyphen{}up window:
\begin{itemize}
\item {} 
\sphinxAtStartPar
Area Name

\item {} 
\sphinxAtStartPar
Area altitude limits

\item {} 
\sphinxAtStartPar
Radio Frequency

\end{itemize}


\section{Parachute Jumping Areas}
\label{\detokenize{03-reference/airspace_display:parachute-jumping-areas}}
\sphinxAtStartPar
Parachute Jumping Exercise areas (PJE) are shown with a solid red dashed outline.
When selecting a point inside the PJE by double touching the screen the information to the related area is given with the waypoint pop\sphinxhyphen{}up window:
\begin{itemize}
\item {} 
\sphinxAtStartPar
Area Name

\item {} 
\sphinxAtStartPar
Area altitude limits

\item {} 
\sphinxAtStartPar
Radio Frequency

\end{itemize}


\section{Nature Reserve Areas}
\label{\detokenize{03-reference/airspace_display:nature-reserve-areas}}
\sphinxAtStartPar
Nature Reserve Areas (NRA) are shown with a solid green outline.
When selecting a point inside the NRA by double touching the screen the information to the related area is given with the waypoint pop\sphinxhyphen{}up window:
\begin{itemize}
\item {} 
\sphinxAtStartPar
Area Name

\item {} 
\sphinxAtStartPar
Area altitude limits

\end{itemize}

\begin{sphinxadmonition}{caution}{Caution:}
\sphinxAtStartPar
Check restrictions applicable for flying inside NRA when planning your flight. For example in Austria high fines are applicable when flying inside NRA.
\begin{quote}

\sphinxAtStartPar
Figure 14:  Nature Reserve Area
\end{quote}
\end{sphinxadmonition}

\begin{figure}[htbp]
\centering

\noindent\sphinxincludegraphics{{fig_nra}.png}
\end{figure}

\sphinxAtStartPar
\sphinxstyleemphasis{Legend}:
\begin{enumerate}
\sphinxsetlistlabels{\arabic}{enumi}{enumii}{}{.}%
\item {} 
\sphinxAtStartPar
Outline of Nature Reserve Area (NRA)

\item {} 
\sphinxAtStartPar
Designation of NRA

\end{enumerate}


\section{Airfields}
\label{\detokenize{03-reference/airspace_display:airfields}}
\sphinxAtStartPar
The symbology used to display airfields follows the ICAO rules.
Airfield Information
When selecting an airfield by double touching the screen the related information is given in a pop\sphinxhyphen{}up window:
\begin{itemize}
\item {} 
\sphinxAtStartPar
Airfield Name and Identifier

\item {} 
\sphinxAtStartPar
Radio Frequency including COM and Information frequencies

\item {} 
\sphinxAtStartPar
Navaid frequencies

\item {} 
\sphinxAtStartPar
Runway orientation, dimensions and surface

\item {} 
\sphinxAtStartPar
Field elevation

\item {} 
\sphinxAtStartPar
Data for associated airspace

\end{itemize}


\section{Approach and Departure Routes}
\label{\detokenize{03-reference/airspace_display:approach-and-departure-routes}}
\sphinxAtStartPar
Approach routes to airfields are shown as solid blue lines. The designation of the route is written along the paths. The associated reporting points are shown as blue triangles with a dashed circle and the reporting point designation.
Approach Routes will be shown by a solid line and Departure Routes will be shown as  dashed lines.
Note
Approach Routes will only be displayed when zooming into the area.
Traffic Pattern
Traffic pattern for motorized aircraft are shown as blue lines.
Traffic circuits for gliders or Ultralight aircraft are shown as red lines.
Entry and exit routes to traffic pattern are indicated by open ends of the pattern.
The traffic circuit will show the traffic circuit altitude when the information is available.
Note
Traffic pattern will only be displayed when zooming into the area.


\chapter{Weather}
\label{\detokenize{03-reference/weather:weather}}\label{\detokenize{03-reference/weather::doc}}
\sphinxAtStartPar
The Weather page is opened via the Menu by touching the “Weather” entry.
The Weather page will display the station overview list for all currently available meteorological reports within 200 NM of the current position.

\begin{figure}[htbp]
\centering

\noindent\sphinxincludegraphics{{fig_Weather}.png}
\end{figure}

\sphinxAtStartPar
\sphinxstyleemphasis{Legend}:
\begin{enumerate}
\sphinxsetlistlabels{\arabic}{enumi}{enumii}{}{.}%
\item {} 
\sphinxAtStartPar
Weather Menu

\item {} 
\sphinxAtStartPar
Station data

\item {} 
\sphinxAtStartPar
Meteorological data closest to own position

\end{enumerate}

\sphinxAtStartPar
The weather data is downloaded from the National Weather Service of the United States of America.

\begin{sphinxadmonition}{note}{Note:}
\sphinxAtStartPar
When opening the Weather page the first time you will have to confirm that you agree to download data from the NWS server to use this service.
\end{sphinxadmonition}

\sphinxAtStartPar
The menu of the Waether page will allow to:
\begin{itemize}
\item {} 
\sphinxAtStartPar
Update the METAR and TAF data

\item {} 
\sphinxAtStartPar
Disallow he internet connection

\end{itemize}

\sphinxAtStartPar
The Weather overview window will provide the following information based on the METAR:
\begin{itemize}
\item {} 
\sphinxAtStartPar
ICAO identifier for Station and Airport name

\item {} 
\sphinxAtStartPar
Distance and magnetic Bearing to Airport

\item {} 
\sphinxAtStartPar
Time of METAR and summary weather state

\end{itemize}

\sphinxAtStartPar
On the lower end of the weather page the following data relevant to your current position will be displayed:
\begin{itemize}
\item {} 
\sphinxAtStartPar
QNH

\item {} 
\sphinxAtStartPar
Location and time of the report the QNH was extracted

\item {} 
\sphinxAtStartPar
Sunset during day or Sunrise during night at current location

\item {} 
\sphinxAtStartPar
Remaining time until sunset or sunrise

\end{itemize}

\sphinxAtStartPar
The information of each airport will be color coded by a system established by the US National Weather Service. The coding scheme is explained in the table below.
When touching a station line METAR and TAF (if available) will be shown in a weather detail sub\sphinxhyphen{}page

\begin{figure}[htbp]
\centering

\noindent\sphinxincludegraphics{{fig_WeatherDetail}.png}
\end{figure}

\sphinxAtStartPar
\sphinxstyleemphasis{Legend}:
\begin{enumerate}
\sphinxsetlistlabels{\arabic}{enumi}{enumii}{}{.}%
\item {} 
\sphinxAtStartPar
Station data including bering and distance

\item {} 
\sphinxAtStartPar
Current meteorological report

\item {} 
\sphinxAtStartPar
Decoded view of Current meteorological report

\item {} 
\sphinxAtStartPar
Weather forecast for station

\item {} 
\sphinxAtStartPar
Decoded view of weather forecast

\end{enumerate}

\begin{sphinxadmonition}{note}{Note:}
\sphinxAtStartPar
To view the full weather forecast you have to scroll down in most cases
\end{sphinxadmonition}

\begin{sphinxadmonition}{caution}{Caution:}
\sphinxAtStartPar
The color coding used for station weather does not match to European VFR criteria. Assessment of  meteorological flight conditions has to be done via an officially approved source of flight weather.
\end{sphinxadmonition}


\begin{savenotes}\sphinxattablestart
\centering
\begin{tabulary}{\linewidth}[t]{|T|T|T|T|T|}
\hline
\sphinxstyletheadfamily 
\sphinxAtStartPar
Category
&\sphinxstyletheadfamily 
\sphinxAtStartPar
Color
&\sphinxstyletheadfamily 
\sphinxAtStartPar
Ceiling
&\sphinxstyletheadfamily &\sphinxstyletheadfamily 
\sphinxAtStartPar
Visibility
\\
\hline
\sphinxAtStartPar
IFR
Instrument Flight Rules
&
\sphinxAtStartPar
Red
&
\sphinxAtStartPar
500 to below
1,000 feet AGL
&
\sphinxAtStartPar
and
/or
&
\sphinxAtStartPar
1 mile to
less than 3 miles
\\
\hline
\sphinxAtStartPar
MVFR
Marginal Visual Flight Rules
&
\sphinxAtStartPar
Yellow
&
\sphinxAtStartPar
1,000 to
3,000 feet AGL
&
\sphinxAtStartPar
and
/or
&
\sphinxAtStartPar
3 to 5 miles
\\
\hline
\sphinxAtStartPar
VFR
Visual Flight Rules
&
\sphinxAtStartPar
Green
&
\sphinxAtStartPar
greater than
3,000 feet AGL
&
\sphinxAtStartPar
and
/or
&
\sphinxAtStartPar
greater than
5 miles
\\
\hline
\end{tabulary}
\par
\sphinxattableend\end{savenotes}

\begin{sphinxadmonition}{note}{Note:}
\sphinxAtStartPar
By definition, IFR is ceiling less than 1,000 feet AGL.
\end{sphinxadmonition}

\begin{sphinxadmonition}{note}{Note:}
\sphinxAtStartPar
By definition, VFR is ceiling greater than or equal to 3,000 feet AGL and visibility greater than or equal to 5 miles while MVFR is a sub\sphinxhyphen{}category of VFR.
\end{sphinxadmonition}

\part{Appendix}


\chapter{Software licenses}
\label{\detokenize{04-appendix/licenses:software-licenses}}\label{\detokenize{04-appendix/licenses::doc}}

\section{License of Enroute Flight Navigation}
\label{\detokenize{04-appendix/licenses:license-of-enroute-flight-navigation}}
\sphinxAtStartPar
The program \sphinxstylestrong{Enroute Flight Navigation} is licensed under the \sphinxhref{https://www.gnu.org/licenses/gpl-3.0-standalone.html}{GNU General
Public License V3} or,
at your choice, any later version of this license.


\section{Third\sphinxhyphen{}Party software included in this program}
\label{\detokenize{04-appendix/licenses:third-party-software-included-in-this-program}}\begin{itemize}
\item {} 
\sphinxAtStartPar
This program includes several libraries from the \sphinxhref{https://qt.io}{Qt project}, licensed under the \sphinxhref{https://www.qt.io/download-open-source}{GNU General Public License V3}.

\item {} 
\sphinxAtStartPar
This program includes the library \sphinxhref{https://github.com/nitroshare/qhttpengine}{qhttpengine}, which is licensed under the
\sphinxhref{https://github.com/nitroshare/qhttpengine/blob/master/LICENSE.txt}{MIT license}.

\item {} 
\sphinxAtStartPar
This program includes the library \sphinxhref{https://openssl.org}{OpenSSL}, licensed
under the \sphinxhref{https://www.openssl.org/source/license.html}{Apache License 2.0}.

\end{itemize}


\section{Data included in this program}
\label{\detokenize{04-appendix/licenses:data-included-in-this-program}}\begin{itemize}
\item {} 
\sphinxAtStartPar
This program includes versions of the \sphinxhref{https://github.com/google/roboto}{Google Roboto Fonts}, which are licensed under the \sphinxhref{https://github.com/google/roboto/blob/master/LICENSE}{Apache
License 2.0}.

\item {} 
\sphinxAtStartPar
This program includes several \sphinxhref{https://github.com/google/material-design-icons}{Google Material Design Icons}, which are licensed under
the \sphinxhref{https://github.com/google/material-design-icons/blob/master/LICENSE}{Apache License 2.0}.

\item {} 
\sphinxAtStartPar
The style specification of the basemap is a modified version of the \sphinxhref{https://github.com/maputnik/osm-liberty}{OSM
liberty map design}, which is in
turn originally derived from OSM Bright from Mapbox Open Styles. The code is
licensed under the \sphinxhref{https://github.com/maputnik/osm-liberty/blob/gh-pages/LICENSE.md}{BSD license}. The OSM
style Bright from Mapbox Open Styles is licensed under the \sphinxhref{https://github.com/maputnik/osm-liberty/blob/gh-pages/LICENSE.md}{Creative Commons
Attribution 3.0 license}.

\end{itemize}


\section{Base maps}
\label{\detokenize{04-appendix/licenses:base-maps}}\begin{itemize}
\item {} 
\sphinxAtStartPar
The base maps are modified data from \sphinxhref{https://github.com/openmaptiles/openmaptiles}{OpenMapTiles}, published under a \sphinxhref{https://github.com/openmaptiles/openmaptiles/blob/master/LICENSE.md}{CC\sphinxhyphen{}BY 4.0
design license}.

\end{itemize}


\section{Aviation maps}
\label{\detokenize{04-appendix/licenses:aviation-maps}}\begin{itemize}
\item {} 
\sphinxAtStartPar
The aviation maps contain data from \sphinxhref{http://www.openaip.net}{openAIP},
licensed under a \sphinxhref{https://creativecommons.org/licenses/by-nc-sa/3.0/}{CC BY\sphinxhyphen{}NC\sphinxhyphen{}SA license}.

\item {} 
\sphinxAtStartPar
The aviation maps contain data from \sphinxhref{https://www.openflightmaps.org/}{open flightmaps}, licensed under the \sphinxhref{https://www.openflightmaps.org/live/downloads/20150306-LCN.pdf}{OFMA General Users´
License}.

\end{itemize}


\chapter{Technical Notes}
\label{\detokenize{04-appendix/technical:technical-notes}}\label{\detokenize{04-appendix/technical::doc}}

\section{Traffic Receiver}
\label{\detokenize{04-appendix/technical:traffic-receiver}}
\sphinxAtStartPar
\sphinxstylestrong{Enroute Flight Navigation} expects that the traffic receiver deploys a WLAN
network via Wi\sphinxhyphen{}Fi and publishes a stream of NMEA sentences at the IP address
192.168.1.1, port 2000of that network. The NMEA sentences must conform to the
specification outlined in in the document FTD\sphinxhyphen{}012 \sphinxhref{https://flarm.com/support/manuals-documents/}{Data Port Interface Control
Document (ICD)}, Version 7.13,
as published by \sphinxhref{https://flarm.com/}{FLARM Technology Ltd}.



\renewcommand{\indexname}{Index}
\printindex
\end{document}