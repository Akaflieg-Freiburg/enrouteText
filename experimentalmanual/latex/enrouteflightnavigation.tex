%% Generated by Sphinx.
\def\sphinxdocclass{report}
\documentclass[letterpaper,10pt,english]{sphinxmanual}
\ifdefined\pdfpxdimen
   \let\sphinxpxdimen\pdfpxdimen\else\newdimen\sphinxpxdimen
\fi \sphinxpxdimen=.75bp\relax

\PassOptionsToPackage{warn}{textcomp}
\usepackage[utf8]{inputenc}
\ifdefined\DeclareUnicodeCharacter
% support both utf8 and utf8x syntaxes
  \ifdefined\DeclareUnicodeCharacterAsOptional
    \def\sphinxDUC#1{\DeclareUnicodeCharacter{"#1}}
  \else
    \let\sphinxDUC\DeclareUnicodeCharacter
  \fi
  \sphinxDUC{00A0}{\nobreakspace}
  \sphinxDUC{2500}{\sphinxunichar{2500}}
  \sphinxDUC{2502}{\sphinxunichar{2502}}
  \sphinxDUC{2514}{\sphinxunichar{2514}}
  \sphinxDUC{251C}{\sphinxunichar{251C}}
  \sphinxDUC{2572}{\textbackslash}
\fi
\usepackage{cmap}
\usepackage[T1]{fontenc}
\usepackage{amsmath,amssymb,amstext}
\usepackage{babel}



\usepackage{times}
\expandafter\ifx\csname T@LGR\endcsname\relax
\else
% LGR was declared as font encoding
  \substitutefont{LGR}{\rmdefault}{cmr}
  \substitutefont{LGR}{\sfdefault}{cmss}
  \substitutefont{LGR}{\ttdefault}{cmtt}
\fi
\expandafter\ifx\csname T@X2\endcsname\relax
  \expandafter\ifx\csname T@T2A\endcsname\relax
  \else
  % T2A was declared as font encoding
    \substitutefont{T2A}{\rmdefault}{cmr}
    \substitutefont{T2A}{\sfdefault}{cmss}
    \substitutefont{T2A}{\ttdefault}{cmtt}
  \fi
\else
% X2 was declared as font encoding
  \substitutefont{X2}{\rmdefault}{cmr}
  \substitutefont{X2}{\sfdefault}{cmss}
  \substitutefont{X2}{\ttdefault}{cmtt}
\fi


\usepackage[Bjarne]{fncychap}
\usepackage{sphinx}

\fvset{fontsize=\small}
\usepackage{geometry}


% Include hyperref last.
\usepackage{hyperref}
% Fix anchor placement for figures with captions.
\usepackage{hypcap}% it must be loaded after hyperref.
% Set up styles of URL: it should be placed after hyperref.
\urlstyle{same}

\addto\captionsenglish{\renewcommand{\contentsname}{Getting started}}

\usepackage{sphinxmessages}
\setcounter{tocdepth}{1}



\title{Enroute Flight Navigation}
\date{Feb 04, 2021}
\release{2.2.4}
\author{Stefan Kebekus}
\newcommand{\sphinxlogo}{\vbox{}}
\renewcommand{\releasename}{Release}
\makeindex
\begin{document}

\pagestyle{empty}
\sphinxmaketitle
\pagestyle{plain}
\sphinxtableofcontents
\pagestyle{normal}
\phantomsection\label{\detokenize{index::doc}}


\noindent{\hspace*{\fill}\sphinxincludegraphics[width=100\sphinxpxdimen]{{de.akaflieg_freiburg.enroute}.png}\hspace*{\fill}}

Enroute Flight Navigation is a free flight navigation app for Android and other
devices. Designed to be simple, functional and elegant, it takes the stress out
of your next flight. The program has been written by flight enthusiasts, as a
project of \sphinxhref{https://akaflieg-freiburg.de/}{Akaflieg Freiburg}, a flight club
based in Freiburg, Germany.

Enroute Flight Navigation features a moving map, similar in style to the
official ICAO maps. Your current position and your flight path for the next five
minutes are marked, and so is your intended flight route. A double tap on the
display gives you all the information about airspaces, airfields and navaids \textendash{}
complete with frequencies, codes, elevations and runway information.

The free aeronautical maps can be downloaded for offline use. In addition to
airspaces, airfields and navaids, selected maps also show traffic circuits as
well as flight procedures for control zones. The maps receive near\sphinxhyphen{}weekly
updates and cover large parts of the world.

Enroute Flight Navigation includes flight weather data downloaded from the
\sphinxhref{https://www.aviationweather.gov/}{NOAA \sphinxhyphen{} Aviation Weather Center}.

While Enroute Flight Navigation is no substitute for full\sphinxhyphen{}featured flight
planning software, it allows you to quickly and easily compute distances,
courses and headings, and gives you an estimate for flight time and fuel
consumption. If the weather turns bad, the app will show you the closest
airfields for landing, complete with distances, directions, runway information
and frequencies.

\part{Getting started}


\chapter{Think before you fly}
\label{\detokenize{01-intro/think:think-before-you-fly}}\label{\detokenize{01-intro/think::doc}}
Enroute Flight Navigation is a free software product that has been published in
the hope that it might be useful as an aid to prudent navigation.  It comes with
no guarantees.  It may not work as expected.  Data shown to you might be wrong.
Your hardware may fail.

This app is no substitute for proper flight preparation or good pilotage.  Any
information \sphinxstylestrong{must always} be validated using an official navigation and
airspace data source.

\begin{sphinxadmonition}{warning}{Warning:}
Always use official flight navigation data for flight preparation
and navigate by officially authorized means. The use of non\sphinxhyphen{}certified
navigation devices and software like Enroute Flight Navigation as primary
source of navigation may cause accidents leading to loss of lives.
\end{sphinxadmonition}

We do not believe that the use of Enroute Flight Navigation fulfills the
requirement of the EU Regulation \sphinxhref{https://eur-lex.europa.eu/LexUriServ/LexUriServ.do?uri=OJ:L:2012:281:0001:0066:EN:PDF}{No 923/2012:SERA.2010}
\begin{quote}

Before beginning a flight, the pilot\sphinxhyphen{}in\sphinxhyphen{}command of an aircraft shall become
familiar with all available information appropriate to the intended operation.
\end{quote}

To put it simply: relying on Enroute Flight Navigation as a primary means of
navigation is most likely illegal in your jurisdiction.  It is most certainly
stupid and potentially suicidal.


\section{Software limitations}
\label{\detokenize{01-intro/think:software-limitations}}
Enroute Flight Navigation is not an officially approved flight navigation
software.  It is not officially approved or certified in any way.  The software
comes with no guarantee and might contain bugs.


\section{Navigational data and aviation data}
\label{\detokenize{01-intro/think:navigational-data-and-aviation-data}}
Navigational\textendash{} and aviation data, including airspace and airfield information,
are provided “as is” and without any guarantee, official validation,
certification or warranty.  The data does not come from official sources.  It
might be incomplete, outdated or otherwise incorrect.


\section{Operating system limitations}
\label{\detokenize{01-intro/think:operating-system-limitations}}
We expect that most users will run the software on mobile phones or tablet
computers running the Android operating system.  Android is not officially
approved or certified for aviation.  While we expect that the app will run fine
for the vast majority of Android users, please keep the following in mind.
\begin{itemize}
\item {} 
The Android operating system can decide at any time to terminate Enroute
Flight Navigation or to slow it down to clear resources for other apps.

\item {} 
Other apps might interfere with the operation of Enroute Flight Navigation.

\item {} 
Many hardware vendors, most notably One Plus, Huawei and Samsung equip their
phone with “battery saving apps” that randomly kill long\sphinxhyphen{}running processes.
These apps cannot be uninstalled by the users, do not comply with Android
standards and are often extremely buggy.  At times, users can manually excempt
apps from “battery saving mode”, but the settings are usually lost on system
updates.  Google’s own “Pixel” and “Nexus” devices do not have these problems.
See the web site \sphinxhref{https://dontkillmyapp.com}{Don’t kill my app} for more
information.

\end{itemize}


\section{Hardware limitations}
\label{\detokenize{01-intro/think:hardware-limitations}}
Enroute Flight Navigation runs on a variety of hardware platforms, but we expect
that most users will run the software on mobile phones, tablet computers and
comparable consumer electronic devices that are not certified to meet aviation
standards.  Keep the following in mind.
\begin{itemize}
\item {} 
Your device might not be designed to operate continuously for extended periods
of time, in particular if the display is on.

\item {} 
Your device can overheat. Batteries can catch fire.

\item {} 
Battery capacity is limited.  Even if your is connected to power via a USB
cable, note that there are devices where display and/or CPU use more energy
than USB can deliver.

\end{itemize}


\chapter{Getting started}
\label{\detokenize{01-intro/getting_started:getting-started}}\label{\detokenize{01-intro/getting_started::doc}}

\section{Installing the App}
\label{\detokenize{01-intro/getting_started:installing-the-app}}
Enroute Flight Navigation is available as an Android App in the Google Play Store.
To install the App Enroute Flight Navigation you may use the following steps:
\begin{itemize}
\item {} 
Open Google Play Store

\item {} 
Select “Apps”

\item {} 
Search for “Enroute Flight Navigation”

\item {} 
Touch the Symbol

\item {} 
Select Install

\end{itemize}


\section{Installing maps}
\label{\detokenize{01-intro/getting_started:installing-maps}}
To use Enroute Flight Navigation you have to install Maps covering the area of flight.
For installing Maps the following steps have to be followed:
\begin{itemize}
\item {} 
Open the Menu by touching the area in the upper right side of the screen

\item {} 
Open the menu item ‘Settings’

\item {} 
touch the item ‘Maps’ in the \sphinxstyleemphasis{Libraries} section

\item {} 
Select the desired Maps by clicking the download Symbol

\end{itemize}

The Map display is composed of two layers selected in the respective Tabs of the ‘Map Library’ screen:
* Aeronautical Map
* Base Map

\begin{sphinxadmonition}{caution}{Caution:}
Make sure Aeronautical and Base Map are installed for desired area of flight to avoid flight into areas without map display.
\end{sphinxadmonition}

\sphinxstylestrong{Aeronautical Maps}

The Aeronautical Map layers is showing the airspace data on the Map screen. If no Base Map is installed for the area only the information coming from the Aviation Map data is displayed.

The Aeronautical Map contains:
\begin{itemize}
\item {} 
Airfields

\item {} 
Airspace boundaries

\item {} 
Navaids

\item {} 
Reporting points and routes (if available)

\end{itemize}

The display used for aerospace data is using the following basic color scheme:
\begin{itemize}
\item {} \begin{description}
\item[{Red:}] \leavevmode\begin{itemize}
\item {} 
Line with shadow inside for Restricted Airspace

\item {} 
Shadow with dashed blue border for Aerodrome Control Zone (CTR)

\item {} 
Dashed Line for Parachute Jumping Exercise area

\item {} 
Line for Glider or Microlight Traffic pattern

\end{itemize}

\end{description}

\item {} 
Blue:
\begin{itemize}
\item {} 
Line with shadow for controlled airspace (A, B, C, D)

\item {} 
Shadow with dashed blue border for Radio Mandatory Zone (RMZ)

\item {} 
Airport, reporting point or Navaid  symbols

\item {} 
For Route or Traffic Pattern for powered aircraft

\end{itemize}

\item {} 
Green:
\begin{itemize}
\item {} 
Line with shadow for Natural Reserve Area (NRA)

\item {} 
Line for airspace control sector boundaries

\end{itemize}

\item {} 
Black:
\begin{itemize}
\item {} 
Dashed Line for Transponder Mandatory Zone (TMZ)

\end{itemize}

\end{itemize}

\sphinxstylestrong{Class 1 and Class 2 maps:}
\begin{itemize}
\item {} 
Class 1 maps are compiled from openAIP and open flightmaps data. These maps contain complete information about airspaces, airfields and navaids. In addition, the maps contain (mandatory) reporting points. Some of our tier 1 maps also show traffic circuits and flight procedures for control zones.

\item {} 
Class 2 maps are compiled from openAIP data only. They contain complete information about airspaces, airfields and navaids.

\end{itemize}

Details on the maps may be found at \textless{}\sphinxurl{https://akaflieg-freiburg.github.io/enroute/maps/}\textgreater{}
The Aeronautical Map data is selected on the “Map Library” \textendash{} “Aviation Data” page accessed via the “Settings” Menu.
To update the list of available maps the “…” option in the upper right corner of the screen may be used.
You may install or uninstall the aviation Map data for a county by the selection on the right hand side of the country list. To find a country you have to scroll up and down in the list.

\begin{sphinxadmonition}{note}{Note:}
To have optimum presentation of the Enroute Flight Navigation map display install the Aviation Map and the Base Map for all areas you intend to use Enroute Flight Navigation.
\end{sphinxadmonition}

\begin{sphinxadmonition}{caution}{Caution:}
No airspace information will be provided in country when the Aeronautical Map is not installed for it.
\end{sphinxadmonition}

\begin{sphinxadmonition}{note}{Note:}
“Enroute Flight Navigation” will automatically check for updated Maps on the Enroute server and show a pop\sphinxhyphen{}up window after start if updated maps have been detected.
You will be asked if you want to update the map or delay the update.
\end{sphinxadmonition}

\sphinxstylestrong{Base Map}

The Base Map layers is showing the geographic data on the Map screen. If no Base Map is shown for an area it will be shown in the white background color. If no Aviation Map is installed for the area only the information coming from the Base Map data is displayed. The Base Map is organized in tiles. This will result in not stopping the Base Map display abruptly at the border of an installed country, but showing some overlap.
The Base Map will show:
\begin{itemize}
\item {} 
Landmass

\item {} 
Water Surface (oceans, lakes and rivers)

\item {} 
Forests

\item {} 
Main Roads

\item {} 
Railroad lines

\item {} 
City names

\end{itemize}

\begin{sphinxadmonition}{note}{Note:}
To have optimum presentation of the Enroute Flight Navigation map display install the Aeronautical Map and the Base Map for all areas you intend to use Enroute Flight Navigation.
\end{sphinxadmonition}

\begin{sphinxadmonition}{note}{Note:}
“Enroute Flight Navigation” will not show  most cultural build ups and limits or settled area boundaries to reduce the map size.
\end{sphinxadmonition}


\section{Flight mode and ground mode}
\label{\detokenize{01-intro/getting_started:flight-mode-and-ground-mode}}
\sphinxstylestrong{Ground Mode}
Ground Mode is displayed by Enroute Flight Navigation whenever the sensed speed is below the threshold and the Menu item ‘Automatic Flight Detection’ is not set to ‘Always in Flight Mode’.
Ground Mode does not display the Flight Data line at the lower end of the screen and is intended for flight planning.

\begin{figure}[htbp]
\centering

\noindent\sphinxincludegraphics{{fig_GroundMode}.png}
\end{figure}

\sphinxstyleemphasis{Legend}:
\begin{enumerate}
\sphinxsetlistlabels{\arabic}{enumi}{enumii}{}{.}%
\item {} 
Own Position (No valid GPS position)

\item {} 
North Indicator, also area to switch between track up and north up

\item {} 
Zoom area to increase map scale (+) and reduce map scale (\sphinxhyphen{})

\item {} 
Map Scale reference indicator

\item {} 
Menu area

\end{enumerate}

There are two basic ways to plan a flight route:
\begin{itemize}
\item {} 
Menu \sphinxhyphen{} Route:
\begin{itemize}
\item {} 
Enter Waypoints

\item {} 
Edit existing Route

\item {} 
Enter Wind data

\item {} 
Enter Aircraft data

\end{itemize}

\item {} \begin{description}
\item[{Double touch Maps and open Waypoints}] \leavevmode\begin{itemize}
\item {} 
Direct will make a route from present position to Waypoint

\item {} 
‘+’ to Route will add the Waypoint to the current Route

\end{itemize}

\end{description}

\end{itemize}

A Route will remain in Enroute Flight Navigation until overwritten or removed. Routes may be saved or shared.

\sphinxstylestrong{Flight Mode}

When Enroute Flight Navigation senses a speed above the threshold it will automatically switch to flight mode.
For the displays given in flight mode refer to Figure 3: Flight Mode (Track Up)
In flight mode the following additional items will be displayed:
* The own position will be changes from a dot to an arrow
* A segmented flight path for the next 5 minutes will be indicated
* A flight data line will indicate the following GPS data:
* Altitude in feet (or meters if metric units selected)
* Ground Speed in knots (or km/h if metric units selected)
* Track in reference to true north
* Universal Coordinated Time (UTC)

\begin{figure}[htbp]
\centering

\noindent\sphinxincludegraphics{{fig_FlightModeTu}.png}
\end{figure}

\sphinxstyleemphasis{Legend}:
\begin{enumerate}
\sphinxsetlistlabels{\arabic}{enumi}{enumii}{}{.}%
\item {} 
Own Position

\item {} 
Flight Path Vector (5 Minutes)

\item {} 
North Indicator, also area to switch between track up and north up

\item {} 
Center on Position area

\item {} 
Zoom area

\item {} 
Menu area

\end{enumerate}

The  Enroute Flight Navigation map display is automatically centered to display the own position to have about 80 \% of the display area in direction  of flight.
The map display may be shifted by touching the display and moving it to the desired position. After shifting the  “Center on Position”  Symbol will be displayed. After touching he  “Center on Position”  Symbol the map will be centered to give maximum map area in direction of flight again.

\sphinxstylestrong{Track Up and North Up Mode}

The Enroute Flight Navigation map display may be switched between a Track Up display and a North Up display by touching the gray window in the upper right area.
Touching the display orientation area toggles between North up and Track Up.

\begin{figure}[htbp]
\centering

\noindent\sphinxincludegraphics{{fig_FlightModeTu}.png}
\end{figure}

\sphinxstyleemphasis{Legend}:
\begin{enumerate}
\sphinxsetlistlabels{\arabic}{enumi}{enumii}{}{.}%
\item {} 
Own Position

\item {} 
Flight Path Vector (5 Minutes)

\item {} 
North Indicator, also area to switch between track up and north up

\item {} 
Zoom area

\item {} 
Scale

\item {} 
Menu area

\end{enumerate}

The North Up mode provides a map display always showing the map north up.
The  Enroute Flight Navigation map display in North Up mode will center the display to provide about 80\% area in direction of flight.
In case the map display has been manually rotated the area besides the direction arrow will show the map orientation in degrees.


\section{Your first flight}
\label{\detokenize{01-intro/getting_started:your-first-flight}}
Now you are ready for the first use of Enroute Flight Navigation General operation is very intuitive. The primary purpose of Enroute Flight Navigation of displaying a moving aeronautical map is directly available after starting the app.
Before using the moving map function you have to make sure the GPS of your mobile device is operating properly. The own position indicator will be gray if GPS position is not available and will be displayed in blue color if GPS position is available. The own position will be indicated as round shape when no motion is sensed and displayed as arrow with flight path marker when moving.

\begin{sphinxadmonition}{warning}{Warning:}
Make sure the GPS position is correct and valid to avoid loss of situational awareness. Loss of situational awareness is a common cause of severe accidents in aviation.
\end{sphinxadmonition}

To show a planned route on the moving map display you may:
\begin{enumerate}
\sphinxsetlistlabels{\arabic}{enumi}{enumii}{}{.}%
\item {} \begin{description}
\item[{Use ‘Direct’}] \leavevmode\begin{itemize}
\item {} 
Double Touch the desired Waypoint

\item {} 
Select ‘Direct’

\end{itemize}

\end{description}

\item {} \begin{description}
\item[{Plan a route}] \leavevmode\begin{itemize}
\item {} 
Double Touch the desired Waypoint

\item {} 
Select (+) ‘to Route’

\end{itemize}

\end{description}

\end{enumerate}

The planned route will be displayed as light green line on the map display. More detailed information on route planning will be given in the dedicated section.

\sphinxstylestrong{Airspace awareness}

Information related to any selected point on the Map will be displayed when double touching a point.

The displayed Information for arbitrary points will include:
\begin{itemize}
\item {} 
Distance to point

\item {} 
True bearing to point

\item {} 
Airspace classification including related frequencies and transponder code

\end{itemize}

The displayed Information for reporting points or Navaids will include:
\begin{itemize}
\item {} 
Distance to point

\item {} 
True bearing to point

\item {} 
Designation, controlling agency and radio frequencies

\item {} 
Airspace classification including related radio frequencies and transponder code

\end{itemize}

The displayed Information for airfields will include:
\begin{itemize}
\item {} 
Distance to point

\item {} 
True bearing to point

\item {} 
Meteorological information summary if available

\item {} 
Designation, controlling agency and radio frequencies and Navaids

\item {} 
Airfield data for Runways and field elevation

\item {} 
Airspace classification including related radio frequencies and transponder code

\end{itemize}

More information on the features and operation will be given in the ‘Further Steps’ part of the Enroute Flight Navigation manual.

The following topics are described in more detail Enroute Flight Navigation ‘Reference’ section of the manual:
\begin{itemize}
\item {} 
Display of Airspace

\item {} 
Display of Aeronautical Data

\item {} 
Weather Data

\item {} 
Settings

\end{itemize}


\chapter{Further Steps}
\label{\detokenize{01-intro/further_steps:further-steps}}\label{\detokenize{01-intro/further_steps::doc}}
When using Enroute Flight Navigation on a frequent basis you may want to get into some more detail on the features and configure the app.


\section{Menu}
\label{\detokenize{01-intro/further_steps:menu}}
Functions not directly accessible on the moving map display and more options are accessed via the menu symbol in the upper left edge of the moving map display.

When touching the Menu area in the left upper corner of the screen the menu will open and give the following options:
\begin{itemize}
\item {} 
Route — see Flight Routes described below

\item {} 
Nearby Waypoints  — will show the closest 20 aerodromes, navaids or reporting points

\item {} 
Weather   — will open the weather display

\item {} 
Set Altimeter — allows to enter current altitude to have altitude with an offset

\item {} 
Settings — see below

\item {} \begin{description}
\item[{Information}] \leavevmode\begin{itemize}
\item {} 
Satellite Status

\item {} 
Manual

\item {} 
About Enroute Flight Navigation

\item {} 
Participate

\item {} 
Donate

\end{itemize}

\end{description}

\item {} 
Bug Report

\item {} 
Exit

\end{itemize}

Only some reference to Menu items is given in this section. More details may be found in the ‘Reference’ section of  the manual.


\section{Settings}
\label{\detokenize{01-intro/further_steps:settings}}
The settings Menu will allow to customize Enroute Flight Navigation and give access to program status.
The settings Menu gives the following options:
\begin{itemize}
\item {} 
\sphinxstylestrong{Hide Airspace above FL 100} — allows to select display of airspace above FL100 displayed.

\item {} 
\sphinxstylestrong{Automatic flight detection} — allows to select the display of flight mode on the ground

\item {} 
\sphinxstylestrong{Flight Routes} — will show a window with the previously stored routes

\item {} 
\sphinxstylestrong{Maps} — will show a window with the available and previously installed Aviation and Base Maps

\item {} 
\sphinxstylestrong{Use metric units} — allows to select display in metric units

\item {} 
\sphinxstylestrong{Use English} — allows to select English Language for display

\end{itemize}

\begin{sphinxadmonition}{note}{Note:}
The items to be selected by the on\sphinxhyphen{}off slider will enable the related function. The current status of the selected item is shown below the item.
\end{sphinxadmonition}

\begin{sphinxadmonition}{note}{Note:}
If you do not select “Hide Airspace above FL 100” the FIS frequencies for the Airspace C above FL100 will be displayed. In general this frequencies are also applicable below FL 100.
\end{sphinxadmonition}

\begin{sphinxadmonition}{note}{Note:}
When Automatic flight detection is not selected the display will always be in flight mode.
\end{sphinxadmonition}

\begin{sphinxadmonition}{note}{Note:}
If “Use English” is not selected the standard language selected for your device will be used if available.
\end{sphinxadmonition}


\section{Flight routes}
\label{\detokenize{01-intro/further_steps:flight-routes}}
Enroute Flight Navigation provides direct planning of one Route. A Route will remain present until it is cleared.
Route planning is entered via the Menu point Route. The Menu is entered via the Menu Symbol in the upper left corner of the map area. Then the Route Symbol has to be touched to go to the Route area.

A Route may be planned in the following ways:
\begin{itemize}
\item {} 
“Direct” in the waypoint window will provide a Route between current position and desired waypoint

\item {} 
“to Route” in the waypoint window will add the waypoint to the last position of the Route.

\item {} 
“Add Waypoint” in the Route window will open a selection window for a waypoint and add the selected waypoint to the route.

\end{itemize}

The Route Display will show the following information:
\begin{itemize}
\item {} 
Symbol of the waypoint

\item {} 
Designation of the waypoint

\item {} 
Route Point Menu

\item {} \begin{description}
\item[{Navigation Data}] \leavevmode\begin{itemize}
\item {} 
Distance between way points

\item {} 
Time calculated between way points using the cruise speed set in the “Aircraft and Wind” page

\item {} 
True Course between way points

\item {} 
True Heading between way points

\end{itemize}

\end{description}

\end{itemize}

\begin{sphinxadmonition}{note}{Note:}
A Route may also be imported from a GPX file from another PC. After sending the GPX file as Email attachment Enroute Flight Navigation will offer to open the GPX file.
\end{sphinxadmonition}

The Route Point Menu provides the option to:
* Move a waypoint up in the Route
* Move a waypoint down in the Route
* Remove a waypoint from the Route

The Route Menu is entered by touching the Route Menu Symbol on the Route page.
The following options are available from the Route Menu:
\begin{itemize}
\item {} 
Open a previously stored route from the library

\item {} 
Save the current route to the library

\item {} 
View the route library

\item {} 
Import a Route from an external source

\item {} 
Send the Route in JSON or GPX format

\item {} 
Open the Route in another APP using the JSON or GPX format

\item {} 
Clear Route

\item {} 
Reverse Route

\end{itemize}

The previously created and stored routes will be kept in a data base within Enroute Flight Navigation. Routes consist of the data for the selected way points. The Route data may be exported for use in other applications.

\sphinxstylestrong{Route \textendash{} Aircraft and Wind}

The Aircraft and Wind sub\sphinxhyphen{}page of the Route page allows to enter aircraft performance and wind data required for navigational calculations.
The Aircraft Data will be used to determine the distance of the flight and the true course.
The Wind data will will be used to calculate the true heading and duration of the flight. The duration of the flight will determine the fuel used.
Enroute Flight Navigation only offers a very superficial flight planning and cannot replace a full flight planning, but is only intended to provide quick reference.

Warning
Always perform a full flight preparation in accordance with the flight manual of the aircraft used. The use of Enroute Flight Navigation as primary flight planning may cause accidents leading to loss of lives.

The Aircraft and Wind sub\sphinxhyphen{}page of the Route page offers the following input fields:
\begin{itemize}
\item {} \begin{description}
\item[{Aircraft}] \leavevmode\begin{itemize}
\item {} 
Cruise Speed: Average Speed for Route

\item {} 
Descent Speed: Allows to enter a different speed for the descent phase (Currently not used)

\item {} 
Fuel Consumption: Average Fuel consumption per hour

\end{itemize}

\end{description}

\item {} \begin{description}
\item[{Wind}] \leavevmode\begin{itemize}
\item {} 
Direction in degrees

\item {} 
Speed in knots

\end{itemize}

\end{description}

\end{itemize}

Only one speed, fuel consumption and wind may be entered for the whole route.

\part{Reference manual}


\chapter{Main window}
\label{\detokenize{02-reference/main_window:main-window}}\label{\detokenize{02-reference/main_window::doc}}
TODO

\part{Appendix}


\chapter{Software licenses}
\label{\detokenize{03-appendix/licenses:software-licenses}}\label{\detokenize{03-appendix/licenses::doc}}
TODO



\renewcommand{\indexname}{Index}
\printindex
\end{document}