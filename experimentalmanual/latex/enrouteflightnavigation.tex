%% Generated by Sphinx.
\def\sphinxdocclass{report}
\documentclass[letterpaper,10pt,english]{sphinxmanual}
\ifdefined\pdfpxdimen
   \let\sphinxpxdimen\pdfpxdimen\else\newdimen\sphinxpxdimen
\fi \sphinxpxdimen=.75bp\relax

\PassOptionsToPackage{warn}{textcomp}
\usepackage[utf8]{inputenc}
\ifdefined\DeclareUnicodeCharacter
% support both utf8 and utf8x syntaxes
  \ifdefined\DeclareUnicodeCharacterAsOptional
    \def\sphinxDUC#1{\DeclareUnicodeCharacter{"#1}}
  \else
    \let\sphinxDUC\DeclareUnicodeCharacter
  \fi
  \sphinxDUC{00A0}{\nobreakspace}
  \sphinxDUC{2500}{\sphinxunichar{2500}}
  \sphinxDUC{2502}{\sphinxunichar{2502}}
  \sphinxDUC{2514}{\sphinxunichar{2514}}
  \sphinxDUC{251C}{\sphinxunichar{251C}}
  \sphinxDUC{2572}{\textbackslash}
\fi
\usepackage{cmap}
\usepackage[T1]{fontenc}
\usepackage{amsmath,amssymb,amstext}
\usepackage{babel}



\usepackage{times}
\expandafter\ifx\csname T@LGR\endcsname\relax
\else
% LGR was declared as font encoding
  \substitutefont{LGR}{\rmdefault}{cmr}
  \substitutefont{LGR}{\sfdefault}{cmss}
  \substitutefont{LGR}{\ttdefault}{cmtt}
\fi
\expandafter\ifx\csname T@X2\endcsname\relax
  \expandafter\ifx\csname T@T2A\endcsname\relax
  \else
  % T2A was declared as font encoding
    \substitutefont{T2A}{\rmdefault}{cmr}
    \substitutefont{T2A}{\sfdefault}{cmss}
    \substitutefont{T2A}{\ttdefault}{cmtt}
  \fi
\else
% X2 was declared as font encoding
  \substitutefont{X2}{\rmdefault}{cmr}
  \substitutefont{X2}{\sfdefault}{cmss}
  \substitutefont{X2}{\ttdefault}{cmtt}
\fi


\usepackage[Bjarne]{fncychap}
\usepackage{sphinx}

\fvset{fontsize=\small}
\usepackage{geometry}


% Include hyperref last.
\usepackage{hyperref}
% Fix anchor placement for figures with captions.
\usepackage{hypcap}% it must be loaded after hyperref.
% Set up styles of URL: it should be placed after hyperref.
\urlstyle{same}

\addto\captionsenglish{\renewcommand{\contentsname}{Getting started}}

\usepackage{sphinxmessages}
\setcounter{tocdepth}{1}



\title{Enroute Flight Navigation}
\date{Jan 10, 2021}
\release{2.2.4}
\author{Stefan Kebekus}
\newcommand{\sphinxlogo}{\vbox{}}
\renewcommand{\releasename}{Release}
\makeindex
\begin{document}

\pagestyle{empty}
\sphinxmaketitle
\pagestyle{plain}
\sphinxtableofcontents
\pagestyle{normal}
\phantomsection\label{\detokenize{index::doc}}


\noindent{\hspace*{\fill}\sphinxincludegraphics[width=100\sphinxpxdimen]{{de.akaflieg_freiburg.enroute}.png}\hspace*{\fill}}

Enroute Flight Navigation is a free flight navigation app for Android and other
devices. Designed to be simple, functional and elegant, it takes the stress out
of your next flight. The program has been written by flight enthusiasts, as a
project of \sphinxhref{https://akaflieg-freiburg.de/}{Akaflieg Freiburg}, a flight club
based in Freiburg, Germany.

Enroute Flight Navigation features a moving map, similar in style to the
official ICAO maps. Your current position and your flight path for the next five
minutes are marked, and so is your intended flight route. A double tap on the
display gives you all the information about airspaces, airfields and navaids \textendash{}
complete with frequencies, codes, elevations and runway information.

The free aeronautical maps can be downloaded for offline use. In addition to
airspaces, airfields and navaids, selected maps also show traffic circuits as
well as flight procedures for control zones. The maps receive near\sphinxhyphen{}weekly
updates and cover large parts of the world.

Enroute Flight Navigation includes flight weather data downloaded from the
\sphinxhref{https://www.aviationweather.gov/}{NOAA \sphinxhyphen{} Aviation Weather Center}.

While Enroute Flight Navigation is no substitute for full\sphinxhyphen{}featured flight
planning software, it allows you to quickly and easily compute distances,
courses and headings, and gives you an estimate for flight time and fuel
consumption. If the weather turns bad, the app will show you the closest
airfields for landing, complete with distances, directions, runway information
and frequencies.

\part{Getting started}


\chapter{Think before you fly}
\label{\detokenize{01-intro/think:think-before-you-fly}}\label{\detokenize{01-intro/think::doc}}
Enroute Flight Navigation is a free software product that has been published in
the hope that it might be useful as an aid to prudent navigation.  It comes with
no guarantees.  It may not work as expected.  Data shown to you might be wrong.
Your hardware may fail.

This app is no substitute for proper flight preparation or good pilotage.  Any
information \sphinxstylestrong{must always} be validated using an official navigation and
airspace data source.

\begin{sphinxadmonition}{warning}{Warning:}
Always use official flight navigation data for flight preparation
and navigate by officially authorized means. The use of non\sphinxhyphen{}certified
navigation devices and software like Enroute Flight Navigation as primary
source of navigation may cause accidents leading to loss of lives.
\end{sphinxadmonition}

We do not believe that the use of Enroute Flight Navigation fulfills the
requirement of the EU Regulation \sphinxhref{https://eur-lex.europa.eu/LexUriServ/LexUriServ.do?uri=OJ:L:2012:281:0001:0066:EN:PDF}{No 923/2012:SERA.2010}
\begin{quote}

Before beginning a flight, the pilot\sphinxhyphen{}in\sphinxhyphen{}command of an aircraft shall become
familiar with all available information appropriate to the intended operation.
\end{quote}

To put it simply: relying on Enroute Flight Navigation as a primary means of
navigation is most likely illegal in your jurisdiction.  It is most certainly
stupid and potentially suicidal.


\section{Software limitations}
\label{\detokenize{01-intro/think:software-limitations}}
Enroute Flight Navigation is not an officially approved flight navigation
software.  It is not officially approved or certified in any way.  The software
comes with no guarantee and might contain bugs.


\section{Navigational data and aviation data}
\label{\detokenize{01-intro/think:navigational-data-and-aviation-data}}
Navigational\textendash{} and aviation data, including airspace and airfield information,
are provided “as is” and without any guarantee, official validation,
certification or warranty.  The data does not come from official sources.  It
might be incomplete, outdated or otherwise incorrect.


\section{Operating system limitations}
\label{\detokenize{01-intro/think:operating-system-limitations}}
We expect that most users will run the software on mobile phones or tablet
computers running the Android operating system.  Android is not officially
approved or certified for aviation.  While we expect that the app will run fine
for the vast majority of Android users, please keep the following in mind.
\begin{itemize}
\item {} 
The Android operating system can decide at any time to terminate Enroute
Flight Navigation or to slow it down to clear resources for other apps.

\item {} 
Other apps might interfere with the operation of Enroute Flight Navigation.

\item {} 
Many hardware vendors, most notably One Plus, Huawei and Samsung equip their
phone with “battery saving apps” that randomly kill long\sphinxhyphen{}running processes.
These apps cannot be uninstalled by the users, do not comply with Android
standards and are often extremely buggy.  At times, users can manually excempt
apps from “battery saving mode”, but the settings are usually lost on system
updates.  Google’s own “Pixel” and “Nexus” devices do not have these problems.
See the web site \sphinxhref{https://dontkillmyapp.com}{Don’t kill my app} for more
information.

\end{itemize}


\section{Hardware limitations}
\label{\detokenize{01-intro/think:hardware-limitations}}
Enroute Flight Navigation runs on a variety of hardware platforms, but we expect
that most users will run the software on mobile phones, tablet computers and
comparable consumer electronic devices that are not certified to meet aviation
standards.  Keep the following in mind.
\begin{itemize}
\item {} 
Your device might not be designed to operate continuously for extended periods
of time, in particular if the display is on.

\item {} 
Your device can overheat. Batteries can catch fire.

\item {} 
Battery capacity is limited.  Even if your is connected to power via a USB
cable, note that there are devices where display and/or CPU use more energy
than USB can deliver.

\end{itemize}


\chapter{Getting started}
\label{\detokenize{01-intro/getting_started:getting-started}}\label{\detokenize{01-intro/getting_started::doc}}

\section{Installing the app}
\label{\detokenize{01-intro/getting_started:installing-the-app}}
TODO


\section{Installing maps}
\label{\detokenize{01-intro/getting_started:installing-maps}}
TODO


\section{Flight mode and ground mode}
\label{\detokenize{01-intro/getting_started:flight-mode-and-ground-mode}}
TODO


\section{Your first flight}
\label{\detokenize{01-intro/getting_started:your-first-flight}}
TODO


\chapter{Further Steps}
\label{\detokenize{01-intro/further_steps:further-steps}}\label{\detokenize{01-intro/further_steps::doc}}
TODO


\section{Flight routes}
\label{\detokenize{01-intro/further_steps:flight-routes}}
TODO

\part{Reference manual}


\chapter{Main window}
\label{\detokenize{02-reference/main_window:main-window}}\label{\detokenize{02-reference/main_window::doc}}
TODO

\part{Appendix}


\chapter{Software licenses}
\label{\detokenize{03-appendix/licenses:software-licenses}}\label{\detokenize{03-appendix/licenses::doc}}
TODO



\renewcommand{\indexname}{Index}
\printindex
\end{document}