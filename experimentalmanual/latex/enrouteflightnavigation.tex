%% Generated by Sphinx.
\def\sphinxdocclass{report}
\documentclass[letterpaper,10pt,english]{sphinxmanual}
\ifdefined\pdfpxdimen
   \let\sphinxpxdimen\pdfpxdimen\else\newdimen\sphinxpxdimen
\fi \sphinxpxdimen=.75bp\relax
\ifdefined\pdfimageresolution
    \pdfimageresolution= \numexpr \dimexpr1in\relax/\sphinxpxdimen\relax
\fi
%% let collapsable pdf bookmarks panel have high depth per default
\PassOptionsToPackage{bookmarksdepth=5}{hyperref}

\PassOptionsToPackage{warn}{textcomp}
\usepackage[utf8]{inputenc}
\ifdefined\DeclareUnicodeCharacter
% support both utf8 and utf8x syntaxes
  \ifdefined\DeclareUnicodeCharacterAsOptional
    \def\sphinxDUC#1{\DeclareUnicodeCharacter{"#1}}
  \else
    \let\sphinxDUC\DeclareUnicodeCharacter
  \fi
  \sphinxDUC{00A0}{\nobreakspace}
  \sphinxDUC{2500}{\sphinxunichar{2500}}
  \sphinxDUC{2502}{\sphinxunichar{2502}}
  \sphinxDUC{2514}{\sphinxunichar{2514}}
  \sphinxDUC{251C}{\sphinxunichar{251C}}
  \sphinxDUC{2572}{\textbackslash}
\fi
\usepackage{cmap}
\usepackage[T1]{fontenc}
\usepackage{amsmath,amssymb,amstext}
\usepackage{babel}



\usepackage{tgtermes}
\usepackage{tgheros}
\renewcommand{\ttdefault}{txtt}



\usepackage[Bjornstrup]{fncychap}
\usepackage{sphinx}

\fvset{fontsize=auto}
\usepackage{geometry}


% Include hyperref last.
\usepackage{hyperref}
% Fix anchor placement for figures with captions.
\usepackage{hypcap}% it must be loaded after hyperref.
% Set up styles of URL: it should be placed after hyperref.
\urlstyle{same}

\addto\captionsenglish{\renewcommand{\contentsname}{Getting started}}

\usepackage{sphinxmessages}
\setcounter{tocdepth}{1}

\input{../../latexPreamble.tex.txt}

\title{Enroute Flight Navigation}
\date{May 16, 2021}
\release{2.2.4}
\author{Stefan Kebekus}
\newcommand{\sphinxlogo}{\vbox{}}
\renewcommand{\releasename}{Release}
\makeindex
\begin{document}

\pagestyle{empty}
\sphinxmaketitle
\pagestyle{plain}
\sphinxtableofcontents
\pagestyle{normal}
\phantomsection\label{\detokenize{index::doc}}


\noindent{\hspace*{\fill}\sphinxincludegraphics[width=100\sphinxpxdimen]{{de.akaflieg_freiburg.enroute}.png}\hspace*{\fill}}

\sphinxAtStartPar
\sphinxstylestrong{Enroute Flight Navigation} is a free flight navigation app for Android and
other devices.  Designed to be simple, functional and elegant, it takes the
stress out of your next flight. The program has been written by flight
enthusiasts, as a project of \sphinxhref{https://akaflieg-freiburg.de/}{Akaflieg Freiburg}%
\begin{footnote}[1]\sphinxAtStartFootnote
\sphinxnolinkurl{https://akaflieg-freiburg.de/}
%
\end{footnote}, a flight club based in Freiburg, Germany.

\sphinxAtStartPar
\sphinxstylestrong{Enroute Flight Navigation} features a moving map, similar in style to the
official ICAO maps. Your current position and your flight path for the next five
minutes are marked, and so is your intended flight route. A double tap on the
display gives you all the information about airspaces, airfields and navaids \textendash{}
complete with frequencies, codes, elevations and runway information.

\sphinxAtStartPar
The free aeronautical maps can be downloaded for offline use. In addition to
airspaces, airfields and navaids, selected maps also show traffic circuits as
well as flight procedures for control zones. The maps receive near\sphinxhyphen{}weekly
updates and cover large parts of the world.

\sphinxAtStartPar
\sphinxstylestrong{Enroute Flight Navigation} includes flight weather data downloaded from the
\sphinxhref{https://www.aviationweather.gov/}{NOAA \sphinxhyphen{} Aviation Weather Center}%
\begin{footnote}[2]\sphinxAtStartFootnote
\sphinxnolinkurl{https://www.aviationweather.gov/}
%
\end{footnote}.

\sphinxAtStartPar
While \sphinxstylestrong{Enroute Flight Navigation} is no substitute for full\sphinxhyphen{}featured flight
planning software, it allows you to quickly and easily compute distances,
courses and headings, and gives you an estimate for flight time and fuel
consumption. If the weather turns bad, the app will show you the closest
airfields for landing, complete with distances, directions, runway information
and frequencies.


\chapter{Think before you fly}
\label{\detokenize{01-intro/01-think:think-before-you-fly}}\label{\detokenize{01-intro/01-think::doc}}
\sphinxAtStartPar
\sphinxstylestrong{Enroute Flight Navigation} is a free software product that has been published
in the hope that it might be useful as an aid to prudent navigation.  It comes
with no guarantees.  It may not work as expected.  Data shown to you might be
wrong.  Your hardware may fail.

\sphinxAtStartPar
This app is no substitute for proper flight preparation or good pilotage.  Any
information \sphinxstylestrong{must always} be validated using an official navigation and
airspace data source.

\begin{sphinxadmonition}{warning}{Warning:}
\sphinxAtStartPar
Always use official flight navigation data for flight preparation
and navigate by officially authorized means. The use of non\sphinxhyphen{}certified
navigation devices and software like \sphinxstylestrong{Enroute Flight Navigation} as a
primary source of navigation may cause accidents leading to loss of lives.
\end{sphinxadmonition}

\sphinxAtStartPar
We do not believe that the use of \sphinxstylestrong{Enroute Flight Navigation} fulfills the
requirement of the EU Regulation \sphinxhref{https://eur-lex.europa.eu/LexUriServ/LexUriServ.do?uri=OJ:L:2012:281:0001:0066:EN:PDF}{No 923/2012:SERA.2010}%
\begin{footnote}[3]\sphinxAtStartFootnote
\sphinxnolinkurl{https://eur-lex.europa.eu/LexUriServ/LexUriServ.do?uri=OJ:L:2012:281:0001:0066:EN:PDF}
%
\end{footnote}
\begin{quote}

\sphinxAtStartPar
Before beginning a flight, the pilot\sphinxhyphen{}in\sphinxhyphen{}command of an aircraft shall become
familiar with all available information appropriate to the intended operation.
\end{quote}

\sphinxAtStartPar
To put it simply: relying on \sphinxstylestrong{Enroute Flight Navigation} as a primary means of
navigation is most likely illegal in your jurisdiction.  It is most certainly
stupid and potentially suicidal.


\section{Software limitations}
\label{\detokenize{01-intro/01-think:software-limitations}}
\sphinxAtStartPar
\sphinxstylestrong{Enroute Flight Navigation} is not an officially approved flight navigation
software.  It is not officially approved or certified in any way.  The software
comes with no guarantee and might contain bugs.


\section{Navigational data and aviation data}
\label{\detokenize{01-intro/01-think:navigational-data-and-aviation-data}}
\sphinxAtStartPar
Navigational\textendash{} and aviation data, including airspace and airfield information,
are provided “as is” and without any guarantee, official validation,
certification or warranty.  The data does not come from official sources.  It
might be incomplete, outdated or otherwise incorrect.


\section{Operating system limitations}
\label{\detokenize{01-intro/01-think:operating-system-limitations}}
\sphinxAtStartPar
We expect that most users will run the software on mobile phones or tablet
computers running the Android operating system.  Android is not officially
approved or certified for aviation.  While we expect that the app will run fine
for the vast majority of Android users, please keep the following in mind.
\begin{itemize}
\item {} 
\sphinxAtStartPar
The Android operating system can decide at any time to terminate \sphinxstylestrong{Enroute
Flight Navigation} or to slow it down to clear resources for other apps.

\item {} 
\sphinxAtStartPar
Other apps might interfere with the operation of \sphinxstylestrong{Enroute Flight
Navigation}.

\item {} 
\sphinxAtStartPar
Many hardware vendors, most notably One Plus, Huawei and Samsung equip their
phone with “battery saving apps” that randomly kill long\sphinxhyphen{}running processes.
These apps cannot be uninstalled by the users, do not comply with Android
standards and are often extremely buggy.  At times, users can manually excempt
apps from “battery saving mode”, but the settings are usually lost on system
updates.  Google’s own “Pixel” and “Nexus” devices do not have these problems.
See the website \sphinxhref{https://dontkillmyapp.com}{Don’t kill my app}%
\begin{footnote}[4]\sphinxAtStartFootnote
\sphinxnolinkurl{https://dontkillmyapp.com}
%
\end{footnote} for more
information.

\end{itemize}


\section{Hardware limitations}
\label{\detokenize{01-intro/01-think:hardware-limitations}}
\sphinxAtStartPar
\sphinxstylestrong{Enroute Flight Navigation} runs on a variety of hardware platforms, but we
expect that most users will run the software on mobile phones, tablet computers
and comparable consumer electronic devices that are not certified to meet
aviation standards.  Keep the following in mind.
\begin{itemize}
\item {} 
\sphinxAtStartPar
Your device might not be designed to operate continuously for extended periods
of time, in particular if the display is on.

\item {} 
\sphinxAtStartPar
Your device can overheat. Batteries can catch fire.

\item {} 
\sphinxAtStartPar
Battery capacity is limited.  Even if your device is connected to power via a
USB cable, the display and/or CPU might use more energy than USB can deliver.

\end{itemize}


\chapter{Installation and setup}
\label{\detokenize{01-intro/02-installation:installation-and-setup}}\label{\detokenize{01-intro/02-installation::doc}}

\section{App installation}
\label{\detokenize{01-intro/02-installation:app-installation}}\begin{description}
\item[{Installation on Android devices}] \leavevmode
\sphinxAtStartPar
\sphinxstylestrong{Enroute Flight Navigation} is available as an Android App in the \sphinxhref{https://play.google.com/store/apps/details?id=de.akaflieg\_freiburg.enroute}{Google
Play Store}%
\begin{footnote}[5]\sphinxAtStartFootnote
\sphinxnolinkurl{https://play.google.com/store/apps/details?id=de.akaflieg\_freiburg.enroute}
%
\end{footnote}.

\sphinxAtStartPar
An unofficial version of the app is also available at \sphinxhref{https://f-droid.org/de/packages/de.akaflieg\_freiburg.enroute/}{F\sphinxhyphen{}Droid}%
\begin{footnote}[6]\sphinxAtStartFootnote
\sphinxnolinkurl{https://f-droid.org/de/packages/de.akaflieg\_freiburg.enroute/}
%
\end{footnote}.  While the
author of \sphinxstylestrong{Enroute Flight Navigation} endorses publication at F\sphinxhyphen{}Droid, he
has not tested this unofficial app for quality.

\item[{Installation on Linux desktop machines}] \leavevmode
\sphinxAtStartPar
\sphinxstylestrong{Enroute Flight Navigation} is available for free download at \sphinxhref{https://flathub.org/apps/details/de.akaflieg\_freiburg.enroute}{flathub.org}%
\begin{footnote}[7]\sphinxAtStartFootnote
\sphinxnolinkurl{https://flathub.org/apps/details/de.akaflieg\_freiburg.enroute}
%
\end{footnote} and
\sphinxhref{https://snapcraft.io/enroute-flight-navigation}{snapcraft.io}%
\begin{footnote}[8]\sphinxAtStartFootnote
\sphinxnolinkurl{https://snapcraft.io/enroute-flight-navigation}
%
\end{footnote}.  Most likely
you will also find the app in the software management application on your
computer.

\end{description}

\sphinxAtStartPar
After installation, start the app.  Depending on the platform, you might need to
grant the necessary permissions.  You will be asked to accept the terms and
conditions.


\section{Map download}
\label{\detokenize{01-intro/02-installation:map-download}}
\sphinxAtStartPar
\sphinxstylestrong{Enroute Flight Navigation} cannot be used without geographic maps.  Two types
of maps need to be installed for every region where you fly.
\begin{itemize}
\item {} 
\sphinxAtStartPar
Aeronautical maps.  These contain airspaces, airfields and navaids.  Some maps
also contain reporting points, airfield traffic circuits and control zone
entry/exit routes.

\item {} 
\sphinxAtStartPar
Base maps.  These contain geographic data, such as rivers, roads, railroads
and land use.

\end{itemize}

\sphinxAtStartPar
Follow these steps to install the maps that you need.
\begin{itemize}
\item {} 
\sphinxAtStartPar
Open the Menu by touching the menu button in the upper right side of the
screen.  The button is marked with the symbol ‘☰’.

\item {} 
\sphinxAtStartPar
Choose the menu item \sphinxstyleemphasis{Library}, then \sphinxstyleemphasis{Maps}.  The map management page will
then open.

\item {} 
\sphinxAtStartPar
On the map management page, click or tap on the desired maps.  The maps will
be downloaded and installed on your device.

\end{itemize}

\sphinxAtStartPar
Please download only those maps that you will actually need.  The infrastructure
and bandwidth for map downloads is kindly sponsored by the University of
Freiburg, under the assumption that the cost stays within reasonable limits.
You will also find that the app performs much better if it does not have to
process many megabytes of map data.

\begin{sphinxadmonition}{note}{Note:}
\sphinxAtStartPar
Do not forget that you need aeronautical maps \sphinxstylestrong{and} base maps for
the desired area of flight.  The base maps are large.  Make sure that you
have a good internet connection before you download maps.  It might be
inadvisable to download base maps via the mobile phone network.
\end{sphinxadmonition}


\section{Done.}
\label{\detokenize{01-intro/02-installation:done}}
\sphinxAtStartPar
Once the map download has finished, \sphinxstylestrong{Enroute Flight Navigation} will process
the map data and update the map display after a minute or so.  Tap or click on
the arrow symbol ‘←’ or use the Android ‘Back’ button to leave the map page and
return to the main screen.

\sphinxAtStartPar
You are now ready to go.  There are many things that you could set up at this
stage, but we recommend that you simply look around any play with the app.
Continue with the next section and take it for your first flight.


\chapter{Before your first flight}
\label{\detokenize{01-intro/03-firstFlight:before-your-first-flight}}\label{\detokenize{01-intro/03-firstFlight::doc}}
\sphinxAtStartPar
Now you are ready for the first use of \sphinxstylestrong{Enroute Flight Navigation}.  General
operation is very intuitive.  Still, we recommend that you take a minute to make
yourself familiar with the moving map display and with the basic controls before
you take the app on its first flight.


\section{The moving map}
\label{\detokenize{01-intro/03-firstFlight:the-moving-map}}
\sphinxAtStartPar
After startup, the app will show a moving map, similar in style to the standard
ICAO maps that most pilots are used to.  You can use the standard gestures to
zoom and pan the map to your liking.  The figures {\hyperref[\detokenize{01-intro/03-firstFlight:movingmapground}]{\sphinxcrossref{\DUrole{std,std-ref}{Moving map display on the ground}}}} and
{\hyperref[\detokenize{01-intro/03-firstFlight:movingmapflight}]{\sphinxcrossref{\DUrole{std,std-ref}{Moving map display in flight}}}} shows how the map will typically look.

\begin{figure}[htbp]
\centering
\capstart

\noindent\sphinxincludegraphics[scale=0.3]{{fig_ground}.png}
\caption{Moving map display on the ground}\label{\detokenize{01-intro/03-firstFlight:id1}}\label{\detokenize{01-intro/03-firstFlight:movingmapground}}\end{figure}

\begin{figure}[htbp]
\centering
\capstart

\noindent\sphinxincludegraphics[scale=0.3]{{fig_flight}.png}
\caption{Moving map display in flight}\label{\detokenize{01-intro/03-firstFlight:id2}}\label{\detokenize{01-intro/03-firstFlight:movingmapflight}}\end{figure}

\sphinxAtStartPar
Initially, your own position is shown as a blue circle (or gray if the system
has not yet acquired a valid position).  Once you are moving, your own position
is shown as a blue arrow shape.  The flight path vector shows the projected
track for the next five minutes.

\sphinxAtStartPar
The bottom of the display shows a little panel with the following information.


\begin{savenotes}\sphinxattablestart
\centering
\begin{tabulary}{\linewidth}[t]{|T|T|}
\hline

\sphinxAtStartPar
T.TALT
&
\sphinxAtStartPar
True altitude (=geometric altitude) above sea level.
\\
\hline
\sphinxAtStartPar
FL
&
\sphinxAtStartPar
Flight level.
\\
\hline
\sphinxAtStartPar
GS
&
\sphinxAtStartPar
Ground speed.
\\
\hline
\sphinxAtStartPar
TT
&
\sphinxAtStartPar
True track.
\\
\hline
\sphinxAtStartPar
UTC
&
\sphinxAtStartPar
Current time.
\\
\hline
\end{tabulary}
\par
\sphinxattableend\end{savenotes}

\sphinxAtStartPar
The flight level is only available if your device is connected to a traffic
receiver (such as a PowerFLARM device) that reports the pressure altitude.
Flight level and current time are hidden if the display is not wide enough.

\begin{sphinxadmonition}{warning}{Warning:}
\sphinxAtStartPar
Vertical airspace boundaries are defined by pressure altitudes
(with respect to QNH or standard pressure).  Depending on temperature and air
density, the pressure altitude will differ from the true altitude that is
shown by the app.  \sphinxstylestrong{Never use true altitude to judge vertical distances to
airspaces.}
\end{sphinxadmonition}


\section{Interactive controls}
\label{\detokenize{01-intro/03-firstFlight:interactive-controls}}
\sphinxAtStartPar
In addition to the pan and pinch gestures, you can use the following buttons to
control the app.


\begin{savenotes}\sphinxattablestart
\centering
\begin{tabulary}{\linewidth}[t]{|T|T|}
\hline

\noindent\sphinxincludegraphics{{ic_menu}.png}
&
\sphinxAtStartPar
Open main menu
\\
\hline
\noindent\sphinxincludegraphics{{NorthArrow}.png}
&
\sphinxAtStartPar
Switch between display modes \sphinxstylestrong{north up} and \sphinxstylestrong{track up}.
\\
\hline
\noindent\sphinxincludegraphics{{ic_my_location}.png}
&
\sphinxAtStartPar
Center map about own position.
\\
\hline
\noindent\sphinxincludegraphics{{ic_add}.png}
&
\sphinxAtStartPar
Zoom in
\\
\hline
\noindent\sphinxincludegraphics{{ic_remove}.png}
&
\sphinxAtStartPar
Zoom out
\\
\hline
\end{tabulary}
\par
\sphinxattableend\end{savenotes}


\section{Information about airspaces, airfields and other facilities}
\label{\detokenize{01-intro/03-firstFlight:information-about-airspaces-airfields-and-other-facilities}}
\sphinxAtStartPar
Double tap or tap\sphinxhyphen{}and\sphinxhyphen{}hold anywhere in the map to obtain information about the
airspace situation at that point.  If you double tap or tap\sphinxhyphen{}and\sphinxhyphen{}hold on an
airfield, navaid or reporting point, detailed information about the facility
will be shown.  The figure {\hyperref[\detokenize{01-intro/03-firstFlight:stuttgartinfo}]{\sphinxcrossref{\DUrole{std,std-ref}{Information about Stuttgart airport}}}} shows how this will typically
look.

\begin{figure}[htbp]
\centering
\capstart

\noindent\sphinxincludegraphics[scale=0.3]{{fig_wpInfo}.png}
\caption{Information about Stuttgart airport}\label{\detokenize{01-intro/03-firstFlight:id3}}\label{\detokenize{01-intro/03-firstFlight:stuttgartinfo}}\end{figure}


\section{Go flying!}
\label{\detokenize{01-intro/03-firstFlight:go-flying}}
\sphinxAtStartPar
\sphinxstylestrong{Enroute Flight Navigation} is designed to be simple.  We think that you are
now ready to take \sphinxstylestrong{Enroute Flight Navigation} on its first flight.  There are
of course many more things that you can do.  Play with the app and have a look
at the next section {\hyperref[\detokenize{index:sec-steps}]{\sphinxcrossref{\DUrole{std,std-ref}{Further Steps}}}}.


\chapter{Connect your traffic receiver}
\label{\detokenize{02-steps/traffic:connect-your-traffic-receiver}}\label{\detokenize{02-steps/traffic::doc}}
\sphinxAtStartPar
In order to display nearby traffic on the moving map, \sphinxstylestrong{Enroute Flight
Navigation} can connect to your aircraft’s traffic receiver (typically a FLARM
device).  In order to show only relevant information, \sphinxstylestrong{Enroute Flight
Navigation} will not display traffic more than 1,500 m above or below the own
position.

\sphinxAtStartPar
The app author has tested the \sphinxstylestrong{Enroute Flight Navigation} with the following
traffic receivers.
\begin{itemize}
\item {} 
\sphinxAtStartPar
\sphinxhref{http://www.air-avionics.com/?page\_id=253}{AT\sphinxhyphen{}1 AIR Traffic}%
\begin{footnote}[9]\sphinxAtStartFootnote
\sphinxnolinkurl{http://www.air-avionics.com/?page\_id=253}
%
\end{footnote} by \sphinxhref{http://www.air-avionics.com/}{Air
Avionics}%
\begin{footnote}[10]\sphinxAtStartFootnote
\sphinxnolinkurl{http://www.air-avionics.com/}
%
\end{footnote} with software version 5.

\end{itemize}

\sphinxAtStartPar
Users reported success with the following traffic receivers.
\begin{itemize}
\item {} 
\sphinxAtStartPar
\sphinxhref{http://stratux.me/}{Stratux devices}%
\begin{footnote}[11]\sphinxAtStartFootnote
\sphinxnolinkurl{http://stratux.me/}
%
\end{footnote}

\item {} 
\sphinxAtStartPar
\sphinxhref{https://www.amazon.de/TTGO-T-Beam-915Mhz-Wireless-Bluetooth/dp/B07SFVQ3Z8}{TTGO T\sphinxhyphen{}Beam devices}%
\begin{footnote}[12]\sphinxAtStartFootnote
\sphinxnolinkurl{https://www.amazon.de/TTGO-T-Beam-915Mhz-Wireless-Bluetooth/dp/B07SFVQ3Z8}
%
\end{footnote}

\end{itemize}

\begin{sphinxadmonition}{note}{Note:}
\sphinxAtStartPar
To show only relevant traffic, \sphinxstylestrong{Enroute Flight Navigation} will
display traffic factors only if the vertical distance is less than 1.500m
and and the horizontal distance less than 20km.
\end{sphinxadmonition}


\section{Before you connect}
\label{\detokenize{02-steps/traffic:before-you-connect}}
\sphinxAtStartPar
Before you try to connect this app to your traffic receiver, make sure that the
following conditions are met.
\begin{itemize}
\item {} 
\sphinxAtStartPar
Your traffic receiver has an integrated Wi\sphinxhyphen{}Fi interface that acts as a
wireless access point. Bluetooth devices are currently not supported.

\item {} 
\sphinxAtStartPar
You know the network name (=SSID) of the Wi\sphinxhyphen{}Fi network deployed by your
traffic receiver. If the network is encrypted, you also need to know the Wi\sphinxhyphen{}Fi
password.

\item {} 
\sphinxAtStartPar
Some devices require an additional password in order to access traffic
data. This is currently \sphinxstylestrong{not} supported. Set up your device so that no
additional password is required.

\end{itemize}


\section{Connect to the traffic receiver}
\label{\detokenize{02-steps/traffic:connect-to-the-traffic-receiver}}
\sphinxAtStartPar
It takes two steps to connect \sphinxstylestrong{Enroute Flight Navigation} to the traffic
receiver for the first time. Once things are set up properly, your device should
automatically detect the traffic receiver’s Wi\sphinxhyphen{}Fi network, enter the network and
connect to the traffic data stream whenever you go flying.


\subsection{Step 1: Enter the traffic receiver’s Wi\sphinxhyphen{}Fi network}
\label{\detokenize{02-steps/traffic:step-1-enter-the-traffic-receiver-s-wi-fi-network}}\begin{itemize}
\item {} 
\sphinxAtStartPar
Make sure that the traffic receiver has power and is switched on. In a typical
aircraft installation, the traffic receiver is connected to the ‘Avionics’
switch and will automatically switch on. You may need to wait a minute before
the Wi\sphinxhyphen{}Fi comes online and is visible to your device.

\item {} 
\sphinxAtStartPar
Enter the Wi\sphinxhyphen{}Fi network deployed by your traffic receiver. This is usually
done in the “Wi\sphinxhyphen{}Fi Settings” of your device. Enter the Wi\sphinxhyphen{}Fi password if
required. Some devices will issue a warning that the Wi\sphinxhyphen{}Fi is not connected to
the internet. In this case, you might need to confirm that you wish to enter
the Wi\sphinxhyphen{}Fi network.

\end{itemize}

\sphinxAtStartPar
Most operating systems will offer to remember the connection, so that your
device will automatically connect to this Wi\sphinxhyphen{}Fi in the future. We recommend
using this option.


\subsection{Step 2: Connect to the traffic data stream}
\label{\detokenize{02-steps/traffic:step-2-connect-to-the-traffic-data-stream}}
\sphinxAtStartPar
Open the main menu and navigate to the “Information” menu.
\begin{itemize}
\item {} 
\sphinxAtStartPar
If the entry “Traffic Receiver” is highlighted in green, then \sphinxstylestrong{Enroute Flight
Navigation} has already found the traffic receiver in the network and has
connected to it. Congratulations, you are done!

\item {} 
\sphinxAtStartPar
If the entry “Traffic Receiver” is not highlighted in green, then select the
entry. The “Traffic Receiver Status” page will open. The page explains the
connection status in detail, and explains how to establish a connection
manually.

\end{itemize}


\section{Troubleshooting}
\label{\detokenize{02-steps/traffic:troubleshooting}}
\sphinxAtStartPar
\sphinxstylestrong{The app cannot connect to the traffic data stream.}
\begin{itemize}
\item {} 
\sphinxAtStartPar
Check that your device is connected to the Wi\sphinxhyphen{}Fi network deployed by your
traffic receiver.

\end{itemize}

\sphinxAtStartPar
\sphinxstylestrong{The connection breaks down after a few seconds.}

\sphinxAtStartPar
Most traffic receivers cannot serve more than one client and abort connections
at random if more than one device tries to access.
\begin{itemize}
\item {} 
\sphinxAtStartPar
Make sure that there no second device connected to the traffic receiver’s
Wi\sphinxhyphen{}Fi network. The other device might well be in your friend’s pocket!

\item {} 
\sphinxAtStartPar
Make sure that there is no other app trying to connect to the traffic
receiver’s data stream.

\item {} 
\sphinxAtStartPar
Many traffic receivers offer “configuration panels” that can be accessed via a
web browser. Close all web browsers.

\end{itemize}


\chapter{Connect your flight simulator}
\label{\detokenize{02-steps/simulator:connect-your-flight-simulator}}\label{\detokenize{02-steps/simulator::doc}}
\sphinxAtStartPar
\sphinxstylestrong{Enroute Flight Navigation} can connect to flight simulator software.  The app
has been tested with the following programs.
\begin{itemize}
\item {} 
\sphinxAtStartPar
\sphinxhref{https://www.x-plane.com/}{X\sphinxhyphen{}Plane 11}%
\begin{footnote}[13]\sphinxAtStartFootnote
\sphinxnolinkurl{https://www.x-plane.com/}
%
\end{footnote}

\end{itemize}

\sphinxAtStartPar
Users have reported success with the following programs.
\begin{itemize}
\item {} 
\sphinxAtStartPar
\sphinxhref{https://www.x-plane.com/}{X\sphinxhyphen{}Plane 10}%
\begin{footnote}[14]\sphinxAtStartFootnote
\sphinxnolinkurl{https://www.x-plane.com/}
%
\end{footnote}

\end{itemize}

\sphinxAtStartPar
Please contact us if you are aware of other programs that also work.

\begin{sphinxadmonition}{note}{Note:}
\sphinxAtStartPar
\sphinxstylestrong{Enroute Flight Navigation} treats flight simulators as traffic
receivers.  To see the connection status, open the main menu and navigate to
the “Information” menu.
\end{sphinxadmonition}


\section{Before you connect}
\label{\detokenize{02-steps/simulator:before-you-connect}}
\sphinxAtStartPar
This manual assumes a typical home setup, where both the computer that runs the
flight simulator and the device that runs \sphinxstylestrong{Enroute Flight Navigation} are
connected to a Wi\sphinxhyphen{}Fi network deployed by a home router.  Make sure that the
following conditions are met.
\begin{itemize}
\item {} 
\sphinxAtStartPar
The computer that runs the flight simulator and the device that runs \sphinxstylestrong{Enroute
Flight Navigation} are connected to the same Wi\sphinxhyphen{}Fi network.  Some routers
deploy two networks, often called “main network” and a “guest network”.

\item {} 
\sphinxAtStartPar
Make sure that the router allows data transfer between the devices in the
Wi\sphinxhyphen{}Fi network.  Some routers have “security settings” that disallow data
transfer between the devices in the “guest network”

\end{itemize}


\section{Set up your flight simulator}
\label{\detokenize{02-steps/simulator:set-up-your-flight-simulator}}
\sphinxAtStartPar
Your flight simulation software needs to broadcast position and traffic
information over the Wi\sphinxhyphen{}Fi network.  Once this is done, there is no further
setup required.  As soon as the flight simulator starts to broadcast information
over the Wi\sphinxhyphen{}Fi network, the moving map of \sphinxstylestrong{Enroute Flight Navigation} will
adjust accordingly.  To end the connection to the flight simulator, simply leave
the flight simulator’s Wi\sphinxhyphen{}Fi network.


\subsection{X\sphinxhyphen{}Plane 10}
\label{\detokenize{02-steps/simulator:id1}}
\sphinxAtStartPar
Follow the explanation on \sphinxhref{https://www.x-plane.com/2012/08/foreflight-charts-supported-in-x-plane-10-10-beta-9/}{this page}%
\begin{footnote}[15]\sphinxAtStartFootnote
\sphinxnolinkurl{https://www.x-plane.com/2012/08/foreflight-charts-supported-in-x-plane-10-10-beta-9/}
%
\end{footnote},
which explains how to connect X\sphinxhyphen{}Plane 10 to the commercial app ForeFlight.  In
short: Open the “Settings” window and click “Internet Options”. There, go to the
“iPhone/iPod” tab and turn on the “ForeFlight” option.  Please be sure to
disable output of data on tab “Data”.

\noindent\sphinxincludegraphics{{X-Plane-10}.png}


\subsection{X\sphinxhyphen{}Plane 11}
\label{\detokenize{02-steps/simulator:id2}}
\sphinxAtStartPar
Open the “Settings” window and choose the “Network” tab.  Locate the settings
group “This machine’s role” on the right\sphinxhyphen{}hand side of the tab. Open the section
“iPHONE, iPAD, and EXTERNAL APPS” and select the item “Broadcast to all mapping
apps on the network” under the headline “OTHER MAPPING APPS”.

\noindent\sphinxincludegraphics{{X-Plane-11}.png}


\subsection{MS Flight Simulator}
\label{\detokenize{02-steps/simulator:ms-flight-simulator}}
\sphinxAtStartPar
UNKNOWN AS OF NOW.


\subsection{Other programs}
\label{\detokenize{02-steps/simulator:other-programs}}
\sphinxAtStartPar
The flight simulator needs to be set up to send UDP datagrams in one of the
standard formats “GDL90” or “XGPS” to ports 4000 or 49002.  Given the choice,
GDL90 is generally the preferred format.


\section{Troubleshooting}
\label{\detokenize{02-steps/simulator:troubleshooting}}
\sphinxAtStartPar
\sphinxstylestrong{Enroute Flight Navigation} treats flight simulators as traffic receivers.  To
see the connection status, open the main menu and navigate to the “Information”
menu.  If the entry “Traffic Receiver” is highlighted in green, then \sphinxstylestrong{Enroute
Flight Navigation} has already found the program in the network and has
connected to it.  If not, then select the entry. The “Traffic Receiver Status”
page will open, which explains the connection status in more detail.


\chapter{Make a donation}
\label{\detokenize{02-steps/donate:make-a-donation}}\label{\detokenize{02-steps/donate::doc}}
\sphinxAtStartPar
\sphinxstylestrong{Enroute Flight Navigation} is a non\sphinxhyphen{}commercial project of \sphinxhref{https://akaflieg-freiburg.de/}{Akaflieg Freiburg}%
\begin{footnote}[16]\sphinxAtStartFootnote
\sphinxnolinkurl{https://akaflieg-freiburg.de/}
%
\end{footnote} and the \sphinxhref{https://uni-freiburg.de/en/}{University of Freiburg}%
\begin{footnote}[17]\sphinxAtStartFootnote
\sphinxnolinkurl{https://uni-freiburg.de/en/}
%
\end{footnote}. The app has been written by flight enthusiasts
in their spare time, as a service to the community. The developers do not take
donations.

\sphinxAtStartPar
If you appreciate the app, please consider a donation to Akaflieg Freiburg, a
tax\sphinxhyphen{}privileged, not\sphinxhyphen{}for\sphinxhyphen{}profit flight club of public utility in Freiburg,
Germany.

\begin{sphinxVerbatim}[commandchars=\\\{\}]
\PYG{n}{IBAN}\PYG{p}{:}    \PYG{n}{DE35} \PYG{l+m+mi}{6809} \PYG{l+m+mi}{0000} \PYG{l+m+mi}{0027} \PYG{l+m+mi}{6409} \PYG{l+m+mi}{07}
\PYG{n}{BIC}\PYG{p}{:}     \PYG{n}{GENODE61FR1}
\PYG{n}{Bank}\PYG{p}{:}    \PYG{n}{Volksbank} \PYG{n}{Freiburg}
\PYG{n}{Message}\PYG{p}{:} \PYG{n}{Enroute} \PYG{n}{Flight} \PYG{n}{Navigation}
\end{sphinxVerbatim}


\chapter{Map Data}
\label{\detokenize{03-reference/map_data:map-data}}\label{\detokenize{03-reference/map_data::doc}}
\sphinxAtStartPar
The Information displayed by the Map of Enroute Flight Navigation is provided by the following resources:
\begin{itemize}
\item {} 
\sphinxAtStartPar
openAIP

\item {} 
\sphinxAtStartPar
open flightmaps

\item {} 
\sphinxAtStartPar
Map Tiler

\item {} 
\sphinxAtStartPar
Open Street Map

\end{itemize}
\begin{description}
\item[{To get more detailed Information on these Resources you may touch the link on the lower edge of the map Display \sphinxstylestrong{Map Data Copyright Info}. After touching the line \sphinxstylestrong{Map Data Copyright Info} a sub window will open showing links to the contributor web sites:}] \leavevmode\begin{itemize}
\item {} 
\sphinxAtStartPar
\sphinxurl{https://www.openaip.net}

\item {} 
\sphinxAtStartPar
\sphinxurl{https://www.openflightmaps.org}

\item {} 
\sphinxAtStartPar
\sphinxurl{https://www.maptiler.com}

\item {} 
\sphinxAtStartPar
\sphinxurl{https://www.openstreetmap.org}

\end{itemize}

\end{description}

\sphinxAtStartPar
\sphinxstylestrong{Open AIP}

\sphinxAtStartPar
Open AIP has the goal to deliver free, current and precise data for air navigation to everyone. Open AIP is a web based and crowd\sphinxhyphen{}sourced platform.
The Open AIP provides the basic source aeronautical data for display in Enroute Flight Navigation.

\sphinxAtStartPar
\sphinxstylestrong{Open Flight Maps}

\sphinxAtStartPar
Open Flight Maps is an open\sphinxhyphen{}source project providing aeronautical data for a high quality VFR Map.
Open Flight Maps is providing some additional information, where available.

\sphinxAtStartPar
The detailed split of the data sources for the Enroute Flight Naviagtion map is shown below:


\begin{savenotes}\sphinxattablestart
\centering
\begin{tabulary}{\linewidth}[t]{|T|T|}
\hline
\sphinxstyletheadfamily 
\sphinxAtStartPar
Map Feature
&\sphinxstyletheadfamily 
\sphinxAtStartPar
Data Origin
\\
\hline
\sphinxAtStartPar
Airfields
&
\sphinxAtStartPar
openAIP
\\
\hline
\sphinxAtStartPar
Airspace: Nature Preserve Areas
&
\sphinxAtStartPar
open flightmaps
\\
\hline
\sphinxAtStartPar
Airspace: all other
&
\sphinxAtStartPar
openAIP
\\
\hline
\sphinxAtStartPar
Navaids
&
\sphinxAtStartPar
openAIP
\\
\hline
\sphinxAtStartPar
Procedures (Traffic Circuits, …)
&
\sphinxAtStartPar
open flightmaps
\\
\hline
\sphinxAtStartPar
Reporting Points
&
\sphinxAtStartPar
open flightmaps
\\
\hline
\end{tabulary}
\par
\sphinxattableend\end{savenotes}

\sphinxAtStartPar
\sphinxstylestrong{Map Tiler}

\sphinxAtStartPar
Is a software application to combine multiple layers of data for maps and provide the map in a format for loading and display.
The Enroute Flight Naviagtion base maps are edited versions of maps kindly provided by Klokan Technologies through the OpenMapTiles project.

\sphinxAtStartPar
\sphinxstylestrong{Open Street Map}

\sphinxAtStartPar
Open Street Map (OSM) is a collaborative project to create a free editable map of the world. The geodata underlying the map is considered the primary output of the project. The creation and growth of OSM has been motivated by restrictions on use or availability of map data across much of the world, and the advent of inexpensive portable satellite navigation devices.
The Open Street Map data is used to crate the base maps.


\chapter{Other}
\label{\detokenize{03-reference/map_data:other}}
\sphinxAtStartPar
The Map display is composed of two layers selected in the respective Tabs of the
‘Map Library’ screen:
\begin{itemize}
\item {} 
\sphinxAtStartPar
Aeronautical Map

\item {} 
\sphinxAtStartPar
Base Map

\end{itemize}

\sphinxAtStartPar
\sphinxstylestrong{Aeronautical Maps}

\sphinxAtStartPar
The Aeronautical Map layers is showing the airspace data on the Map screen. If
no Base Map is installed for the area only the information coming from the
Aviation Map data is displayed.

\sphinxAtStartPar
The Aeronautical Map contains:
\begin{itemize}
\item {} 
\sphinxAtStartPar
Airfields

\item {} 
\sphinxAtStartPar
Airspace boundaries

\item {} 
\sphinxAtStartPar
Navaids

\item {} 
\sphinxAtStartPar
Reporting points and routes (if available)

\end{itemize}

\sphinxAtStartPar
The display used for aerospace data is using the following basic color scheme:
\begin{itemize}
\item {} \begin{description}
\item[{Red:}] \leavevmode\begin{itemize}
\item {} 
\sphinxAtStartPar
Line with shadow inside for Restricted Airspace

\item {} 
\sphinxAtStartPar
Shadow with dashed blue border for Aerodrome Control Zone (CTR)

\item {} 
\sphinxAtStartPar
Dashed Line for Parachute Jumping Exercise area

\item {} 
\sphinxAtStartPar
Line for Glider or Microlight Traffic pattern

\end{itemize}

\end{description}

\item {} 
\sphinxAtStartPar
Blue:
\begin{itemize}
\item {} 
\sphinxAtStartPar
Line with shadow for controlled airspace (A, B, C, D)

\item {} 
\sphinxAtStartPar
Shadow with dashed blue border for Radio Mandatory Zone (RMZ)

\item {} 
\sphinxAtStartPar
Airport, reporting point or Navaid  symbols

\item {} 
\sphinxAtStartPar
For Route or Traffic Pattern for powered aircraft

\end{itemize}

\item {} 
\sphinxAtStartPar
Green:
\begin{itemize}
\item {} 
\sphinxAtStartPar
Line with shadow for Natural Reserve Area (NRA)

\item {} 
\sphinxAtStartPar
Line for airspace control sector boundaries

\end{itemize}

\item {} 
\sphinxAtStartPar
Black:
\begin{itemize}
\item {} 
\sphinxAtStartPar
Dashed Line for Transponder Mandatory Zone (TMZ)

\end{itemize}

\end{itemize}

\sphinxAtStartPar
\sphinxstylestrong{Class 1 and Class 2 maps:}
\begin{itemize}
\item {} 
\sphinxAtStartPar
Class 1 maps are compiled from openAIP and open flightmaps data. These maps
contain complete information about airspaces, airfields and navaids. In
addition, the maps contain (mandatory) reporting points. Some of our tier 1
maps also show traffic circuits and flight procedures for control zones.

\item {} 
\sphinxAtStartPar
Class 2 maps are compiled from openAIP data only. They contain complete
information about airspaces, airfields and navaids.

\end{itemize}

\sphinxAtStartPar
Details on the maps may be found at
\textless{}\sphinxurl{https://akaflieg-freiburg.github.io/enroute/maps/}\textgreater{} The Aeronautical Map data is
selected on the “Map Library” \textendash{} “Aviation Data” page accessed via the “Settings”
Menu.  To update the list of available maps the “…” option in the upper right
corner of the screen may be used.  You may install or uninstall the aviation Map
data for a county by the selection on the right hand side of the country
list. To find a country you have to scroll up and down in the list.

\begin{sphinxadmonition}{note}{Note:}
\sphinxAtStartPar
To have optimum presentation of the \sphinxstylestrong{Enroute Flight Navigation} map
display install the Aviation Map and the Base Map for all areas you intend
to use \sphinxstylestrong{Enroute Flight Navigation}.
\end{sphinxadmonition}

\begin{sphinxadmonition}{caution}{Caution:}
\sphinxAtStartPar
No airspace information will be provided in country when the
Aeronautical Map is not installed for it.
\end{sphinxadmonition}

\begin{sphinxadmonition}{note}{Note:}
\sphinxAtStartPar
\sphinxstylestrong{Enroute Flight Navigation} will automatically check for updated
Maps on the Enroute server and show a pop\sphinxhyphen{}up window after start if updated
maps have been detected.  You will be asked if you want to update the map or
delay the update.
\end{sphinxadmonition}

\sphinxAtStartPar
\sphinxstylestrong{Base Map}

\sphinxAtStartPar
The Base Map layers is showing the geographic data on the Map screen. If no Base
Map is shown for an area it will be shown in the white background color. If no
Aviation Map is installed for the area only the information coming from the Base
Map data is displayed. The Base Map is organized in tiles. This will result in
not stopping the Base Map display abruptly at the border of an installed
country, but showing some overlap.  The Base Map will show:
\begin{itemize}
\item {} 
\sphinxAtStartPar
Landmass

\item {} 
\sphinxAtStartPar
Water Surface (oceans, lakes and rivers)

\item {} 
\sphinxAtStartPar
Forests

\item {} 
\sphinxAtStartPar
Main Roads

\item {} 
\sphinxAtStartPar
Railroad lines

\item {} 
\sphinxAtStartPar
City names

\end{itemize}

\begin{sphinxadmonition}{note}{Note:}
\sphinxAtStartPar
To have optimum presentation of the \sphinxstylestrong{Enroute Flight Navigation} map
display install the Aeronautical Map and the Base Map for all areas you
intend to use \sphinxstylestrong{Enroute Flight Navigation}.
\end{sphinxadmonition}

\begin{sphinxadmonition}{note}{Note:}
\sphinxAtStartPar
\sphinxstylestrong{Enroute Flight Navigation} will not show most cultural build ups
and limits or settled area boundaries to reduce the map size.
\end{sphinxadmonition}


\section{Flight mode and ground mode}
\label{\detokenize{03-reference/map_data:flight-mode-and-ground-mode}}
\sphinxAtStartPar
\sphinxstylestrong{Ground Mode}

\sphinxAtStartPar
Ground Mode is displayed by \sphinxstylestrong{Enroute Flight Navigation} whenever the sensed
speed is below the threshold and the Menu item ‘Automatic Flight Detection’ is
not set to ‘Always in Flight Mode’.  Ground Mode does not display the Flight
Data line at the lower end of the screen and is intended for flight planning.


\chapter{Airspace Display}
\label{\detokenize{03-reference/airspace_display:airspace-display}}\label{\detokenize{03-reference/airspace_display::doc}}
\sphinxAtStartPar
The display of airspace will generally follow the common ICAO symbology.
Restricted Airspace
Restricted airspace will be surrounded by an intense red dashed line and a thick transparent red line inside the restricted area boundaries.
When selecting a point inside the restricted area by double touching the screen the information to the related area is given with the waypoint pop\sphinxhyphen{}up window:
\begin{itemize}
\item {} 
\sphinxAtStartPar
Area Name

\item {} 
\sphinxAtStartPar
Area altitude limits

\item {} 
\sphinxAtStartPar
Area activation time

\end{itemize}

\begin{figure}[htbp]
\centering

\noindent\sphinxincludegraphics{{fig_Restricted}.png}
\end{figure}

\sphinxAtStartPar
\sphinxstyleemphasis{Legend}:
\begin{enumerate}
\sphinxsetlistlabels{\arabic}{enumi}{enumii}{}{.}%
\item {} 
\sphinxAtStartPar
Outline of Restricted Airspace

\item {} 
\sphinxAtStartPar
Designation and activation time of airspace

\end{enumerate}


\section{Controlled Airspace}
\label{\detokenize{03-reference/airspace_display:controlled-airspace}}
\sphinxAtStartPar
All boundaries of controlled airspace are shown by a solid blue line and a thick transparent blue line inside the airspace. Figure 13:  Controlled Airspace
When selecting a point inside the controlled airspace by double touching the screen the information to the related area is given with the waypoint pop\sphinxhyphen{}up window:
\begin{itemize}
\item {} 
\sphinxAtStartPar
Area Name

\item {} 
\sphinxAtStartPar
Area altitude limits

\end{itemize}

\begin{sphinxadmonition}{caution}{Caution:}
\sphinxAtStartPar
All controlled airspace (Class A \textendash{} Class D) are shown in the same way even if different restrictions or ATC clearance requirements may be present.
\end{sphinxadmonition}


\section{Control Zone}
\label{\detokenize{03-reference/airspace_display:control-zone}}
\sphinxAtStartPar
The Control Zone of an airport is shown with a dashed blue line filled in transparent red color. Figure 13:  Controlled Airspace
When selecting a point inside the Control Zone (CTR) by double touching the screen the information to the related area is given with the waypoint pop\sphinxhyphen{}up window:
\begin{itemize}
\item {} 
\sphinxAtStartPar
Area Name

\item {} 
\sphinxAtStartPar
Area altitude limits

\end{itemize}

\begin{figure}[htbp]
\centering

\noindent\sphinxincludegraphics{{fig_AirspaceMUC}.png}
\end{figure}

\sphinxAtStartPar
\sphinxstyleemphasis{Legend}:
\begin{enumerate}
\sphinxsetlistlabels{\arabic}{enumi}{enumii}{}{.}%
\item {} 
\sphinxAtStartPar
Airport ICAO Symbol

\item {} 
\sphinxAtStartPar
Airport Control Zone (CTR)

\item {} 
\sphinxAtStartPar
Radio Mandatory Zone (RMZ)

\item {} 
\sphinxAtStartPar
Boundary of Controlled Airspace

\item {} 
\sphinxAtStartPar
Restricted Airspace

\end{enumerate}


\section{Transponder Mandatory Zones}
\label{\detokenize{03-reference/airspace_display:transponder-mandatory-zones}}
\sphinxAtStartPar
Transponder Mandatory Zones TMZ are shown with a black dashed outline.
When selecting a point inside the Transponder Mandatory Zone (TMZ) by double touching the screen the information to the related ares is given with the waypoint pop\sphinxhyphen{}up window:
\begin{itemize}
\item {} 
\sphinxAtStartPar
Area Name

\item {} 
\sphinxAtStartPar
Area altitude limits

\item {} 
\sphinxAtStartPar
Monitoring Frequency

\item {} 
\sphinxAtStartPar
Mode 3 Squawk

\end{itemize}


\section{Radio Mandatory Zone}
\label{\detokenize{03-reference/airspace_display:radio-mandatory-zone}}
\sphinxAtStartPar
Radio Mandatory Zones (RMZ) are shown with a solid blue dashed outline and filled in transparent blue.
When selecting a point inside the Radio Mandatory Zone (RMZ) by double touching the screen the information to the related area is given with the waypoint pop\sphinxhyphen{}up window:
\begin{itemize}
\item {} 
\sphinxAtStartPar
Area Name

\item {} 
\sphinxAtStartPar
Area altitude limits

\item {} 
\sphinxAtStartPar
Radio Frequency

\end{itemize}


\section{Parachute Jumping Areas}
\label{\detokenize{03-reference/airspace_display:parachute-jumping-areas}}
\sphinxAtStartPar
Parachute Jumping Exercise areas (PJE) are shown with a solid red dashed outline.
When selecting a point inside the PJE by double touching the screen the information to the related area is given with the waypoint pop\sphinxhyphen{}up window:
\begin{itemize}
\item {} 
\sphinxAtStartPar
Area Name

\item {} 
\sphinxAtStartPar
Area altitude limits

\item {} 
\sphinxAtStartPar
Radio Frequency

\end{itemize}


\section{Nature Reserve Areas}
\label{\detokenize{03-reference/airspace_display:nature-reserve-areas}}
\sphinxAtStartPar
Nature Reserve Areas (NRA) are shown with a solid green outline.
When selecting a point inside the NRA by double touching the screen the information to the related area is given with the waypoint pop\sphinxhyphen{}up window:
\begin{itemize}
\item {} 
\sphinxAtStartPar
Area Name

\item {} 
\sphinxAtStartPar
Area altitude limits

\end{itemize}

\begin{sphinxadmonition}{caution}{Caution:}
\sphinxAtStartPar
Check restrictions applicable for flying inside NRA when planning your flight. For example in Austria high fines are applicable when flying inside NRA.
\begin{quote}

\sphinxAtStartPar
Figure 14:  Nature Reserve Area
\end{quote}
\end{sphinxadmonition}

\begin{figure}[htbp]
\centering

\noindent\sphinxincludegraphics{{fig_nra}.png}
\end{figure}

\sphinxAtStartPar
\sphinxstyleemphasis{Legend}:
\begin{enumerate}
\sphinxsetlistlabels{\arabic}{enumi}{enumii}{}{.}%
\item {} 
\sphinxAtStartPar
Outline of Nature Reserve Area (NRA)

\item {} 
\sphinxAtStartPar
Designation of NRA

\end{enumerate}


\section{Airfields}
\label{\detokenize{03-reference/airspace_display:airfields}}
\sphinxAtStartPar
The symbology used to display airfields follows the ICAO rules.
Airfield Information
When selecting an airfield by double touching the screen the related information is given in a pop\sphinxhyphen{}up window:
\begin{itemize}
\item {} 
\sphinxAtStartPar
Airfield Name and Identifier

\item {} 
\sphinxAtStartPar
Radio Frequency including COM and Information frequencies

\item {} 
\sphinxAtStartPar
Navaid frequencies

\item {} 
\sphinxAtStartPar
Runway orientation, dimensions and surface

\item {} 
\sphinxAtStartPar
Field elevation

\item {} 
\sphinxAtStartPar
Data for associated airspace

\end{itemize}


\section{Approach and Departure Routes}
\label{\detokenize{03-reference/airspace_display:approach-and-departure-routes}}
\sphinxAtStartPar
Approach routes to airfields are shown as solid blue lines. The designation of the route is written along the paths. The associated reporting points are shown as blue triangles with a dashed circle and the reporting point designation.
Approach Routes will be shown by a solid line and Departure Routes will be shown as  dashed lines.
Note
Approach Routes will only be displayed when zooming into the area.
Traffic Pattern
Traffic pattern for motorized aircraft are shown as blue lines.
Traffic circuits for gliders or Ultralight aircraft are shown as red lines.
Entry and exit routes to traffic pattern are indicated by open ends of the pattern.
The traffic circuit will show the traffic circuit altitude when the information is available.
Note
Traffic pattern will only be displayed when zooming into the area.


\chapter{Weather}
\label{\detokenize{03-reference/weather:weather}}\label{\detokenize{03-reference/weather::doc}}
\sphinxAtStartPar
The Weather page is opened via the Menu by touching the “Weather” entry.
The Weather page will display the station overview list for all currently available meteorological reports within 200 NM of the current position.

\begin{figure}[htbp]
\centering

\noindent\sphinxincludegraphics{{fig_Weather}.png}
\end{figure}

\sphinxAtStartPar
\sphinxstyleemphasis{Legend}:
\begin{enumerate}
\sphinxsetlistlabels{\arabic}{enumi}{enumii}{}{.}%
\item {} 
\sphinxAtStartPar
Weather Menu

\item {} 
\sphinxAtStartPar
Station data

\item {} 
\sphinxAtStartPar
Meteorological data closest to own position

\end{enumerate}

\sphinxAtStartPar
The weather data is downloaded from the National Weather Service of the United States of America.

\begin{sphinxadmonition}{note}{Note:}
\sphinxAtStartPar
When opening the Weather page the first time you will have to confirm that you agree to download data from the NWS server to use this service.
\end{sphinxadmonition}

\sphinxAtStartPar
The menu of the Waether page will allow to:
\begin{itemize}
\item {} 
\sphinxAtStartPar
Update the METAR and TAF data

\item {} 
\sphinxAtStartPar
Disallow he internet connection

\end{itemize}

\sphinxAtStartPar
The Weather overview window will provide the following information based on the METAR:
\begin{itemize}
\item {} 
\sphinxAtStartPar
ICAO identifier for Station and Airport name

\item {} 
\sphinxAtStartPar
Distance and magnetic Bearing to Airport

\item {} 
\sphinxAtStartPar
Time of METAR and summary weather state

\end{itemize}

\sphinxAtStartPar
On the lower end of the weather page the following data relevant to your current position will be displayed:
\begin{itemize}
\item {} 
\sphinxAtStartPar
QNH

\item {} 
\sphinxAtStartPar
Location and time of the report the QNH was extracted

\item {} 
\sphinxAtStartPar
Sunset during day or Sunrise during night at current location

\item {} 
\sphinxAtStartPar
Remaining time until sunset or sunrise

\end{itemize}

\sphinxAtStartPar
The information of each airport will be color coded by a system established by the US National Weather Service. The coding scheme is explained in the table below.
When touching a station line METAR and TAF (if available) will be shown in a weather detail sub\sphinxhyphen{}page

\begin{figure}[htbp]
\centering

\noindent\sphinxincludegraphics{{fig_WeatherDetail}.png}
\end{figure}

\sphinxAtStartPar
\sphinxstyleemphasis{Legend}:
\begin{enumerate}
\sphinxsetlistlabels{\arabic}{enumi}{enumii}{}{.}%
\item {} 
\sphinxAtStartPar
Station data including bering and distance

\item {} 
\sphinxAtStartPar
Current meteorological report

\item {} 
\sphinxAtStartPar
Decoded view of Current meteorological report

\item {} 
\sphinxAtStartPar
Weather forecast for station

\item {} 
\sphinxAtStartPar
Decoded view of weather forecast

\end{enumerate}

\begin{sphinxadmonition}{note}{Note:}
\sphinxAtStartPar
To view the full weather forecast you have to scroll down in most cases
\end{sphinxadmonition}

\begin{sphinxadmonition}{caution}{Caution:}
\sphinxAtStartPar
The color coding used for station weather does not match to European VFR criteria. Assessment of  meteorological flight conditions has to be done via an officially approved source of flight weather.
\end{sphinxadmonition}


\begin{savenotes}\sphinxattablestart
\centering
\begin{tabulary}{\linewidth}[t]{|T|T|T|T|T|}
\hline
\sphinxstyletheadfamily 
\sphinxAtStartPar
Category
&\sphinxstyletheadfamily 
\sphinxAtStartPar
Color
&\sphinxstyletheadfamily 
\sphinxAtStartPar
Ceiling
&\sphinxstyletheadfamily &\sphinxstyletheadfamily 
\sphinxAtStartPar
Visibility
\\
\hline
\sphinxAtStartPar
IFR
Instrument Flight Rules
&
\sphinxAtStartPar
Red
&
\sphinxAtStartPar
500 to below
1,000 feet AGL
&
\sphinxAtStartPar
and
/or
&
\sphinxAtStartPar
1 mile to
less than 3 miles
\\
\hline
\sphinxAtStartPar
MVFR
Marginal Visual Flight Rules
&
\sphinxAtStartPar
Yellow
&
\sphinxAtStartPar
1,000 to
3,000 feet AGL
&
\sphinxAtStartPar
and
/or
&
\sphinxAtStartPar
3 to 5 miles
\\
\hline
\sphinxAtStartPar
VFR
Visual Flight Rules
&
\sphinxAtStartPar
Green
&
\sphinxAtStartPar
greater than
3,000 feet AGL
&
\sphinxAtStartPar
and
/or
&
\sphinxAtStartPar
greater than
5 miles
\\
\hline
\end{tabulary}
\par
\sphinxattableend\end{savenotes}

\begin{sphinxadmonition}{note}{Note:}
\sphinxAtStartPar
By definition, IFR is ceiling less than 1,000 feet AGL.
\end{sphinxadmonition}

\begin{sphinxadmonition}{note}{Note:}
\sphinxAtStartPar
By definition, VFR is ceiling greater than or equal to 3,000 feet AGL and visibility greater than or equal to 5 miles while MVFR is a sub\sphinxhyphen{}category of VFR.
\end{sphinxadmonition}

\part{Appendix}


\chapter{Software license}
\label{\detokenize{04-appendix/license_enroute:software-license}}\label{\detokenize{04-appendix/license_enroute::doc}}
\sphinxAtStartPar
The program \sphinxstylestrong{Enroute Flight Navigation} is licensed under the \sphinxhref{https://www.gnu.org/licenses/gpl-3.0-standalone.html}{GNU General
Public License V3}%
\begin{footnote}[18]\sphinxAtStartFootnote
\sphinxnolinkurl{https://www.gnu.org/licenses/gpl-3.0-standalone.html}
%
\end{footnote} or,
at your choice, any later version of this license.

\begin{sphinxVerbatim}[commandchars=\\\{\}]
 GNU GENERAL PUBLIC LICENSE

 Version 3, 29 June 2007

 Copyright © 2007 Free Software Foundation, Inc. \PYGZlt{}https://fsf.org/\PYGZgt{}

 Everyone is permitted to copy and distribute verbatim copies of this license
 document, but changing it is not allowed.

 Preamble

   The GNU General Public License is a free, copyleft license for software and
   other kinds of works.

   The licenses for most software and other practical works are designed to
   take away your freedom to share and change the works. By contrast, the GNU
   General Public License is intended to guarantee your freedom to share and
   change all versions of a program\PYGZhy{}\PYGZhy{}to make sure it remains free software for
   all its users. We, the Free Software Foundation, use the GNU General Public
   License for most of our software; it applies also to any other work released
   this way by its authors. You can apply it to your programs, too.

   When we speak of free software, we are referring to freedom, not price. Our
   General Public Licenses are designed to make sure that you have the freedom
   to distribute copies of free software (and charge for them if you wish),
   that you receive source code or can get it if you want it, that you can
   change the software or use pieces of it in new free programs, and that you
   know you can do these things.

   To protect your rights, we need to prevent others from denying you these
   rights or asking you to surrender the rights. Therefore, you have certain
   responsibilities if you distribute copies of the software, or if you modify
   it: responsibilities to respect the freedom of others.

   For example, if you distribute copies of such a program, whether gratis or
   for a fee, you must pass on to the recipients the same freedoms that you
   received. You must make sure that they, too, receive or can get the source
   code. And you must show them these terms so they know their rights.

   Developers that use the GNU GPL protect your rights with two steps: (1)
   assert copyright on the software, and (2) offer you this License giving you
   legal permission to copy, distribute and/or modify it.

   For the developers\PYGZsq{} and authors\PYGZsq{} protection, the GPL clearly explains that
   there is no warranty for this free software. For both users\PYGZsq{} and authors\PYGZsq{}
   sake, the GPL requires that modified versions be marked as changed, so that
   their problems will not be attributed erroneously to authors of previous
   versions.

   Some devices are designed to deny users access to install or run modified
   versions of the software inside them, although the manufacturer can do
   so. This is fundamentally incompatible with the aim of protecting users\PYGZsq{}
   freedom to change the software. The systematic pattern of such abuse occurs
   in the area of products for individuals to use, which is precisely where it
   is most unacceptable. Therefore, we have designed this version of the GPL to
   prohibit the practice for those products. If such problems arise
   substantially in other domains, we stand ready to extend this provision to
   those domains in future versions of the GPL, as needed to protect the
   freedom of users.

   Finally, every program is threatened constantly by software patents. States
   should not allow patents to restrict development and use of software on
   general\PYGZhy{}purpose computers, but in those that do, we wish to avoid the
   special danger that patents applied to a free program could make it
   effectively proprietary. To prevent this, the GPL assures that patents
   cannot be used to render the program non\PYGZhy{}free.

   The precise terms and conditions for copying, distribution and modification
   follow.

 TERMS AND CONDITIONS

 0. Definitions.

   “This License” refers to version 3 of the GNU General Public License.

   “Copyright” also means copyright\PYGZhy{}like laws that apply to other kinds of
   works, such as semiconductor masks.

   “The Program” refers to any copyrightable work licensed under this
   License. Each licensee is addressed as “you”. “Licensees” and “recipients”
   may be individuals or organizations.

   To “modify” a work means to copy from or adapt all or part of the work in a
   fashion requiring copyright permission, other than the making of an exact
   copy. The resulting work is called a “modified version” of the earlier work
   or a work “based on” the earlier work.

   A “covered work” means either the unmodified Program or a work based on the
   Program.

   To “propagate” a work means to do anything with it that, without permission,
   would make you directly or secondarily liable for infringement under
   applicable copyright law, except executing it on a computer or modifying a
   private copy. Propagation includes copying, distribution (with or without
   modification), making available to the public, and in some countries other
   activities as well.

   To “convey” a work means any kind of propagation that enables other parties
   to make or receive copies. Mere interaction with a user through a computer
   network, with no transfer of a copy, is not conveying.

   An interactive user interface displays “Appropriate Legal Notices” to the
   extent that it includes a convenient and prominently visible feature
   that (1) displays an appropriate copyright notice, and (2) tells the user
   that there is no warranty for the work (except to the extent that warranties
   are provided), that licensees may convey the work under this License, and
   how to view a copy of this License. If the interface presents a list of user
   commands or options, such as a menu, a prominent item in the list meets this
   criterion.

 1. Source Code.

   The “source code” for a work means the preferred form of the work for making
   modifications to it. “Object code” means any non\PYGZhy{}source form of a work.

   A “Standard Interface” means an interface that either is an official
   standard defined by a recognized standards body, or, in the case of
   interfaces specified for a particular programming language, one that is
   widely used among developers working in that language.

   The “System Libraries” of an executable work include anything, other than
   the work as a whole, that (a) is included in the normal form of packaging a
   Major Component, but which is not part of that Major Component, and (b)
   serves only to enable use of the work with that Major Component, or to
   implement a Standard Interface for which an implementation is available to
   the public in source code form. A “Major Component”, in this context, means
   a major essential component (kernel, window system, and so on) of the
   specific operating system (if any) on which the executable work runs, or a
   compiler used to produce the work, or an object code interpreter used to run
   it.

   The “Corresponding Source” for a work in object code form means all the
   source code needed to generate, install, and (for an executable work) run
   the object code and to modify the work, including scripts to control those
   activities. However, it does not include the work\PYGZsq{}s System Libraries, or
   general\PYGZhy{}purpose tools or generally available free programs which are used
   unmodified in performing those activities but which are not part of the
   work. For example, Corresponding Source includes interface definition files
   associated with source files for the work, and the source code for shared
   libraries and dynamically linked subprograms that the work is specifically
   designed to require, such as by intimate data communication or control flow
   between those subprograms and other parts of the work.

   The Corresponding Source need not include anything that users can regenerate
   automatically from other parts of the Corresponding Source.

   The Corresponding Source for a work in source code form is that same work.

 2. Basic Permissions.

   All rights granted under this License are granted for the term of copyright
   on the Program, and are irrevocable provided the stated conditions are
   met. This License explicitly affirms your unlimited permission to run the
   unmodified Program. The output from running a covered work is covered by
   this License only if the output, given its content, constitutes a covered
   work. This License acknowledges your rights of fair use or other equivalent,
   as provided by copyright law.

   You may make, run and propagate covered works that you do not convey,
   without conditions so long as your license otherwise remains in force. You
   may convey covered works to others for the sole purpose of having them make
   modifications exclusively for you, or provide you with facilities for
   running those works, provided that you comply with the terms of this License
   in conveying all material for which you do not control copyright. Those thus
   making or running the covered works for you must do so exclusively on your
   behalf, under your direction and control, on terms that prohibit them from
   making any copies of your copyrighted material outside their relationship
   with you.

   Conveying under any other circumstances is permitted solely under the
   conditions stated below. Sublicensing is not allowed; section 10 makes it
   unnecessary.

 3. Protecting Users\PYGZsq{} Legal Rights From Anti\PYGZhy{}Circumvention Law.

   No covered work shall be deemed part of an effective technological measure
   under any applicable law fulfilling obligations under article 11 of the WIPO
   copyright treaty adopted on 20 December 1996, or similar laws prohibiting or
   restricting circumvention of such measures.

   When you convey a covered work, you waive any legal power to forbid
   circumvention of technological measures to the extent such circumvention is
   effected by exercising rights under this License with respect to the covered
   work, and you disclaim any intention to limit operation or modification of
   the work as a means of enforcing, against the work\PYGZsq{}s users, your or third
   parties\PYGZsq{} legal rights to forbid circumvention of technological measures.

 4. Conveying Verbatim Copies.

   You may convey verbatim copies of the Program\PYGZsq{}s source code as you receive
   it, in any medium, provided that you conspicuously and appropriately publish
   on each copy an appropriate copyright notice; keep intact all notices
   stating that this License and any non\PYGZhy{}permissive terms added in accord with
   section 7 apply to the code; keep intact all notices of the absence of any
   warranty; and give all recipients a copy of this License along with the
   Program.

   You may charge any price or no price for each copy that you convey, and you
   may offer support or warranty protection for a fee.

 5. Conveying Modified Source Versions.

   You may convey a work based on the Program, or the modifications to produce
   it from the Program, in the form of source code under the terms of section
   4, provided that you also meet all of these conditions:

   a) The work must carry prominent notices stating that you modified it, and
      giving a relevant date.

   b) The work must carry prominent notices stating that it is released under
      this License and any conditions added under section 7. This requirement
      modifies the requirement in section 4 to “keep intact all notices”.

   c) You must license the entire work, as a whole, under this License to
      anyone who comes into possession of a copy. This License will therefore
      apply, along with any applicable section 7 additional terms, to the whole
      of the work, and all its parts, regardless of how they are packaged. This
      License gives no permission to license the work in any other way, but it
      does not invalidate such permission if you have separately received it.

   d) If the work has interactive user interfaces, each must display
      Appropriate Legal Notices; however, if the Program has interactive
      interfaces that do not display Appropriate Legal Notices, your work need
      not make them do so.

   A compilation of a covered work with other separate and independent works,
   which are not by their nature extensions of the covered work, and which are
   not combined with it such as to form a larger program, in or on a volume of
   a storage or distribution medium, is called an “aggregate” if the
   compilation and its resulting copyright are not used to limit the access or
   legal rights of the compilation\PYGZsq{}s users beyond what the individual works
   permit. Inclusion of a covered work in an aggregate does not cause this
   License to apply to the other parts of the aggregate.

 6. Conveying Non\PYGZhy{}Source Forms.

   You may convey a covered work in object code form under the terms of
   sections 4 and 5, provided that you also convey the machine\PYGZhy{}readable
   Corresponding Source under the terms of this License, in one of these ways:

   a) Convey the object code in, or embodied in, a physical product (including
      a physical distribution medium), accompanied by the Corresponding Source
      fixed on a durable physical medium customarily used for software
      interchange.

   b) Convey the object code in, or embodied in, a physical product (including
      a physical distribution medium), accompanied by a written offer, valid
      for at least three years and valid for as long as you offer spare parts
      or customer support for that product model, to give anyone who possesses
      the object code either (1) a copy of the Corresponding Source for all the
      software in the product that is covered by this License, on a durable
      physical medium customarily used for software interchange, for a price no
      more than your reasonable cost of physically performing this conveying of
      source, or (2) access to copy the Corresponding Source from a network
      server at no charge.

   c) Convey individual copies of the object code with a copy of the written
      offer to provide the Corresponding Source. This alternative is allowed
      only occasionally and noncommercially, and only if you received the
      object code with such an offer, in accord with subsection 6b.

   d) Convey the object code by offering access from a designated place (gratis
      or for a charge), and offer equivalent access to the Corresponding Source
      in the same way through the same place at no further charge. You need not
      require recipients to copy the Corresponding Source along with the object
      code. If the place to copy the object code is a network server, the
      Corresponding Source may be on a different server (operated by you or a
      third party) that supports equivalent copying facilities, provided you
      maintain clear directions next to the object code saying where to find
      the Corresponding Source. Regardless of what server hosts the
      Corresponding Source, you remain obligated to ensure that it is available
      for as long as needed to satisfy these requirements.

   e) Convey the object code using peer\PYGZhy{}to\PYGZhy{}peer transmission, provided you
      inform other peers where the object code and Corresponding Source of the
      work are being offered to the general public at no charge under
      subsection 6d.

   A separable portion of the object code, whose source code is excluded from
   the Corresponding Source as a System Library, need not be included in
   conveying the object code work.

   A “User Product” is either (1) a “consumer product”, which means any
   tangible personal property which is normally used for personal, family, or
   household purposes, or (2) anything designed or sold for incorporation into
   a dwelling. In determining whether a product is a consumer product, doubtful
   cases shall be resolved in favor of coverage. For a particular product
   received by a particular user, “normally used” refers to a typical or common
   use of that class of product, regardless of the status of the particular
   user or of the way in which the particular user actually uses, or expects or
   is expected to use, the product. A product is a consumer product regardless
   of whether the product has substantial commercial, industrial or
   non\PYGZhy{}consumer uses, unless such uses represent the only significant mode of
   use of the product.

   “Installation Information” for a User Product means any methods, procedures,
   authorization keys, or other information required to install and execute
   modified versions of a covered work in that User Product from a modified
   version of its Corresponding Source. The information must suffice to ensure
   that the continued functioning of the modified object code is in no case
   prevented or interfered with solely because modification has been made.

   If you convey an object code work under this section in, or with, or
   specifically for use in, a User Product, and the conveying occurs as part of
   a transaction in which the right of possession and use of the User Product
   is transferred to the recipient in perpetuity or for a fixed term
   (regardless of how the transaction is characterized), the Corresponding
   Source conveyed under this section must be accompanied by the Installation
   Information. But this requirement does not apply if neither you nor any
   third party retains the ability to install modified object code on the User
   Product (for example, the work has been installed in ROM).

   The requirement to provide Installation Information does not include a
   requirement to continue to provide support service, warranty, or updates for
   a work that has been modified or installed by the recipient, or for the User
   Product in which it has been modified or installed. Access to a network may
   be denied when the modification itself materially and adversely affects the
   operation of the network or violates the rules and protocols for
   communication across the network.

   Corresponding Source conveyed, and Installation Information provided, in
   accord with this section must be in a format that is publicly documented
   (and with an implementation available to the public in source code form),
   and must require no special password or key for unpacking, reading or
   copying.

 7. Additional Terms.

    “Additional permissions” are terms that supplement the terms of this
    License by making exceptions from one or more of its conditions. Additional
    permissions that are applicable to the entire Program shall be treated as
    though they were included in this License, to the extent that they are
    valid under applicable law. If additional permissions apply only to part of
    the Program, that part may be used separately under those permissions, but
    the entire Program remains governed by this License without regard to the
    additional permissions.

    When you convey a copy of a covered work, you may at your option remove any
    additional permissions from that copy, or from any part of it. (Additional
    permissions may be written to require their own removal in certain cases
    when you modify the work.) You may place additional permissions on
    material, added by you to a covered work, for which you have or can give
    appropriate copyright permission.

    Notwithstanding any other provision of this License, for material you add
    to a covered work, you may (if authorized by the copyright holders of that
    material) supplement the terms of this License with terms:

    a) Disclaiming warranty or limiting liability differently from the terms of
       sections 15 and 16 of this License; or

    b) Requiring preservation of specified reasonable legal notices or author
       attributions in that material or in the Appropriate Legal Notices
       displayed by works containing it; or

    c) Prohibiting misrepresentation of the origin of that material, or
       requiring that modified versions of such material be marked in
       reasonable ways as different from the original version; or

    d) Limiting the use for publicity purposes of names of licensors or authors
       of the material; or

    e) Declining to grant rights under trademark law for use of some trade
       names, trademarks, or service marks; or

    f) Requiring indemnification of licensors and authors of that material by
       anyone who conveys the material (or modified versions of it) with
       contractual assumptions of liability to the recipient, for any liability
       that these contractual assumptions directly impose on those licensors
       and authors.

    All other non\PYGZhy{}permissive additional terms are considered “further
    restrictions” within the meaning of section 10. If the Program as you
    received it, or any part of it, contains a notice stating that it is
    governed by this License along with a term that is a further restriction,
    you may remove that term. If a license document contains a further
    restriction but permits relicensing or conveying under this License, you
    may add to a covered work material governed by the terms of that license
    document, provided that the further restriction does not survive such
    relicensing or conveying.

    If you add terms to a covered work in accord with this section, you must
    place, in the relevant source files, a statement of the additional terms
    that apply to those files, or a notice indicating where to find the
    applicable terms.

    Additional terms, permissive or non\PYGZhy{}permissive, may be stated in the form
    of a separately written license, or stated as exceptions; the above
    requirements apply either way.

 8. Termination.

    You may not propagate or modify a covered work except as expressly provided
    under this License. Any attempt otherwise to propagate or modify it is
    void, and will automatically terminate your rights under this License
    (including any patent licenses granted under the third paragraph of section
    11).

    However, if you cease all violation of this License, then your license from
    a particular copyright holder is reinstated (a) provisionally, unless and
    until the copyright holder explicitly and finally terminates your license,
    and (b) permanently, if the copyright holder fails to notify you of the
    violation by some reasonable means prior to 60 days after the cessation.

    Moreover, your license from a particular copyright holder is reinstated
    permanently if the copyright holder notifies you of the violation by some
    reasonable means, this is the first time you have received notice of
    violation of this License (for any work) from that copyright holder, and
    you cure the violation prior to 30 days after your receipt of the notice.

    Termination of your rights under this section does not terminate the
    licenses of parties who have received copies or rights from you under this
    License. If your rights have been terminated and not permanently
    reinstated, you do not qualify to receive new licenses for the same
    material under section 10.

 9. Acceptance Not Required for Having Copies.

    You are not required to accept this License in order to receive or run a
    copy of the Program. Ancillary propagation of a covered work occurring
    solely as a consequence of using peer\PYGZhy{}to\PYGZhy{}peer transmission to receive a
    copy likewise does not require acceptance. However, nothing other than this
    License grants you permission to propagate or modify any covered
    work. These actions infringe copyright if you do not accept this
    License. Therefore, by modifying or propagating a covered work, you
    indicate your acceptance of this License to do so.

 10. Automatic Licensing of Downstream Recipients.

    Each time you convey a covered work, the recipient automatically receives a
    license from the original licensors, to run, modify and propagate that
    work, subject to this License. You are not responsible for enforcing
    compliance by third parties with this License.

    An “entity transaction” is a transaction transferring control of an
    organization, or substantially all assets of one, or subdividing an
    organization, or merging organizations. If propagation of a covered work
    results from an entity transaction, each party to that transaction who
    receives a copy of the work also receives whatever licenses to the work the
    party\PYGZsq{}s predecessor in interest had or could give under the previous
    paragraph, plus a right to possession of the Corresponding Source of the
    work from the predecessor in interest, if the predecessor has it or can get
    it with reasonable efforts.

    You may not impose any further restrictions on the exercise of the rights
    granted or affirmed under this License. For example, you may not impose a
    license fee, royalty, or other charge for exercise of rights granted under
    this License, and you may not initiate litigation (including a cross\PYGZhy{}claim
    or counterclaim in a lawsuit) alleging that any patent claim is infringed
    by making, using, selling, offering for sale, or importing the Program or
    any portion of it.

 11. Patents.

    A “contributor” is a copyright holder who authorizes use under this License
    of the Program or a work on which the Program is based. The work thus
    licensed is called the contributor\PYGZsq{}s “contributor version”.

    A contributor\PYGZsq{}s “essential patent claims” are all patent claims owned or
    controlled by the contributor, whether already acquired or hereafter
    acquired, that would be infringed by some manner, permitted by this
    License, of making, using, or selling its contributor version, but do not
    include claims that would be infringed only as a consequence of further
    modification of the contributor version. For purposes of this definition,
    “control” includes the right to grant patent sublicenses in a manner
    consistent with the requirements of this License.

    Each contributor grants you a non\PYGZhy{}exclusive, worldwide, royalty\PYGZhy{}free patent
    license under the contributor\PYGZsq{}s essential patent claims, to make, use,
    sell, offer for sale, import and otherwise run, modify and propagate the
    contents of its contributor version.

    In the following three paragraphs, a “patent license” is any express
    agreement or commitment, however denominated, not to enforce a patent (such
    as an express permission to practice a patent or covenant not to sue for
    patent infringement). To “grant” such a patent license to a party means to
    make such an agreement or commitment not to enforce a patent against the
    party.

    If you convey a covered work, knowingly relying on a patent license, and
    the Corresponding Source of the work is not available for anyone to copy,
    free of charge and under the terms of this License, through a publicly
    available network server or other readily accessible means, then you must
    either (1) cause the Corresponding Source to be so available, or (2)
    arrange to deprive yourself of the benefit of the patent license for this
    particular work, or (3) arrange, in a manner consistent with the
    requirements of this License, to extend the patent license to downstream
    recipients. “Knowingly relying” means you have actual knowledge that, but
    for the patent license, your conveying the covered work in a country, or
    your recipient\PYGZsq{}s use of the covered work in a country, would infringe one
    or more identifiable patents in that country that you have reason to
    believe are valid.

    If, pursuant to or in connection with a single transaction or arrangement,
    you convey, or propagate by procuring conveyance of, a covered work, and
    grant a patent license to some of the parties receiving the covered work
    authorizing them to use, propagate, modify or convey a specific copy of the
    covered work, then the patent license you grant is automatically extended
    to all recipients of the covered work and works based on it.

    A patent license is “discriminatory” if it does not include within the
    scope of its coverage, prohibits the exercise of, or is conditioned on the
    non\PYGZhy{}exercise of one or more of the rights that are specifically granted
    under this License. You may not convey a covered work if you are a party to
    an arrangement with a third party that is in the business of distributing
    software, under which you make payment to the third party based on the
    extent of your activity of conveying the work, and under which the third
    party grants, to any of the parties who would receive the covered work from
    you, a discriminatory patent license (a) in connection with copies of the
    covered work conveyed by you (or copies made from those copies), or (b)
    primarily for and in connection with specific products or compilations that
    contain the covered work, unless you entered into that arrangement, or that
    patent license was granted, prior to 28 March 2007.

    Nothing in this License shall be construed as excluding or limiting any
    implied license or other defenses to infringement that may otherwise be
    available to you under applicable patent law.

 12. No Surrender of Others\PYGZsq{} Freedom.

   If conditions are imposed on you (whether by court order, agreement or
   otherwise) that contradict the conditions of this License, they do not
   excuse you from the conditions of this License. If you cannot convey a
   covered work so as to satisfy simultaneously your obligations under this
   License and any other pertinent obligations, then as a consequence you may
   not convey it at all. For example, if you agree to terms that obligate you
   to collect a royalty for further conveying from those to whom you convey the
   Program, the only way you could satisfy both those terms and this License
   would be to refrain entirely from conveying the Program.

13. Use with the GNU Affero General Public License.

   Notwithstanding any other provision of this License, you have permission to
   link or combine any covered work with a work licensed under version 3 of the
   GNU Affero General Public License into a single combined work, and to convey
   the resulting work. The terms of this License will continue to apply to the
   part which is the covered work, but the special requirements of the GNU
   Affero General Public License, section 13, concerning interaction through a
   network will apply to the combination as such.

 14. Revised Versions of this License.

   The Free Software Foundation may publish revised and/or new versions of the
   GNU General Public License from time to time. Such new versions will be
   similar in spirit to the present version, but may differ in detail to
   address new problems or concerns.

   Each version is given a distinguishing version number. If the Program
   specifies that a certain numbered version of the GNU General Public License
   “or any later version” applies to it, you have the option of following the
   terms and conditions either of that numbered version or of any later version
   published by the Free Software Foundation. If the Program does not specify a
   version number of the GNU General Public License, you may choose any version
   ever published by the Free Software Foundation.

   If the Program specifies that a proxy can decide which future versions of
   the GNU General Public License can be used, that proxy\PYGZsq{}s public statement of
   acceptance of a version permanently authorizes you to choose that version
   for the Program.

   Later license versions may give you additional or different
   permissions. However, no additional obligations are imposed on any author or
   copyright holder as a result of your choosing to follow a later version.

 15. Disclaimer of Warranty.

   THERE IS NO WARRANTY FOR THE PROGRAM, TO THE EXTENT PERMITTED BY APPLICABLE
   LAW. EXCEPT WHEN OTHERWISE STATED IN WRITING THE COPYRIGHT HOLDERS AND/OR
   OTHER PARTIES PROVIDE THE PROGRAM “AS IS” WITHOUT WARRANTY OF ANY KIND,
   EITHER EXPRESSED OR IMPLIED, INCLUDING, BUT NOT LIMITED TO, THE IMPLIED
   WARRANTIES OF MERCHANTABILITY AND FITNESS FOR A PARTICULAR PURPOSE. THE
   ENTIRE RISK AS TO THE QUALITY AND PERFORMANCE OF THE PROGRAM IS WITH
   YOU. SHOULD THE PROGRAM PROVE DEFECTIVE, YOU ASSUME THE COST OF ALL
   NECESSARY SERVICING, REPAIR OR CORRECTION.

 16. Limitation of Liability.

   IN NO EVENT UNLESS REQUIRED BY APPLICABLE LAW OR AGREED TO IN WRITING WILL
   ANY COPYRIGHT HOLDER, OR ANY OTHER PARTY WHO MODIFIES AND/OR CONVEYS THE
   PROGRAM AS PERMITTED ABOVE, BE LIABLE TO YOU FOR DAMAGES, INCLUDING ANY
   GENERAL, SPECIAL, INCIDENTAL OR CONSEQUENTIAL DAMAGES ARISING OUT OF THE USE
   OR INABILITY TO USE THE PROGRAM (INCLUDING BUT NOT LIMITED TO LOSS OF DATA
   OR DATA BEING RENDERED INACCURATE OR LOSSES SUSTAINED BY YOU OR THIRD
   PARTIES OR A FAILURE OF THE PROGRAM TO OPERATE WITH ANY OTHER PROGRAMS),
   EVEN IF SUCH HOLDER OR OTHER PARTY HAS BEEN ADVISED OF THE POSSIBILITY OF
   SUCH DAMAGES.

 17. Interpretation of Sections 15 and 16.

   If the disclaimer of warranty and limitation of liability provided above
   cannot be given local legal effect according to their terms, reviewing
   courts shall apply local law that most closely approximates an absolute
   waiver of all civil liability in connection with the Program, unless a
   warranty or assumption of liability accompanies a copy of the Program in
   return for a fee.
\end{sphinxVerbatim}


\chapter{Third party software and data}
\label{\detokenize{04-appendix/license_3rdParty:third-party-software-and-data}}\label{\detokenize{04-appendix/license_3rdParty::doc}}
\sphinxAtStartPar
\sphinxstylestrong{Enroute Flight Navigation} builds on a large number of open\sphinxhyphen{}source software
components and on open\sphinxhyphen{}source data.


\section{Geographic maps}
\label{\detokenize{04-appendix/license_3rdParty:geographic-maps}}
\sphinxAtStartPar
As a flight navigation program, \sphinxstylestrong{Enroute Flight Navigation} heavily relies on
geographic map data.  The geographic maps are not included in the program, but
are downloaded at runtime.  They are compiled from the following sources.
\begin{itemize}
\item {} 
\sphinxAtStartPar
The base maps are modified data from \sphinxhref{https://github.com/openmaptiles/openmaptiles}{OpenMapTiles}%
\begin{footnote}[19]\sphinxAtStartFootnote
\sphinxnolinkurl{https://github.com/openmaptiles/openmaptiles}
%
\end{footnote}, published under a \sphinxhref{https://github.com/openmaptiles/openmaptiles/blob/master/LICENSE.md}{CC\sphinxhyphen{}BY 4.0
design license}%
\begin{footnote}[20]\sphinxAtStartFootnote
\sphinxnolinkurl{https://github.com/openmaptiles/openmaptiles/blob/master/LICENSE.md}
%
\end{footnote}.

\item {} 
\sphinxAtStartPar
The aviation maps contain data from \sphinxhref{http://www.openaip.net}{openAIP}%
\begin{footnote}[21]\sphinxAtStartFootnote
\sphinxnolinkurl{http://www.openaip.net}
%
\end{footnote},
licensed under a \sphinxhref{https://creativecommons.org/licenses/by-nc-sa/3.0/}{CC BY\sphinxhyphen{}NC\sphinxhyphen{}SA license}%
\begin{footnote}[22]\sphinxAtStartFootnote
\sphinxnolinkurl{https://creativecommons.org/licenses/by-nc-sa/3.0/}
%
\end{footnote}.

\item {} 
\sphinxAtStartPar
The aviation maps contain data from \sphinxhref{https://www.openflightmaps.org/}{open flightmaps}%
\begin{footnote}[23]\sphinxAtStartFootnote
\sphinxnolinkurl{https://www.openflightmaps.org/}
%
\end{footnote}, licensed under the \sphinxhref{https://www.openflightmaps.org/live/downloads/20150306-LCN.pdf}{OFMA General Users´
License}%
\begin{footnote}[24]\sphinxAtStartFootnote
\sphinxnolinkurl{https://www.openflightmaps.org/live/downloads/20150306-LCN.pdf}
%
\end{footnote}.

\end{itemize}


\section{Software and data included in the program}
\label{\detokenize{04-appendix/license_3rdParty:software-and-data-included-in-the-program}}
\sphinxAtStartPar
Depending on platform and configuration, the following components might be
included in the installation of \sphinxstylestrong{Enroute Flight Navigation}.
\begin{itemize}
\item {} 
\sphinxAtStartPar
\sphinxhref{https://github.com/adobe-type-tools/agl-aglfn}{Adobe Glyph List For New Fonts}%
\begin{footnote}[25]\sphinxAtStartFootnote
\sphinxnolinkurl{https://github.com/adobe-type-tools/agl-aglfn}
%
\end{footnote}. BSD 3\sphinxhyphen{}Clause “New” or “Revised” License.

\item {} 
\sphinxAtStartPar
\sphinxhref{http://angleproject.org/}{ANGLE Library}%
\begin{footnote}[26]\sphinxAtStartFootnote
\sphinxnolinkurl{http://angleproject.org/}
%
\end{footnote}. BSD 3\sphinxhyphen{}clause “New” or “Revised” License.

\item {} 
\sphinxAtStartPar
ANGLE: Array Bounds Clamper for WebKit. BSD 2\sphinxhyphen{}clause “Simplified” License.

\item {} 
\sphinxAtStartPar
ANGLE: Khronos Headers. MIT License.

\item {} 
\sphinxAtStartPar
ANGLE: Murmurhash. Public Domain.

\item {} 
\sphinxAtStartPar
ANGLE: Systeminfo. BSD 2\sphinxhyphen{}clause “Simplified” License.

\item {} 
\sphinxAtStartPar
ANGLE: trace\_event. BSD 3\sphinxhyphen{}clause “New” or “Revised” License.

\item {} 
\sphinxAtStartPar
\sphinxhref{http://www.freetype.org}{Anti\sphinxhyphen{}aliasing rasterizer from FreeType 2}%
\begin{footnote}[27]\sphinxAtStartFootnote
\sphinxnolinkurl{http://www.freetype.org}
%
\end{footnote}. Freetype Project License or GNU General Public License v2.0 only.

\item {} 
\sphinxAtStartPar
\sphinxhref{https://www.gnome.org/fonts/}{Bitstream Vera Font}%
\begin{footnote}[28]\sphinxAtStartFootnote
\sphinxnolinkurl{https://www.gnome.org/fonts/}
%
\end{footnote}. Bitstream Vera Font License.

\item {} 
\sphinxAtStartPar
\sphinxhref{https://github.com/raptor/clip2tri}{Clip2Tri Polygon Triangulation Library}%
\begin{footnote}[29]\sphinxAtStartFootnote
\sphinxnolinkurl{https://github.com/raptor/clip2tri}
%
\end{footnote}. MIT License.

\item {} 
\sphinxAtStartPar
\sphinxhref{http://www.angusj.com/delphi/clipper.php}{Clipper Polygon Clipping Library}%
\begin{footnote}[30]\sphinxAtStartFootnote
\sphinxnolinkurl{http://www.angusj.com/delphi/clipper.php}
%
\end{footnote}. Boost Software License 1.0.

\item {} 
\sphinxAtStartPar
Cocoa Platform Plugin. BSD 3\sphinxhyphen{}clause “New” or “Revised” License.

\item {} 
\sphinxAtStartPar
Cycle. MIT License.

\item {} 
\sphinxAtStartPar
\sphinxhref{http://zlib.net/}{Data Compression Library (zlib)}%
\begin{footnote}[31]\sphinxAtStartFootnote
\sphinxnolinkurl{http://zlib.net/}
%
\end{footnote}. zlib License.

\item {} 
\sphinxAtStartPar
\sphinxhref{https://dejavu-fonts.github.io/}{DejaVu Fonts}%
\begin{footnote}[32]\sphinxAtStartFootnote
\sphinxnolinkurl{https://dejavu-fonts.github.io/}
%
\end{footnote}. Bitstream Vera Font License.

\item {} 
\sphinxAtStartPar
\sphinxhref{https://github.com/mapbox/earcut.hpp}{Earcut Polygon Triangulation Library}%
\begin{footnote}[33]\sphinxAtStartFootnote
\sphinxnolinkurl{https://github.com/mapbox/earcut.hpp}
%
\end{footnote}. ISC License.

\item {} 
\sphinxAtStartPar
\sphinxhref{https://earth-info.nga.mil}{Earth Gravitational Model}%
\begin{footnote}[34]\sphinxAtStartFootnote
\sphinxnolinkurl{https://earth-info.nga.mil}
%
\end{footnote}. Public Domain.

\item {} 
\sphinxAtStartPar
\sphinxhref{http://robertpenner.com/easing/}{Easing Equations by Robert Penner}%
\begin{footnote}[35]\sphinxAtStartFootnote
\sphinxnolinkurl{http://robertpenner.com/easing/}
%
\end{footnote}. BSD 3\sphinxhyphen{}clause “New” or “Revised” License.

\item {} 
\sphinxAtStartPar
\sphinxhref{https://github.com/google/double-conversion}{Efficient Binary\sphinxhyphen{}Decimal and Decimal\sphinxhyphen{}Binary Conversion Routines for IEEE Doubles}%
\begin{footnote}[36]\sphinxAtStartFootnote
\sphinxnolinkurl{https://github.com/google/double-conversion}
%
\end{footnote}. BSD 3\sphinxhyphen{}clause “New” or “Revised” License.

\item {} 
\sphinxAtStartPar
forkfd. MIT License.

\item {} 
\sphinxAtStartPar
\sphinxhref{https://github.com/freebsd/freebsd/}{FreeBSD strtoll and strtoull}%
\begin{footnote}[37]\sphinxAtStartFootnote
\sphinxnolinkurl{https://github.com/freebsd/freebsd/}
%
\end{footnote}. BSD 3\sphinxhyphen{}clause “New” or “Revised” License.

\item {} 
\sphinxAtStartPar
\sphinxhref{http://www.freetype.org}{Freetype 2}%
\begin{footnote}[38]\sphinxAtStartFootnote
\sphinxnolinkurl{http://www.freetype.org}
%
\end{footnote}. Freetype Project License or GNU General Public License v2.0 only.

\item {} 
\sphinxAtStartPar
\sphinxhref{http://www.freetype.org}{Freetype 2 \sphinxhyphen{} Bitmap Distribution Format (BDF) support}%
\begin{footnote}[39]\sphinxAtStartFootnote
\sphinxnolinkurl{http://www.freetype.org}
%
\end{footnote}. MIT License.

\item {} 
\sphinxAtStartPar
\sphinxhref{http://www.freetype.org}{Freetype 2 \sphinxhyphen{} Portable Compiled Format (PCF) support}%
\begin{footnote}[40]\sphinxAtStartFootnote
\sphinxnolinkurl{http://www.freetype.org}
%
\end{footnote}. MIT License.

\item {} 
\sphinxAtStartPar
\sphinxhref{http://www.freetype.org}{Freetype 2 \sphinxhyphen{} zlib}%
\begin{footnote}[41]\sphinxAtStartFootnote
\sphinxnolinkurl{http://www.freetype.org}
%
\end{footnote}. zlib License.

\item {} 
\sphinxAtStartPar
\sphinxhref{https://github.com/mapbox/geosimplify-js}{geosimplify\sphinxhyphen{}js polyline simplification library}%
\begin{footnote}[42]\sphinxAtStartFootnote
\sphinxnolinkurl{https://github.com/mapbox/geosimplify-js}
%
\end{footnote}. geosimplify\sphinxhyphen{}js License.

\item {} 
\sphinxAtStartPar
\sphinxhref{https://github.com/google/fonts}{Google Fonts}%
\begin{footnote}[43]\sphinxAtStartFootnote
\sphinxnolinkurl{https://github.com/google/fonts}
%
\end{footnote}. Apache 2 License.

\item {} 
\sphinxAtStartPar
\sphinxhref{https://gradle.org}{Gradle wrapper}%
\begin{footnote}[44]\sphinxAtStartFootnote
\sphinxnolinkurl{https://gradle.org}
%
\end{footnote}. Apache License 2.0.

\item {} 
\sphinxAtStartPar
\sphinxhref{https://github.com/microsoft/GSL}{Guidelines Support Library}%
\begin{footnote}[45]\sphinxAtStartFootnote
\sphinxnolinkurl{https://github.com/microsoft/GSL}
%
\end{footnote}. MIT License.

\item {} 
\sphinxAtStartPar
HarfBuzz. MIT License.

\item {} 
\sphinxAtStartPar
\sphinxhref{http://harfbuzz.org}{HarfBuzz\sphinxhyphen{}NG}%
\begin{footnote}[46]\sphinxAtStartFootnote
\sphinxnolinkurl{http://harfbuzz.org}
%
\end{footnote}. MIT License.

\item {} 
\sphinxAtStartPar
\sphinxhref{https://wiki.linuxfoundation.org/accessibility/iaccessible2/}{IAccessible2 IDL Specification}%
\begin{footnote}[47]\sphinxAtStartFootnote
\sphinxnolinkurl{https://wiki.linuxfoundation.org/accessibility/iaccessible2/}
%
\end{footnote}. BSD 3\sphinxhyphen{}clause “New” or “Revised” License.

\item {} 
\sphinxAtStartPar
\sphinxhref{https://trac.webkit.org/wiki/JavaScriptCore}{JavaScriptCore Macro Assembler}%
\begin{footnote}[48]\sphinxAtStartFootnote
\sphinxnolinkurl{https://trac.webkit.org/wiki/JavaScriptCore}
%
\end{footnote}. BSD 2\sphinxhyphen{}clause “Simplified” License.

\item {} 
\sphinxAtStartPar
\sphinxhref{https://github.com/KDAB/KDSingleApplication}{KDAB’s helper class for single\sphinxhyphen{}instance policy applications}%
\begin{footnote}[49]\sphinxAtStartFootnote
\sphinxnolinkurl{https://github.com/KDAB/KDSingleApplication}
%
\end{footnote}. MIT License.

\item {} 
\sphinxAtStartPar
\sphinxhref{https://www.freedesktop.org/wiki/Software/dbus/}{libdus\sphinxhyphen{}1 headers}%
\begin{footnote}[50]\sphinxAtStartFootnote
\sphinxnolinkurl{https://www.freedesktop.org/wiki/Software/dbus/}
%
\end{footnote}. Academic Free License v2.1, or GNU General Public License v2.0 or later.

\item {} 
\sphinxAtStartPar
\sphinxhref{http://libjpeg-turbo.virtualgl.org/}{LibJPEG\sphinxhyphen{}turbo}%
\begin{footnote}[51]\sphinxAtStartFootnote
\sphinxnolinkurl{http://libjpeg-turbo.virtualgl.org/}
%
\end{footnote}. Independent JPEG Group License.

\item {} 
\sphinxAtStartPar
\sphinxhref{http://www.libpng.org/pub/png/libpng.html}{LibPNG}%
\begin{footnote}[52]\sphinxAtStartFootnote
\sphinxnolinkurl{http://www.libpng.org/pub/png/libpng.html}
%
\end{footnote}. libpng License and PNG Reference Library version 2.

\item {} 
\sphinxAtStartPar
\sphinxhref{https://www.kernel.org}{Linux Performance Events}%
\begin{footnote}[53]\sphinxAtStartFootnote
\sphinxnolinkurl{https://www.kernel.org}
%
\end{footnote}. GNU General Public License v2.0 only with Linux Syscall Note.

\item {} 
\sphinxAtStartPar
\sphinxhref{https://github.com/google/material-design-icons}{Material Design Icons}%
\begin{footnote}[54]\sphinxAtStartFootnote
\sphinxnolinkurl{https://github.com/google/material-design-icons}
%
\end{footnote}. Apache License.

\item {} 
\sphinxAtStartPar
MD4. Public Domain.

\item {} 
\sphinxAtStartPar
\sphinxhref{https://github.com/mity/md4c}{MD4C}%
\begin{footnote}[55]\sphinxAtStartFootnote
\sphinxnolinkurl{https://github.com/mity/md4c}
%
\end{footnote}. MIT License.

\item {} 
\sphinxAtStartPar
MD5. Public Domain.

\item {} 
\sphinxAtStartPar
\sphinxhref{https://github.com/nnaumenko/metaf}{Metaf library}%
\begin{footnote}[56]\sphinxAtStartFootnote
\sphinxnolinkurl{https://github.com/nnaumenko/metaf}
%
\end{footnote}. MIT License.

\item {} 
\sphinxAtStartPar
Native Style for Android. Apache License 2.0.

\item {} 
\sphinxAtStartPar
\sphinxhref{https://www.khronos.org/}{OpenGL ES 2 Headers}%
\begin{footnote}[57]\sphinxAtStartFootnote
\sphinxnolinkurl{https://www.khronos.org/}
%
\end{footnote}. MIT License.

\item {} 
\sphinxAtStartPar
\sphinxhref{https://www.khronos.org/}{OpenGL Headers}%
\begin{footnote}[58]\sphinxAtStartFootnote
\sphinxnolinkurl{https://www.khronos.org/}
%
\end{footnote}. MIT License.

\item {} 
\sphinxAtStartPar
\sphinxhref{https://www.openssl.org}{openSSL library}%
\begin{footnote}[59]\sphinxAtStartFootnote
\sphinxnolinkurl{https://www.openssl.org}
%
\end{footnote}. Apache 2 License.

\item {} 
\sphinxAtStartPar
\sphinxhref{https://github.com/maputnik/osm-liberty}{OSM Liberty}%
\begin{footnote}[60]\sphinxAtStartFootnote
\sphinxnolinkurl{https://github.com/maputnik/osm-liberty}
%
\end{footnote}. BSD License.

\item {} 
\sphinxAtStartPar
\sphinxhref{http://www.pcre.org/}{PCRE2}%
\begin{footnote}[61]\sphinxAtStartFootnote
\sphinxnolinkurl{http://www.pcre.org/}
%
\end{footnote}. BSD 3\sphinxhyphen{}clause “New” or “Revised” License.

\item {} 
\sphinxAtStartPar
\sphinxhref{http://www.pcre.org/}{PCRE2 \sphinxhyphen{} Stack\sphinxhyphen{}less Just\sphinxhyphen{}In\sphinxhyphen{}Time Compiler}%
\begin{footnote}[62]\sphinxAtStartFootnote
\sphinxnolinkurl{http://www.pcre.org/}
%
\end{footnote}. BSD 2\sphinxhyphen{}clause “Simplified” License.

\item {} 
\sphinxAtStartPar
\sphinxhref{http://www.pixman.org/}{Pixman}%
\begin{footnote}[63]\sphinxAtStartFootnote
\sphinxnolinkurl{http://www.pixman.org/}
%
\end{footnote}. MIT License.

\item {} 
\sphinxAtStartPar
\sphinxhref{http://code.google.com/p/poly2tri/}{Poly2Tri Polygon Triangulation Library}%
\begin{footnote}[64]\sphinxAtStartFootnote
\sphinxnolinkurl{http://code.google.com/p/poly2tri/}
%
\end{footnote}. BSD 3\sphinxhyphen{}clause “New” or “Revised” License.

\item {} 
\sphinxAtStartPar
QEventDispatcher on macOS. BSD 3\sphinxhyphen{}clause “New” or “Revised” License.

\item {} 
\sphinxAtStartPar
\sphinxhref{https://github.com/nitroshare/qhttpengine}{QHttpEngine}%
\begin{footnote}[65]\sphinxAtStartFootnote
\sphinxnolinkurl{https://github.com/nitroshare/qhttpengine}
%
\end{footnote}. MIT License.

\item {} 
\sphinxAtStartPar
\sphinxhref{https://qt.io}{Qt Toolkit, Libraries and Modules}%
\begin{footnote}[66]\sphinxAtStartFootnote
\sphinxnolinkurl{https://qt.io}
%
\end{footnote}. GNU General Public License v3.0.

\item {} 
\sphinxAtStartPar
\sphinxhref{http://www.dominik-reichl.de/projects/csha1/}{Secure Hash Algorithm SHA\sphinxhyphen{}1}%
\begin{footnote}[67]\sphinxAtStartFootnote
\sphinxnolinkurl{http://www.dominik-reichl.de/projects/csha1/}
%
\end{footnote}. Public Domain.

\item {} 
\sphinxAtStartPar
Secure Hash Algorithm SHA\sphinxhyphen{}3 \sphinxhyphen{} brg\_endian. BSD 2\sphinxhyphen{}clause “Simplified” License.

\item {} 
\sphinxAtStartPar
Secure Hash Algorithm SHA\sphinxhyphen{}3 \sphinxhyphen{} Keccak. Creative Commons Zero v1.0 Universal.

\item {} 
\sphinxAtStartPar
Secure Hash Algorithms SHA\sphinxhyphen{}384 and SHA\sphinxhyphen{}512. BSD 3\sphinxhyphen{}clause “New” or “Revised” License.

\item {} 
\sphinxAtStartPar
\sphinxhref{https://angularjs.org/}{Shadow values from Angular Material}%
\begin{footnote}[68]\sphinxAtStartFootnote
\sphinxnolinkurl{https://angularjs.org/}
%
\end{footnote}. MIT License.

\item {} 
\sphinxAtStartPar
Smooth Scaling Algorithm. BSD 2\sphinxhyphen{}clause “Simplified” License and Imlib2 License.

\item {} 
\sphinxAtStartPar
\sphinxhref{https://www.sqlite.org/}{SQLite}%
\begin{footnote}[69]\sphinxAtStartFootnote
\sphinxnolinkurl{https://www.sqlite.org/}
%
\end{footnote}. Public Domain.

\item {} 
\sphinxAtStartPar
\sphinxhref{http://www.color.org/}{sRGB color profile icc file}%
\begin{footnote}[70]\sphinxAtStartFootnote
\sphinxnolinkurl{http://www.color.org/}
%
\end{footnote}. International Color Consortium License.

\item {} 
\sphinxAtStartPar
\sphinxhref{https://github.com/buelowp/sunset}{Sunset library}%
\begin{footnote}[71]\sphinxAtStartFootnote
\sphinxnolinkurl{https://github.com/buelowp/sunset}
%
\end{footnote}. GNU General Public License v2.0.

\item {} 
\sphinxAtStartPar
\sphinxhref{http://tango.freedesktop.org/Tango\_Desktop\_Project}{Tango Icons}%
\begin{footnote}[72]\sphinxAtStartFootnote
\sphinxnolinkurl{http://tango.freedesktop.org/Tango\_Desktop\_Project}
%
\end{footnote}. Public Domain.

\item {} 
\sphinxAtStartPar
\sphinxhref{https://www.deviantart.com/darkobra/art/Tango-Weather-Icon-Pack-98024429}{Tango Weather Icon Pack by Darkobra}%
\begin{footnote}[73]\sphinxAtStartFootnote
\sphinxnolinkurl{https://www.deviantart.com/darkobra/art/Tango-Weather-Icon-Pack-98024429}
%
\end{footnote}. Public Domain.

\item {} 
\sphinxAtStartPar
Text Codec: EUC\sphinxhyphen{}JP. BSD 2\sphinxhyphen{}clause “Simplified” License.

\item {} 
\sphinxAtStartPar
Text Codec: EUC\sphinxhyphen{}KR. BSD 2\sphinxhyphen{}clause “Simplified” License.

\item {} 
\sphinxAtStartPar
Text Codec: GBK. BSD 2\sphinxhyphen{}clause “Simplified” License.

\item {} 
\sphinxAtStartPar
Text Codec: ISO 2022\sphinxhyphen{}JP (JIS). BSD 2\sphinxhyphen{}clause “Simplified” License.

\item {} 
\sphinxAtStartPar
Text Codec: Shift\sphinxhyphen{}JIS. BSD 2\sphinxhyphen{}clause “Simplified” License.

\item {} 
\sphinxAtStartPar
Text Codec: TSCII. BSD 2\sphinxhyphen{}clause “Simplified” License.

\item {} 
\sphinxAtStartPar
Text Codecs: Big5, Big5\sphinxhyphen{}HKSCS. BSD 2\sphinxhyphen{}clause “Simplified” License.

\item {} 
\sphinxAtStartPar
\sphinxhref{http://publicsuffix.org/}{The Public Suffix List}%
\begin{footnote}[74]\sphinxAtStartFootnote
\sphinxnolinkurl{http://publicsuffix.org/}
%
\end{footnote}. Mozilla Public License 2.0.

\item {} 
\sphinxAtStartPar
\sphinxhref{https://github.com/intel/tinycbor}{TinyCBOR}%
\begin{footnote}[75]\sphinxAtStartFootnote
\sphinxnolinkurl{https://github.com/intel/tinycbor}
%
\end{footnote}. MIT License.

\item {} 
\sphinxAtStartPar
\sphinxhref{https://www.unicode.org/ucd/}{Unicode Character Database (UCD)}%
\begin{footnote}[76]\sphinxAtStartFootnote
\sphinxnolinkurl{https://www.unicode.org/ucd/}
%
\end{footnote}. Unicode License Agreement \sphinxhyphen{} Data Files and Software (2016).

\item {} 
\sphinxAtStartPar
\sphinxhref{http://cldr.unicode.org/}{Unicode Common Locale Data Repository (CLDR)}%
\begin{footnote}[77]\sphinxAtStartFootnote
\sphinxnolinkurl{http://cldr.unicode.org/}
%
\end{footnote}. Unicode License Agreement \sphinxhyphen{} Data Files and Software (2016).

\item {} 
\sphinxAtStartPar
\sphinxhref{http://valgrind.org/}{Valgrind}%
\begin{footnote}[78]\sphinxAtStartFootnote
\sphinxnolinkurl{http://valgrind.org/}
%
\end{footnote}. BSD 4\sphinxhyphen{}clause “Original” or “Old” License.

\item {} 
\sphinxAtStartPar
\sphinxhref{https://www.khronos.org/}{Vulkan API Registry}%
\begin{footnote}[79]\sphinxAtStartFootnote
\sphinxnolinkurl{https://www.khronos.org/}
%
\end{footnote}. MIT License.

\item {} 
\sphinxAtStartPar
\sphinxhref{https://github.com/GPUOpen-LibrariesAndSDKs/VulkanMemoryAllocator}{Vulkan Memory Allocator}%
\begin{footnote}[80]\sphinxAtStartFootnote
\sphinxnolinkurl{https://github.com/GPUOpen-LibrariesAndSDKs/VulkanMemoryAllocator}
%
\end{footnote}. MIT License.

\item {} 
\sphinxAtStartPar
\sphinxhref{https://webgradients.com/}{WebGradients}%
\begin{footnote}[81]\sphinxAtStartFootnote
\sphinxnolinkurl{https://webgradients.com/}
%
\end{footnote}. MIT License.

\item {} 
\sphinxAtStartPar
Wintab API. LCS\sphinxhyphen{}Telegraphics License.

\item {} 
\sphinxAtStartPar
\sphinxhref{https://www.x.org/}{X Server helper}%
\begin{footnote}[82]\sphinxAtStartFootnote
\sphinxnolinkurl{https://www.x.org/}
%
\end{footnote}. X11 License and Historical Permission Notice and Disclaimer.

\item {} 
\sphinxAtStartPar
\sphinxhref{https://xcb.freedesktop.org/}{XCB\sphinxhyphen{}XInput}%
\begin{footnote}[83]\sphinxAtStartFootnote
\sphinxnolinkurl{https://xcb.freedesktop.org/}
%
\end{footnote}. MIT License.

\item {} 
\sphinxAtStartPar
XSVG. Historical Permission Notice and Disclaimer \sphinxhyphen{} sell variant.

\end{itemize}


\chapter{Technical Notes}
\label{\detokenize{04-appendix/technical:technical-notes}}\label{\detokenize{04-appendix/technical::doc}}

\section{Platform notes}
\label{\detokenize{04-appendix/technical:platform-notes}}

\subsection{Android}
\label{\detokenize{04-appendix/technical:android}}
\sphinxAtStartPar
Wi\sphinxhyphen{}Fi locking
\begin{quote}

\sphinxAtStartPar
When running on Android, the app acquires a Wi\sphinxhyphen{}Fi lock as soon as the app
receives heartbeat messages from one of the channels where it listens for
traffic receivers.  The lock is released when the messages no longer arrive.
\end{quote}


\section{Traffic Receiver}
\label{\detokenize{04-appendix/technical:traffic-receiver}}

\subsection{Communication}
\label{\detokenize{04-appendix/technical:communication}}
\sphinxAtStartPar
\sphinxstylestrong{Enroute Flight Navigation} expects that the traffic receiver deploys a WLAN
network via Wi\sphinxhyphen{}Fi and publishes traffic data via that network.  In order to
support a wide range of devices, including flight simulators, the app listens to
several network addresses simultaneously and understands a variety of protocols.

\sphinxAtStartPar
\sphinxstylestrong{Enroute Flight Navigation} watches the following data channels, in order of
preference.
\begin{itemize}
\item {} 
\sphinxAtStartPar
A TCP connection to port 2000 at the IP addresses 192.168.1.1, where the app
expects a stream of FLARM/NMEA sentences.

\item {} 
\sphinxAtStartPar
A TCP connection to port 2000 at the IP addresses 192.168.10.1, where the app
expects a stream of FLARM/NMEA sentences.

\item {} 
\sphinxAtStartPar
A UDP connection to port 4000, where the app expects datagrams in GDL90 or
XGPS format.

\item {} 
\sphinxAtStartPar
A UDP connection to port 49002, where the app expects datagrams in GDL90 or
XGPS format.

\end{itemize}

\sphinxAtStartPar
\sphinxstylestrong{Enroute Flight Navigation} expects traffic data in the following formats.
\begin{itemize}
\item {} 
\sphinxAtStartPar
FLARM/NMEA sentences must conform to the specification outlined in the
document FTD\sphinxhyphen{}012 \sphinxhref{https://flarm.com/support/manuals-documents/}{Data Port Interface Control Document (ICD)}%
\begin{footnote}[84]\sphinxAtStartFootnote
\sphinxnolinkurl{https://flarm.com/support/manuals-documents/}
%
\end{footnote}, Version 7.13, as published
by \sphinxhref{https://flarm.com/}{FLARM Technology Ltd.}%
\begin{footnote}[85]\sphinxAtStartFootnote
\sphinxnolinkurl{https://flarm.com/}
%
\end{footnote}.

\item {} 
\sphinxAtStartPar
Datagrams in GDL90 format must conform to the \sphinxhref{https://www.faa.gov/nextgen/programs/adsb/archival/media/gdl90\_public\_icd\_reva.pdf}{GDL 90 Data Interface
Specification}%
\begin{footnote}[86]\sphinxAtStartFootnote
\sphinxnolinkurl{https://www.faa.gov/nextgen/programs/adsb/archival/media/gdl90\_public\_icd\_reva.pdf}
%
\end{footnote}.

\item {} 
\sphinxAtStartPar
Datagrams in XGPS format must conform to the format specified on the
\sphinxhref{https://www.foreflight.com/support/network-gps/}{ForeFlight Web site}%
\begin{footnote}[87]\sphinxAtStartFootnote
\sphinxnolinkurl{https://www.foreflight.com/support/network-gps/}
%
\end{footnote}.

\end{itemize}


\subsection{Known issues}
\label{\detokenize{04-appendix/technical:known-issues}}
\sphinxAtStartPar
The GDL90 protocol has a number of shortcomings, and we recommend to use
FLARM/NMEA whenever possible.  We are aware of the following issues.

\sphinxAtStartPar
Altitude measurements
\begin{quote}

\sphinxAtStartPar
According to the GDL90 Specification, the ownship geometric height is reported
as height above WGS\sphinxhyphen{}84 ellipsoid.  There are however many devices on the
market that wrongly report height above main sea level.  Different apps have
different strategies to deal with these shortcomings.
\begin{itemize}
\item {} 
\sphinxAtStartPar
\sphinxstylestrong{Enroute Flight Navigation} as well as the app Skydemon expect that
traffic receivers comply with the GDL90 Specification.

\item {} 
\sphinxAtStartPar
ForeFlight has extended the GDL90 Specification so that traffic receivers
can indicate if they comply with the specification or not.

\item {} 
\sphinxAtStartPar
Many other apps expect wrong GDL90 implementations and interpret the
geometric height has height above main sea level.

\end{itemize}
\end{quote}

\sphinxAtStartPar
MODE\sphinxhyphen{}S traffic
\begin{quote}

\sphinxAtStartPar
Most traffic receivers see traffic equipped with MODE\sphinxhyphen{}S transponders and can
give an estimate for the distance to the traffic.  They are, however, unable
to obtain the precise traffic position.  Unlike FLARM/NMEA, the GDL90
Specification does not support traffic factors whose position is unknown.
Different devices implement different workarounds.
\begin{itemize}
\item {} 
\sphinxAtStartPar
Stratux devices generate a ring of eight virtual targets around the own
position.  These targets are named “Mode S”.

\item {} 
\sphinxAtStartPar
Air Avioncs devices do the same, but only with one target.

\item {} 
\sphinxAtStartPar
Other devices create a virtual target, either at the ownship position or at
the north pole and abuse the field “Navigation Accuracy Category for
Position” to give the approximate position to the target.

\end{itemize}

\sphinxAtStartPar
\sphinxstylestrong{Enroute Flight Navigation} has special provisions for handling targets
called “Mode S”, but users should expect that this workaround is not perfect.
\end{quote}


\subsection{ForeFlight Broadcast}
\label{\detokenize{04-appendix/technical:foreflight-broadcast}}
\sphinxAtStartPar
Following the standards established by the app ForeFlight, \sphinxstylestrong{Enroute Flight
Navigation} broadcasts a UDP message on port 63093 every 5 seconds while the
app is running in the foreground.  This message allows devices to discover
Enroute’s IP address, which can be used as the target of UDP unicast messages.
This broadcast will be a JSON message, with at least these fields:

\begin{sphinxVerbatim}[commandchars=\\\{\}]
\PYG{p}{\PYGZob{}}
   \PYG{n+nt}{\PYGZdq{}App\PYGZdq{}}\PYG{p}{:}\PYG{l+s+s2}{\PYGZdq{}Enroute Flight Navigation\PYGZdq{}}\PYG{p}{,}
   \PYG{n+nt}{\PYGZdq{}GDL90\PYGZdq{}}\PYG{p}{:}\PYG{p}{\PYGZob{}}
      \PYG{n+nt}{\PYGZdq{}port\PYGZdq{}}\PYG{p}{:}\PYG{l+m+mi}{4000}
   \PYG{p}{\PYGZcb{}}
\PYG{p}{\PYGZcb{}}
\end{sphinxVerbatim}

\sphinxAtStartPar
The GDL90 “port” field is currently 4000, but might change in the future.



\renewcommand{\indexname}{Index}
\footnotesize\raggedright\printindex
\end{document}